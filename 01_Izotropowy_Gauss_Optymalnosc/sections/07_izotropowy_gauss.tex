\clearpage
%% ============================================================
\section{Co to jest izotropowy rozkład Gaussa?}
%% ============================================================

\begin{definition}[Izotropowy rozkład Gaussa]
Wektor losowy $\mathbf{z} \in \mathbb{R}^K$ ma \textbf{izotropowy rozkład Gaussa},
jeśli:
\begin{equation}
\mathbf{z} \sim \mathcal{N}(\mathbf{0}, \mathbf{I}_K),
\quad \text{tzn.} \quad
p(\mathbf{z}) = \frac{1}{(2\pi)^{K/2}} \exp\!\left(-\frac{1}{2}\|\mathbf{z}\|^2\right)
\label{eq:isotropic}
\end{equation}
gdzie $\mathbf{I}_K$ jest macierzą jednostkową $K \times K$.
\end{definition}

\textbf{Izotropia} oznacza, że rozkład wygląda identycznie we \textit{wszystkich} kierunkach:
\begin{itemize}
  \item Macierz kowariancji: $\mathrm{Cov}(\mathbf{z}) = \mathbf{I}_K$
        (wszystkie wartości własne $\lambda_1 = \lambda_2 = \cdots = \lambda_K = 1$),
  \item Izolinie gęstości to \textbf{hipersferery} $\|\mathbf{z}\| = r$,
  \item Żaden wymiar nie jest ``ważniejszy'' od innego.
\end{itemize}

\begin{figure}[H]
\centering
\includegraphics[width=\textwidth]{figures/isotropic_gaussian_3d.pdf}
\caption{Gęstość izotropowego Gaussa $\mathcal{N}(\mathbf{0}, \mathbf{I}_2)$.
\textbf{Lewo}: powierzchnia 3D. \textbf{Prawo}: izolinie tworzą koncentryczne okręgi —
cecha izotropii.}
\label{fig:3d}
\end{figure}

\subsection{Izotropowy vs.\ anizotropowy vs.\ kolaps}

\begin{figure}[H]
\centering
\includegraphics[width=\textwidth]{figures/isotropic_vs_anisotropic_2d.pdf}
\caption{Trzy scenariusze embeddingów 2D. \textbf{Lewo}: izotropowy Gauss —
równomierne rozłożenie we wszystkich kierunkach.
\textbf{Środek}: anizotropowy — informacja skoncentrowana wzdłuż jednej osi.
\textbf{Prawo}: kolaps — brak jakiejkolwiek informacji.
Czerwone strzałki to wektory własne macierzy kowariancji.}
\label{fig:2d}
\end{figure}

