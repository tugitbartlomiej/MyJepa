\documentclass[11pt,a4paper]{article}
\usepackage[utf8]{inputenc}
\usepackage[T1]{fontenc}
\usepackage[polish]{babel}
\usepackage{amsmath,amssymb,amsthm}
\usepackage{mathtools}
\usepackage{graphicx}
\usepackage[left=0.3cm, right=0.3cm, top=0.3cm, bottom=1.5cm]{geometry}
\usepackage{fancyhdr}
\pagestyle{fancy}
\fancyhf{}
\renewcommand{\headrulewidth}{0pt}
\fancyfoot[C]{\thepage}
\usepackage{xcolor}
\usepackage{tcolorbox}
\usepackage{hyperref}
\usepackage{booktabs}
\usepackage{enumitem}
\usepackage{caption}
\usepackage{float}
\usepackage{placeins}  % for \FloatBarrier

% Theorem environments
\theoremstyle{definition}
\newtheorem{definition}{Definicja}[section]
\newtheorem{theorem}{Twierdzenie}[section]
\newtheorem{lemma}{Lemat}[section]
\newtheorem{remark}{Uwaga}[section]

% Custom colors
\definecolor{lejepaBlue}{RGB}{33,150,243}
\definecolor{lejepaRed}{RGB}{211,47,47}
\definecolor{lejepaGreen}{RGB}{76,175,80}

% tcolorbox styles
\tcbuselibrary{theorems,skins,breakable}

\newtcolorbox{keyinsight}[1][]{
  colback=lejepaBlue!8,
  colframe=lejepaBlue!80,
  fonttitle=\bfseries,
  title={#1},
  breakable,
  sharp corners=south,
}

\newtcolorbox{warningbox}[1][]{
  colback=lejepaRed!8,
  colframe=lejepaRed!80,
  fonttitle=\bfseries,
  title={#1},
  breakable,
}

\title{%
  \textbf{Dlaczego izotropowy rozkład Gaussa jest optymalny\\
  dla embeddingów w LeJEPA?}\\[0.5em]
  \large Teoria i intuicja z artykułu Balestriero \& LeCun (2025)
}
\author{Notatki do artykułu \textit{LeJEPA: Provable and Scalable SSL Without the Heuristics}}
\date{\today}

\setlength{\textfloatsep}{10pt plus 2pt minus 4pt}     % odstęp tekst-figura (góra/dół)
\setlength{\floatsep}{8pt plus 2pt minus 4pt}         % odstęp między figurami
\setlength{\intextsep}{8pt plus 2pt minus 4pt}        % odstęp dla figur [h]
\renewcommand{\topfraction}{0.95}      % max 95% strony na figury u góry
\renewcommand{\bottomfraction}{0.95}   % max 95% strony na figury u dołu
\renewcommand{\textfraction}{0.05}     % min 5% strony na tekst
\renewcommand{\floatpagefraction}{0.9} % min 90% wypełnienia strony z samymi figurami
\setcounter{topnumber}{4}              % max 4 figury u góry strony
\setcounter{bottomnumber}{4}           % max 4 figury u dołu strony
\setcounter{totalnumber}{8}            % max 8 figur na stronie

\begin{document}
\raggedbottom  % nie rozciągaj pustych przestrzeni na stronach
\maketitle

\begin{abstract}
Ten dokument wyjaśnia kluczowy wynik teoretyczny artykułu LeJEPA:
\textbf{izotropowy rozkład Gaussa} $\mathcal{N}(\mathbf{0}, \mathbf{I})$
jest \textbf{jedynym optymalnym rozkładem} embeddingów, który minimalizuje
ryzyko downstream dla \textit{dowolnego} zadania klasyfikacji/regresji.
Przedstawiamy dowody dla linear probe i nieliniowych metod (k-NN, kernel),
intuicję geometryczną oraz wizualizacje.
\end{abstract}

\tableofcontents

\clearpage
%% ============================================================
\section{Ogólny obraz: co budujemy i po co?}
\label{sec:big_picture}
%% ============================================================

Zanim wejdziemy w matematykę, ustalmy \textbf{co jest czym}.
W tym projekcie mamy do czynienia z trzema rzeczami, które łatwo pomylić:

\begin{center}
\renewcommand{\arraystretch}{1.8}
\begin{tabular}{cp{10cm}}
\toprule
\textbf{Nazwa} & \textbf{Czym jest?} \\
\midrule
\textbf{ViT} \newline (Vision Transformer) &
\textbf{Sieć neuronowa} -- architektura encodera.
Zwykła maszyna: obraz wchodzi, wektor liczb (embedding) wychodzi.
ViT sam z siebie \textit{nie wie}, czego się uczyć -- potrzebuje metody treningu. \\
\textbf{LeJEPA} \newline (metoda treningu) &
\textbf{Przepis na trening} ViT-a bez etykiet (self-supervised).
Mówi: ``podaj sieci dwa widoki tego samego kadru, policz taki-a-taki loss,
zaktualizuj wagi''. LeJEPA \textit{nie jest} siecią neuronową -- to algorytm,
który \textit{używa} ViT-a jako swojego encodera. \\
\textbf{SIGReg} \newline (regularyzator) &
\textbf{Składnik lossu} wewnątrz LeJEPA.
Wymusza, żeby embeddingi miały rozkład $\mathcal{N}(\mathbf{0}, \mathbf{I})$
(izotropowy Gauss -- temu poświęcony jest cały ten dokument). \\
\bottomrule
\end{tabular}
\end{center}

\subsection{Jak to się łączy? Schemat}

Cały system wygląda tak:

\begin{center}
\fbox{\parbox{0.92\textwidth}{
\centering
\textbf{PRETRAINING (uczenie bez etykiet)} \\[0.8em]
\begin{tabular}{ccccc}
Ramka wideo & $\xrightarrow{\text{augmentacje}}$ & Dwa widoki &
$\xrightarrow{\quad\text{ViT}\quad}$ & Embeddingi \\
$\mathbf{x}$ & & $\mathbf{x}', \mathbf{x}''$ & & $\mathbf{z}', \mathbf{z}'' \in \mathbb{R}^{384}$
\end{tabular}
\\[0.8em]
$\downarrow$ \\[0.3em]
\textbf{Loss LeJEPA} = $\lambda \cdot \underbrace{\text{SIGReg}(\mathbf{z})}_{\text{wymusza } \mathcal{N}(\mathbf{0}, \mathbf{I})}
+ (1 - \lambda) \cdot \underbrace{\text{prediction loss}}_{\text{podobne widoki} \to \text{podobne embeddingi}}$
\\[0.5em]
$\downarrow$ \\[0.3em]
Gradient descent aktualizuje wagi $\theta$ ViT-a
}}
\end{center}

\bigskip

\begin{center}
\fbox{\parbox{0.92\textwidth}{
\centering
\textbf{DOWNSTREAM (użycie po treningu)} \\[0.8em]
\begin{tabular}{ccccc}
Nowy obraz & $\xrightarrow{\quad\text{ViT (zamrożony)}\quad}$ & Embedding &
$\xrightarrow{\text{mały klasyfikator}}$ & Wynik \\
$\mathbf{x}$ & & $\mathbf{z} \in \mathbb{R}^{384}$ & & ``faza phaco''
\end{tabular}
\\[0.5em]
Wagi $\theta$ \textbf{nie zmieniają się} -- ViT działa jak zamrożona ``czarna skrzynka''.\\
Uczymy \textit{tylko} mały klasyfikator (linear probe / k-NN).
}}
\end{center}

\subsection{Analogia}

\begin{itemize}[leftmargin=2em]
  \item \textbf{ViT} = \textbf{uczeń} (sieć neuronowa z wagami $\theta$).
  \item \textbf{LeJEPA} = \textbf{program nauczania w szkole}
        -- mówi uczniowi \textit{jak} się uczyć (jakie ćwiczenia robić, jak oceniać postępy).
  \item \textbf{SIGReg} = \textbf{jedno z ćwiczeń} w programie
        -- ``upewnij się, że twoje notatki (embeddingi) są równomiernie rozłożone''.
  \item \textbf{Downstream} = \textbf{praca po szkole}
        -- uczeń (ViT) stosuje swoją wiedzę (zamrożone $\theta^*$) do nowych zadań.
        Program nauczania (LeJEPA) już \textit{nie istnieje} -- został tylko wytrenowany uczeń.
\end{itemize}

\begin{keyinsight}[Zapamiętaj to!]
\textbf{ViT} produkuje embeddingi. \textbf{LeJEPA} mówi ViT-owi \textit{jak się uczyć}.
\textbf{SIGReg} pilnuje, żeby embeddingi miały \textit{właściwy rozkład}.

Po treningu LeJEPA i SIGReg ``znikają'' -- zostaje \textbf{wytrenowany ViT}
z wagami $\theta^*$, który produkuje dobre embeddingi.

Cały ten dokument wyjaśnia \textit{dlaczego} SIGReg wymusza akurat izotropowy Gauss
$\mathcal{N}(\mathbf{0}, \mathbf{I})$ -- i dlaczego to jest matematycznie optymalne.
\end{keyinsight}


\clearpage
%% ============================================================
\section{Notacja: jak czytać $f_\theta: \mathbb{R}^D \to \mathbb{R}^K$?}
\label{sec:notacja}
%% ============================================================

\begin{tcolorbox}[
  colback=lejepaGreen!8,
  colframe=lejepaGreen!80,
  fonttitle=\bfseries,
  title={Wstawka dla początkujących},
  breakable,
]

\subsection*{Rozbiór symboli — element po elemencie}

Zapis $f_\theta: \mathbb{R}^D \to \mathbb{R}^K$ czytamy:

\begin{center}
\large
$\underbrace{f}_{\text{funkcja}}
\underbrace{_\theta}_{\text{param.\ }\theta}
: \;
\underbrace{\mathbb{R}^D}_{\text{wejście}}
\;\to\;
\underbrace{\mathbb{R}^K}_{\text{wyjście}}$
\end{center}

\bigskip

\renewcommand{\arraystretch}{1.6}
\begin{tabular}{p{1.5cm}p{0.68\textwidth}}
\toprule
\textbf{Symbol} & \textbf{Co oznacza?} \\
\midrule
$f$ & \textbf{Funkcja} — maszyna, która coś dostaje i coś wypluwa.
      Tutaj: sieć neuronowa (encoder). \\
$\theta$ & \textbf{Parametry} sieci — wagi i biasy, których sieć uczy się podczas
           treningu. Dolny indeks $\theta$ mówi: ``ta funkcja zależy od parametrów $\theta$''. \\
$\mathbb{R}^D$ & \textbf{Wejście}: wektor $D$ liczb rzeczywistych.
                  $\mathbb{R}$ = liczby rzeczywiste ($-3.14$, $0$, $42.5$, \ldots).
                  Indeks górny $D$ = ile tych liczb.
                  Np.\ obraz $224 \times 224 \times 3$ (RGB) to wektor $D = 150{,}528$ liczb. \\
$\to$ & \textbf{Strzałka}: ``mapuje na'' / ``zamienia na'' / ``produkuje''. \\
$\mathbb{R}^K$ & \textbf{Wyjście}: wektor $K$ liczb — tzw.\ \textbf{embedding}.
                  Np.\ $K=384$ w ViT-Small. Dużo mniejszy niż $D$! \\
\bottomrule
\end{tabular}

\subsection*{Analogia: tłumacz}

Wyobraź sobie \textbf{tłumacza} w biurze ONZ:

\begin{center}\small
\begin{tabular}{rcl}
\textbf{Mowa po polsku} & $\xrightarrow{f_\theta}$ & \textbf{Notatka uniwersalna} \\
(długa, $D$ słów) & & (krótka, $K$ słów) \\[0.3em]
$\mathbb{R}^D$ & $\to$ & $\mathbb{R}^K$ \\
\end{tabular}
\end{center}

\begin{itemize}[leftmargin=2em]
  \item \textbf{Wejście} ($\mathbb{R}^D$): oryginalna mowa — dużo szczegółów, szumu, powtórzeń.
  \item \textbf{Tłumacz} ($f_\theta$): sieć neuronowa — wyciąga \textit{sens}, odrzuca bałagan.
  \item \textbf{Wyjście} ($\mathbb{R}^K$): zwięzła notatka — zawiera tylko to, co ważne.
  \item $\theta$: \textit{doświadczenie} tłumacza — im lepiej wytrenowany, tym lepsza notatka.
\end{itemize}

\subsection*{Konkretny przykład}

\begin{center}\small
\begin{tabular}{rcl}
Ramka wideo & $\xrightarrow{\text{ViT-Small}}$ & Embedding \\
$224{\times}224{\times}3$ pikseli & & $384$ liczby \\
$\mathbf{x} \in \mathbb{R}^{150528}$ & $\to$ & $\mathbf{z} \in \mathbb{R}^{384}$ \\
\end{tabular}
\end{center}

Te $384$ liczby to \textbf{reprezentacja} — kompresja tego, co jest na obrazie
(``tu jest narzędzie chirurgiczne, tu tęczówka, tu soczewka'').
Zamiast $150{,}528$ pikseli, mamy $384$ liczby, które \textit{opisują scenę}.

\end{tcolorbox}

\bigskip

%% ============================================================
\subsection{Co to jest \textit{downstream}?}
\label{sec:downstream}
%% ============================================================

\begin{tcolorbox}[
  colback=lejepaGreen!8,
  colframe=lejepaGreen!80,
  fonttitle=\bfseries,
  title={Wstawka: Downstream task = zadanie dalsze},
  breakable,
]

Trening modeli SSL (self-supervised learning) dzieli się na dwa etapy:

\bigskip

\begin{center}
\renewcommand{\arraystretch}{1.5}
\begin{tabular}{|c|c|}
\hline
\textbf{Etap 1: Pretraining (upstream)} & \textbf{Etap 2: Downstream task} \\
\hline
Trenuj encoder $f_\theta$ &
Zamroź encoder, użyj embeddingów \\
na surowych danych (bez etykiet) &
do \textit{konkretnego} zadania (z etykietami) \\
\hline
``Naucz się języka'' &
``Zdaj egzamin'' \\
\hline
\end{tabular}
\end{center}

\bigskip

\textbf{Analogia}: Pomyśl o wykształceniu:

\begin{enumerate}[leftmargin=2em]
  \item \textbf{Upstream} (pretraining) = \textbf{szkoła ogólna}:
  \begin{itemize}
    \item Uczysz się czytać, pisać, liczyć, myśleć logicznie.
    \item Nie wiesz jeszcze, jaki zawód wybierzesz.
    \item Nie ma ``ocen'' z przyszłego zawodu — uczysz się \textit{ogólnie}.
  \end{itemize}

  \item \textbf{Downstream} = \textbf{praca po szkole}:
  \begin{itemize}
    \item Dostajesz konkretne zadanie: ``rozpoznaj fazę operacji'' albo ``wykryj instrument''.
    \item Używasz tego, czego nauczyłeś się w szkole (embeddingi).
    \item Dodajesz tylko mały ``dodatek'' (linear probe, k-NN), żeby rozwiązać zadanie.
  \end{itemize}
\end{enumerate}

\bigskip

\textbf{Przykłady downstream tasks}:

\begin{center}\footnotesize
\begin{tabular}{lll}
\toprule
\textbf{Zadanie} & \textbf{Wejście} & \textbf{Wyjście} \\
\midrule
Klasyfik.\ fazy & emb.\ ramki & ``phaco''/``IOL'' \\
Wykrycie instr. & emb.\ ramki & ``tak''/``nie'' \\
Segmentacja & emb.\ patchy & mapa pikseli \\
Retrieval & emb.\ zapytania & top-$k$ \\
\bottomrule
\end{tabular}
\end{center}

\bigskip

\begin{keyinsight}[Dlaczego to ważne dla izotropowego Gaussa?]
Podczas pretreningu \textbf{nie wiemy}, jakie zadanie downstream przyjdzie.
Dlatego chcemy, żeby embeddingi były dobre na \textit{każde} możliwe zadanie.

Izotropowy Gauss gwarantuje to matematycznie: żaden kierunek nie jest
uprzywilejowany, więc \textit{dowolna} granica decyzyjna (dowolny klasyfikator)
będzie miała minimalny błąd.
\end{keyinsight}

\end{tcolorbox}

%% ============================================================
\subsection{Co to jest granica decyzyjna?}
\label{sec:granica}
%% ============================================================

\begin{tcolorbox}[
  colback=lejepaGreen!8,
  colframe=lejepaGreen!80,
  fonttitle=\bfseries,
  title={Wstawka: Granica decyzyjna i klasyfikator},
  breakable,
]

\textbf{Klasyfikator} to ten mały ``dodatek'' (linear probe, k-NN),
który nakładamy na zamrożone embeddingi, żeby rozwiązać konkretne zadanie.

\textbf{Granica decyzyjna} to linia (w 2D), płaszczyzna (w 3D) lub
hiperpłaszczyzna (w $K$D), którą klasyfikator rysuje w przestrzeni embeddingów,
żeby \textbf{podzielić} ją na klasy:

\begin{itemize}[leftmargin=2em]
  \item Punkty \textbf{po jednej stronie} granicy $\Rightarrow$ klasa A (np.\ ``incision''),
  \item Punkty \textbf{po drugiej stronie} $\Rightarrow$ klasa B (np.\ ``phaco''),
  \item Nowy, niewidziany punkt: patrzymy \textbf{po której stronie} granicy leży
        $\Rightarrow$ przypisujemy klasę.
\end{itemize}

\begin{figure}[H]
\centering
\includegraphics[width=\textwidth]{figures/decision_boundary.pdf}
\caption{\textbf{Lewo}: Granica decyzyjna = linia dzieląca embeddingi dwóch klas.
Nowy punkt (zielona gwiazdka) wpada po stronie A $\Rightarrow$ klasyfikujemy go jako A.
\textbf{Środek}: Izotropowe embeddingi — granica działa dobrze
w \textit{każdym} kierunku, bo punkty równomiernie wypełniają przestrzeń.
\textbf{Prawo}: Anizotropowe — wzdłuż ``ściśniętej'' osi ($z_2$) granica
prawie nie rozdziela klas, bo punkty leżą na kupce.}
\label{fig:decision}
\end{figure}

\end{tcolorbox}

\bigskip

%% ============================================================
\subsection{Dlaczego izotropia jest kluczowa? Przykład i kontrprzykład}
\label{sec:izotropia_przyklad}
%% ============================================================

\begin{tcolorbox}[
  colback=lejepaBlue!5,
  colframe=lejepaBlue!80,
  fonttitle=\bfseries,
  title={Przykład: ten sam encoder, dwa różne zadania downstream},
  breakable,
]

Wyobraź sobie, że masz wytrenowany encoder (ViT + LeJEPA)
i dostajesz \textbf{dwa różne zadania} do rozwiązania:

\begin{itemize}[leftmargin=2em]
  \item \textbf{Zadanie A}: rozpoznaj fazę operacji
        $\Rightarrow$ granica decyzyjna biegnie \textit{pionowo} (wzdłuż $z_1$),
  \item \textbf{Zadanie B}: wykryj obecność instrumentu
        $\Rightarrow$ granica decyzyjna biegnie \textit{poziomo} (wzdłuż $z_2$).
\end{itemize}

Nie wiesz z góry, które zadanie dostaniesz -- a embeddingi są już zamrożone.

\subsubsection{Przypadek 1: Embeddingi izotropowe ($\boldsymbol{\Sigma} = \mathbf{I}$)}

Punkty rozłożone \textbf{równomiernie} we wszystkich kierunkach:

\begin{center}
\renewcommand{\arraystretch}{1.5}
\begin{tabular}{lcc}
\toprule
 & \textbf{Zad. A (pion.)} & \textbf{Zad. B (poz.)} \\
\midrule
Rozrzut & $\sigma = 1$ & $\sigma = 1$ \\
Separacja klas & wyraźna & wyraźna \\
\textbf{Accuracy} & \textbf{89\%} & \textbf{89\%} \\
\bottomrule
\end{tabular}
\end{center}

Oba zadania działają \textbf{równie dobrze}, bo w obu kierunkach jest tyle samo ``miejsca''.

\subsubsection{Przypadek 2: Embeddingi anizotropowe ($\sigma_1 \gg \sigma_2$)}

Punkty \textbf{rozciągnięte} wzdłuż $z_1$ i \textbf{ściśnięte} wzdłuż $z_2$:

\begin{center}
\renewcommand{\arraystretch}{1.5}
\begin{tabular}{lcc}
\toprule
 & \textbf{Zad. A (pion.)} & \textbf{Zad. B (poz.)} \\
\midrule
Rozrzut & $\sigma_1 = 3.5$ (ogromny!) & $\sigma_2 = 0.12$ (malutki) \\
Separacja klas & rozmyta (dużo szumu) & idealnie ostra \\
\textbf{Accuracy} & \textbf{68\%} (porażka!) & \textbf{99\%} (świetnie!) \\
\bottomrule
\end{tabular}
\end{center}

\textbf{Co się stało?}
\begin{itemize}[leftmargin=2em]
  \item \textbf{Zadanie A} wymagało separacji wzdłuż $z_1$ --
        ale tam jest ogromny rozrzut ($\sigma_1 = 3.5$),
        więc klasy się mieszają i granica decyzyjna popełnia dużo błędów.
  \item \textbf{Zadanie B} wymagało separacji wzdłuż $z_2$ --
        tam rozrzut jest malutki ($\sigma_2 = 0.12$),
        więc nawet mała separacja klas jest wystarczająca.
\end{itemize}

\begin{figure}[H]
\centering
\includegraphics[width=\textwidth]{figures/isotropy_example_vs_counterexample.pdf}
\caption{Porównanie izotropowego (góra) i anizotropowego (dół) rozkładu embeddingów
na dwóch zadaniach downstream.
\textbf{Górny rząd}: izotropowy -- oba zadania na 89\%, bo rozrzut jest jednakowy w obu kierunkach.
\textbf{Dolny rząd}: anizotropowy -- Zadanie A spada do 68\% (klasy się mieszają wzdłuż $z_1$),
Zadanie B skacze do 99\% (klasy ostro rozdzielone wzdłuż $z_2$).}
\label{fig:isotropy_example}
\end{figure}

\subsubsection{Kluczowa obserwacja}

\begin{center}
\renewcommand{\arraystretch}{1.5}
\begin{tabular}{lcc}
\toprule
\textbf{Rozkład embeddingów} & \textbf{Najgorsze zadanie} & \textbf{Najlepsze zadanie} \\
\midrule
Izotropowy & 89\% & 89\% \\
Anizotropowy & 68\% & 99\% \\
\bottomrule
\end{tabular}
\end{center}

Anizotropowy encoder to \textbf{hazard}: może trafić w zadanie, dla którego jest świetny (99\%),
albo w zadanie, dla którego jest fatalny (68\%).
Izotropowy encoder daje \textbf{gwarancję}: niezależnie od zadania, wynik jest stabilnie dobry.

\begin{keyinsight}[Dlaczego izotropia?]
Podczas pretreningu \textbf{nie wiemy}, jakie zadanie downstream przyjdzie.
Nie wiemy, w jakim kierunku będzie biegła granica decyzyjna.

Izotropowy Gauss $\mathcal{N}(\mathbf{0}, \mathbf{I})$ gwarantuje,
że \textbf{żaden kierunek nie jest gorszy od innego} --
rozrzut jest identyczny we \textit{wszystkich} $384$ wymiarach embeddingu.
Dlatego \textit{dowolne} zadanie downstream (dowolna orientacja granicy decyzyjnej)
da co najmniej tak dobry wynik, jak najgorszy przypadek izotropowego rozkładu.

To nie jest kwestia ``więcej linii'' -- to kwestia
\textbf{eliminacji ryzyka}, że trafimy na kierunek, w którym embeddingi są bezużyteczne.
\end{keyinsight}

\end{tcolorbox}


\clearpage
%% ============================================================
\section{Jak działa Vision Transformer (ViT)?}
\label{sec:vit}
%% ============================================================

W LeJEPA encoder $f_\theta$ to \textbf{Vision Transformer} (ViT).
Zanim powiemy \textit{jaki rozkład powinny mieć embeddingi},
musimy zrozumieć \textit{jak ViT je produkuje}.

\subsection{Krok 1: Obraz $\to$ patche (łatki)}

ViT \textbf{nie przetwarza pikseli po kolei} jak CNN.
Zamiast tego dzieli obraz na siatkę kwadratowych kawałków:

\begin{center}
\renewcommand{\arraystretch}{1.5}
\begin{tabular}{rl}
Obraz wejściowy: & $224 \times 224 \times 3$ pikseli (RGB) \\
Rozmiar patcha: & $16 \times 16$ pikseli \\
Liczba patchy: & $\frac{224}{16} \times \frac{224}{16} = 14 \times 14 = \mathbf{196}$ patchy \\
Wymiar jednego patcha: & $16 \times 16 \times 3 = \mathbf{768}$ liczb (spłaszczone piksele) \\
\end{tabular}
\end{center}

Każdy patch to kawałek obrazu (np.\ fragment tęczówki, kawałek narzędzia).

\begin{figure}[H]
\centering
\includegraphics[width=\textwidth]{figures/vit_patches.pdf}
\caption{Od obrazu do tokenów. \textbf{1}: Obraz wejściowy.
\textbf{2}: Podział na $14 \times 14 = 196$ patchy (żółty = przykładowy patch).
\textbf{3}: Każdy patch spłaszczony do wektora $\mathbb{R}^{768}$.
\textbf{4}: Projekcja liniowa do $\mathbb{R}^{384}$ + dodanie tokenu [CLS].}
\label{fig:patches}
\end{figure}

\subsection{Krok 2: Projekcja liniowa (Patch Embedding)}

Wektor 768-wymiarowy to za dużo. ViT kompresuje go \textbf{mnożeniem macierzowym}:

\begin{equation}
\mathbf{e}_i = \mathbf{W}_{\text{patch}} \cdot \mathbf{p}_i + \mathbf{b}_{\text{patch}}
\quad \text{gdzie } \mathbf{W}_{\text{patch}} \in \mathbb{R}^{384 \times 768}
\end{equation}

To zwykłe mnożenie macierzy: $768$ pikseli $\to$ $384$ wymiarów.
Macierz $\mathbf{W}_{\text{patch}}$ jest \textbf{uczona} — sieć sama uczy się,
jakie cechy wyciągać z patcha.

\bigskip

% ============================================================
% ROZBUDOWANY PRZYKŁAD PROJEKCJI LINIOWEJ
% ============================================================

\begin{tcolorbox}[
  colback=lejepaBlue!5,
  colframe=lejepaBlue!80,
  fonttitle=\bfseries,
  title={Przykład krok po kroku: co dokładnie robi mnożenie macierzowe?},
  breakable,
]

Prawdziwe wymiary ($384 \times 768$) są za duże, żeby je zobaczyć.
Użyjemy \textbf{miniaturowego przykładu}: patch $3$ pikseli $\to$ embedding $2$ cech.
Zasada jest \textit{identyczna} — zmienia się tylko rozmiar.

\subsubsection{Krok A: Mamy patch (wektor pikseli)}

Wyobraź sobie maleńki patch $1 \times 1 \times 3$ (jeden piksel RGB):

\begin{equation}
\mathbf{p} = \begin{bmatrix} 0.8 \\ 0.2 \\ 0.1 \end{bmatrix}
\quad \leftarrow \text{3 liczby: } (\underbrace{0.8}_{\text{Red}},\; \underbrace{0.2}_{\text{Green}},\; \underbrace{0.1}_{\text{Blue}})
\end{equation}

To jest nasz ``wektor wejściowy'' $\mathbf{p} \in \mathbb{R}^3$.

\subsubsection{Krok B: Mamy macierz wag (uczoną)}

Sieć ma macierz $\mathbf{W} \in \mathbb{R}^{2 \times 3}$ — dwa wiersze, trzy kolumny:

\begin{equation}
\mathbf{W} = \begin{bmatrix}
w_{11} & w_{12} & w_{13} \\
w_{21} & w_{22} & w_{23}
\end{bmatrix}
= \begin{bmatrix}
0.5 & -0.3 & 0.1 \\
-0.2 & 0.7 & 0.4
\end{bmatrix}
\end{equation}

\textbf{Skąd te liczby?} Na początku treningu — \textbf{losowe}!
Potem gradient descent je zmienia, żeby dawały coraz lepsze cechy.

\subsubsection{Krok C: Mnożenie macierz $\times$ wektor}

\begin{equation}
\mathbf{e} = \mathbf{W} \cdot \mathbf{p} =
\begin{bmatrix}
w_{11} & w_{12} & w_{13} \\
w_{21} & w_{22} & w_{23}
\end{bmatrix}
\begin{bmatrix} p_1 \\ p_2 \\ p_3 \end{bmatrix}
= \begin{bmatrix}
w_{11} p_1 + w_{12} p_2 + w_{13} p_3 \\
w_{21} p_1 + w_{22} p_2 + w_{23} p_3
\end{bmatrix}
\end{equation}

\textbf{Reguła}: każdy element wyniku to \textbf{iloczyn skalarny} wiersza macierzy z wektorem wejściowym.

\medskip

Podstawiamy liczby:

\begin{align}
e_1 &= \underbrace{0.5}_{\mathclap{w_{11}}} \cdot \underbrace{0.8}_{\mathclap{p_1}}
+ \underbrace{(-0.3)}_{\mathclap{w_{12}}} \cdot \underbrace{0.2}_{\mathclap{p_2}}
+ \underbrace{0.1}_{\mathclap{w_{13}}} \cdot \underbrace{0.1}_{\mathclap{p_3}}
\nonumber\\
&= 0.40 + (-0.06) + 0.01 = \mathbf{0.35}
\label{eq:e1_example}
\\[0.8em]
e_2 &= \underbrace{(-0.2)}_{\mathclap{w_{21}}} \cdot \underbrace{0.8}_{\mathclap{p_1}}
+ \underbrace{0.7}_{\mathclap{w_{22}}} \cdot \underbrace{0.2}_{\mathclap{p_2}}
+ \underbrace{0.4}_{\mathclap{w_{23}}} \cdot \underbrace{0.1}_{\mathclap{p_3}}
\nonumber\\
&= (-0.16) + 0.14 + 0.04 = \mathbf{0.02}
\label{eq:e2_example}
\end{align}

\subsubsection{Krok D: Dodajemy bias}

Bias $\mathbf{b} \in \mathbb{R}^2$ to dodatkowy uczony wektor (przesunięcie):

\begin{equation}
\mathbf{e} = \mathbf{W} \cdot \mathbf{p} + \mathbf{b}
= \begin{bmatrix} 0.35 \\ 0.02 \end{bmatrix}
+ \begin{bmatrix} 0.1 \\ -0.05 \end{bmatrix}
= \begin{bmatrix} 0.45 \\ -0.03 \end{bmatrix}
\end{equation}

\textbf{Wynik}: z 3 pikseli RGB dostaliśmy 2-wymiarowy embedding $\mathbf{e} = (0.45,\; {-0.03})^\top$.

\subsubsection{Krok E: Co to znaczy geometrycznie?}

\begin{center}
\renewcommand{\arraystretch}{1.5}
\begin{tabular}{cp{10cm}}
\toprule
\textbf{Element} & \textbf{Interpretacja} \\
\midrule
Wiersz 1 macierzy: $(0.5,\; {-0.3},\; 0.1)$ &
\textbf{Filtr cechy 1}: ``dużo Red, mało Green, trochę Blue''
$\Rightarrow$ reaguje na \textit{ciepłe} kolory (czerwień). \\
Wiersz 2 macierzy: $(-0.2,\; 0.7,\; 0.4)$ &
\textbf{Filtr cechy 2}: ``anty-Red, dużo Green, trochę Blue''
$\Rightarrow$ reaguje na \textit{zimne} kolory (zieleń/cyan). \\
$e_1 = 0.45$ & Nasz patch jest dość ``ciepły'' (dużo Red). \\
$e_2 = -0.03$ & Nasz patch jest prawie neutralny w ``zimnej'' cesze. \\
\bottomrule
\end{tabular}
\end{center}

Każdy wiersz macierzy $\mathbf{W}$ to \textbf{jeden ``detektor cechy''}.
Iloczyn skalarny mierzy \textbf{jak bardzo} patch pasuje do danego detektora.

\subsubsection{Krok F: Skalowanie do prawdziwego ViT-Small}

Teraz podstaw prawdziwe wymiary:

\begin{center}
\renewcommand{\arraystretch}{1.5}
\begin{tabular}{rll}
\toprule
 & \textbf{Nasz przykład} & \textbf{ViT-Small (prawdziwy)} \\
\midrule
Patch (wejście) & $\mathbf{p} \in \mathbb{R}^{3}$ (3 piksele RGB) & $\mathbf{p} \in \mathbb{R}^{768}$ ($16 \times 16 \times 3$ pikseli) \\
Macierz wag & $\mathbf{W} \in \mathbb{R}^{2 \times 3}$ (6 wag) & $\mathbf{W} \in \mathbb{R}^{384 \times 768}$ (294\,912 wag) \\
Bias & $\mathbf{b} \in \mathbb{R}^{2}$ (2 biasy) & $\mathbf{b} \in \mathbb{R}^{384}$ (384 biasy) \\
Embedding (wyjście) & $\mathbf{e} \in \mathbb{R}^{2}$ (2 cechy) & $\mathbf{e} \in \mathbb{R}^{384}$ (384 cechy) \\
\midrule
\textit{Operacja na 1 cechę} & $3$ mnożenia + $3$ dodawania & $768$ mnożeń + $768$ dodawań \\
\textit{Wszystkie cechy} & $2 \times 3 = 6$ operacji & $384 \times 768 = 294\,912$ operacji \\
\textit{Wszystkie patche} & -- & $196 \times 294\,912 \approx \mathbf{57.8\text{M}}$ operacji \\
\bottomrule
\end{tabular}
\end{center}

\subsubsection{Wzorzec ogólny: co robi \textit{każdy} wiersz macierzy?}

Zapiszmy $k$-tą cechę embeddingu ($k = 1, \ldots, 384$):

\begin{equation}
\boxed{
e_k = \sum_{j=1}^{768} w_{kj}\, p_j + b_k
= \underbrace{w_{k1}\, p_1 + w_{k2}\, p_2 + \cdots + w_{k,768}\, p_{768}}_{\text{iloczyn skalarny: wiersz $k$ macierzy $\mathbf{W}$ z wektorem $\mathbf{p}$}} + \;b_k
}
\end{equation}

\textbf{Podsumowanie}: Każda z $384$ cech embeddingu to \textbf{ważona suma wszystkich 768 pikseli patcha}, z wagami wyuczonymi podczas treningu.
Sieć sama decyduje, jakie kombinacje pikseli są informatywne — to jej ``receptory''.

\end{tcolorbox}

\subsection{Krok 3: Embedding pozycyjny}

Po projekcji ViT dodaje \textbf{informację o pozycji} — inaczej nie wiedziałby,
\textit{gdzie} na obrazie jest dany patch (bo Transformer nie ma poczucia kolejności):

\begin{equation}
\mathbf{h}_i^{(0)} = \mathbf{e}_i + \mathbf{pos}_i
\quad \text{gdzie } \mathbf{pos}_i \in \mathbb{R}^{384} \text{ — uczony wektor dla pozycji } i
\end{equation}

Jest $196$ wektorów pozycyjnych (po jednym na każdy patch).

\bigskip

\begin{tcolorbox}[
  colback=lejepaBlue!5,
  colframe=lejepaBlue!80,
  fonttitle=\bfseries,
  title={Pogłębione wyjaśnienie: embedding pozycyjny},
  breakable,
]

\subsubsection{Co znaczy ``uczony wektor''?}

Embedding pozycyjny to \textbf{parametr sieci} -- nie jest wyliczony z żadnej formuły
matematycznej, tylko \textbf{trenowany przez gradient descent} razem z resztą wag.

W kodzie PyTorch to dosłownie jedna linia:
\begin{center}
\texttt{self.pos\_embed = nn.Parameter(torch.randn(1, 196, 384))}
\end{center}

To tworzy tablicę $196$ losowych wektorów po $384$ liczb każdy (łącznie $196 \times 384 = 75\,264$ parametrów).
Są one aktualizowane przez backpropagation tak samo, jak macierze
$\mathbf{W}_Q$, $\mathbf{W}_K$, $\mathbf{W}_V$ i wszystkie inne wagi sieci.

Alternatywą byłyby \textbf{stałe} (fixed) embeddingi pozycyjne -- np.\ sinusoidalne
jak w oryginalnym Transformerze (Vaswani et~al., 2017),
gdzie pozycje koduje się wzorem $\sin(\text{pos}/10000^{2i/d})$.
Ale ViT (Dosovitskiy et~al., 2020) wybrał wariant \textbf{uczony},
bo działa lepiej na obrazach.

\subsubsection{Jak działa mapowanie: patch $i$ $\to$ $\mathbf{pos}_i$?}

To jest \textbf{zwykłe indeksowanie tablicy} -- najprostsza możliwa operacja:

\begin{center}
\renewcommand{\arraystretch}{1.5}
\begin{tabular}{cl}
\toprule
\textbf{Numer patcha} & \textbf{Wektor pozycyjny} \\
\midrule
$1$ (pozycja $(1,1)$ -- lewy górny róg) & $\mathbf{pos}_1 = [0.31,\; 0.28,\; {-0.15},\; \ldots]$ \\
$2$ (pozycja $(1,2)$ -- obok w prawo) & $\mathbf{pos}_2 = [0.33,\; 0.27,\; {-0.14},\; \ldots]$ \\
$15$ (pozycja $(2,1)$ -- rząd niżej) & $\mathbf{pos}_{15} = [0.30,\; 0.31,\; {-0.12},\; \ldots]$ \\
$\vdots$ & $\vdots$ \\
$196$ (pozycja $(14,14)$ -- prawy dolny róg) & $\mathbf{pos}_{196} = [{-0.45},\; {-0.38},\; 0.52,\; \ldots]$ \\
\bottomrule
\end{tabular}
\end{center}

Numeracja patchy jest \textbf{ustalona z góry} -- od lewej do prawej, z góry na dół:
\begin{center}
\begin{tabular}{cccccc}
$1$ & $2$ & $3$ & $4$ & $\cdots$ & $14$ \\
$15$ & $16$ & $17$ & $18$ & $\cdots$ & $28$ \\
$29$ & $30$ & $31$ & $32$ & $\cdots$ & $42$ \\
$\vdots$ & & & & & $\vdots$ \\
$183$ & $184$ & $185$ & $186$ & $\cdots$ & $196$ \\
\end{tabular}
\end{center}

Mapowanie jest \textbf{sztywne} i \textbf{stałe} przez cały trening:
patch w pozycji $(r, c)$ na obrazie \textit{zawsze} dostaje wektor
$\mathbf{pos}_{(r-1) \cdot 14 + c}$.
Zmienia się \textbf{zawartość} wektorów (gradient je modyfikuje),
ale \textbf{przypisanie} numer $\to$ wektor się nigdy nie zmienia.

\subsubsection{Ale skoro na starcie wektory są losowe -- skąd sieć wie, co jest obok czego?}

\textbf{Na starcie -- nie wie!} I to jest OK, bo na początku sieć produkuje śmieciowe embeddingi.
Trening to naprawia krok po kroku:

\textbf{Epoka 0} (przed treningiem):
\begin{center}
$\mathbf{pos}_1 = [0.52,\; {-0.11},\; 0.87,\; \ldots]$ -- losowe \\
$\mathbf{pos}_2 = [{-0.34},\; 0.65,\; 0.02,\; \ldots]$ -- losowe \\
$\mathbf{pos}_{15} = [0.91,\; {-0.73},\; 0.44,\; \ldots]$ -- losowe
\end{center}
Sieć nie ma pojęcia, kto jest obok kogo.
Attention jest równomierny -- każdy patch ``patrzy'' na wszystkich po równo.
Embedding $=$ losowy szum.

\medskip

\textbf{W trakcie treningu}: LeJEPA podaje dwa widoki (augmentacje) tego samego kadru
i mówi: ``embeddingi powinny być podobne''. Żeby to osiągnąć,
sieć \textit{musi} rozumieć \textbf{co gdzie jest} na obrazie -- inaczej
nie potrafi dopasować widoków.

Gradient descent zmienia wektory pozycyjne:
\begin{itemize}[leftmargin=2em]
  \item Patch 1 i Patch 2 (sąsiedzi) \textbf{wielokrotnie} widzą
        podobne rzeczy w danych (sąsiednie kawałki tej samej tęczówki).
        Loss wymusza, żeby attention między nimi działał dobrze.
        Gradient popycha $\mathbf{pos}_1$ i $\mathbf{pos}_2$
        w kierunku \textbf{podobnych} wartości.
  \item Patch 1 i Patch 196 (przeciwne rogi) rzadko widzą podobne rzeczy
        -- gradient je \textbf{odpycha}.
\end{itemize}

\textbf{Po 100 epokach}:
\begin{center}
$\mathbf{pos}_1 = [0.31,\; 0.28,\; {-0.15},\; \ldots]$ \\
$\mathbf{pos}_2 = [0.33,\; 0.27,\; {-0.14},\; \ldots]$
\quad $\leftarrow$ podobny do $\mathbf{pos}_1$ (sąsiad w rzędzie!) \\
$\mathbf{pos}_{15} = [0.30,\; 0.31,\; {-0.12},\; \ldots]$
\quad $\leftarrow$ podobny do $\mathbf{pos}_1$ (sąsiad w kolumnie!) \\
$\mathbf{pos}_{196} = [{-0.45},\; {-0.38},\; 0.52,\; \ldots]$
\quad $\leftarrow$ zupełnie inny (daleki róg)
\end{center}

Sieć \textbf{sama odkryła geometrię} siatki -- nikt jej tego nie powiedział.
To wyłoniło się z danych.

\begin{keyinsight}[Analogia: puzzle]
Embedding pozycyjny to jak \textbf{składanie puzzli}:
\begin{itemize}
  \item \textbf{Epoka 0}: wysypujesz puzzle na stół --
        na odwrocie każdego kawałka jest numer, ale nie wiesz, co pasuje do czego.
  \item \textbf{W trakcie treningu}: zauważasz wzorce --
        kawałek z fragmentem tęczówki pasuje do innego z fragmentem tęczówki
        (attention uczy się, kto jest ważny).
  \item \textbf{Po treningu}: ``ułożyłeś puzzle'' --
        numery na odwrocie nabrały sensu, bo wiesz,
        że kawałek 33 jest zawsze obok kawałka 34.
\end{itemize}
\end{keyinsight}

\subsubsection{Dlaczego dodawanie $\mathbf{e}_i + \mathbf{pos}_i$, a nie konkatenacja?}

Mamy dwie opcje łączenia informacji z patcha ($\mathbf{e}_i$) z pozycją ($\mathbf{pos}_i$):

\begin{center}
\renewcommand{\arraystretch}{1.5}
\begin{tabular}{lcc}
\toprule
\textbf{Metoda} & \textbf{Wymiar wyniku} & \textbf{Koszt obliczeniowy} \\
\midrule
Dodawanie: $\mathbf{h} = \mathbf{e} + \mathbf{pos}$ & $384$ (bez zmiany) & zerowy \\
Konkatenacja: $\mathbf{h} = [\mathbf{e}\;;\;\mathbf{pos}]$ & $768$ (podwojenie!) & $2\times$ w każdej warstwie \\
\bottomrule
\end{tabular}
\end{center}

Dodawanie jest ``za darmo'' i w praktyce daje tak samo dobre wyniki
(sprawdzone empirycznie w artykule o ViT).
Sieć sama uczy się ``rozdzielać'' wymiary -- część wymiarów
w $\mathbf{h}_i$ będzie zdominowana przez $\mathbf{e}_i$ (treść wizualna),
część przez $\mathbf{pos}_i$ (pozycja), a część będzie mieszanką.

\subsubsection{Jak gradient przepływa przez dodawanie?}

Zanim zastosujemy to do ViT-a, przypomnijmy \textbf{podstawową regułę pochodnych}:

\medskip

\textbf{Reguła: pochodna sumy.}
Jeśli $f(x) = x + c$, gdzie $c$ nie zależy od $x$, to:
\begin{equation}
\frac{\partial f}{\partial x} = \frac{\partial (x + c)}{\partial x}
= \underbrace{\frac{\partial x}{\partial x}}_{= 1}
+ \underbrace{\frac{\partial c}{\partial x}}_{= 0}
= 1
\end{equation}

\textbf{Dlaczego?}
\begin{itemize}[leftmargin=2em]
  \item Pochodna $x$ po $x$ $= 1$, bo zmiana $x$ o $\Delta$ zmienia wynik dokładnie o $\Delta$.
  \item Pochodna $c$ po $x$ $= 0$, bo $c$ jest niezależne od $x$ -- zmiana $x$ nie rusza $c$.
\end{itemize}

\textbf{Przykład liczbowy.}
Niech $f(x) = x + 7$. Wtedy:
\begin{center}
\renewcommand{\arraystretch}{1.4}
\begin{tabular}{ccc}
$x$ & $f(x) = x + 7$ & Zmiana $f$ gdy $x$ rośnie o $1$ \\
\midrule
$3$ & $10$ & -- \\
$4$ & $11$ & $+1$ \\
$5$ & $12$ & $+1$ \\
$6$ & $13$ & $+1$
\end{tabular}
\end{center}
Każdy wzrost $x$ o $1$ daje wzrost $f$ o dokładnie $1$. Stąd $\frac{\partial f}{\partial x} = 1$.
Stała $7$ nic nie zmienia -- przesuwa krzywą w górę, ale nie zmienia nachylenia.

\medskip

\textbf{Zastosowanie do ViT-a.}
Sieć liczy loss na podstawie $\mathbf{h}_i$, a $\mathbf{h}_i = \mathbf{e}_i + \mathbf{pos}_i$.
Dla dowolnego wymiaru $k$ (od $1$ do $384$):
\begin{equation}
h_i[k] = \underbrace{e_i[k]}_{\text{nasza ``$x$''}} + \underbrace{\text{pos}_i[k]}_{\text{nasze ``$c$'' (inna zmienna)}}
\end{equation}

Stosujemy regułę pochodnej sumy:
\begin{equation}
\frac{\partial h_i[k]}{\partial e_i[k]}
= \underbrace{\frac{\partial\, e_i[k]}{\partial\, e_i[k]}}_{=1}
+ \underbrace{\frac{\partial\, \text{pos}_i[k]}{\partial\, e_i[k]}}_{=0,\text{ bo } \text{pos}_i \text{ nie zależy od } e_i}
= 1
\end{equation}

Analogicznie:
\begin{equation}
\frac{\partial h_i[k]}{\partial\, \text{pos}_i[k]}
= \underbrace{\frac{\partial\, e_i[k]}{\partial\, \text{pos}_i[k]}}_{=0}
+ \underbrace{\frac{\partial\, \text{pos}_i[k]}{\partial\, \text{pos}_i[k]}}_{=1}
= 1
\end{equation}

Ponieważ to zachodzi dla \textit{każdego} wymiaru $k$, zapisujemy wektorowo:
\begin{equation}
\frac{\partial \mathbf{h}_i}{\partial \mathbf{e}_i} = \mathbf{1}, \qquad
\frac{\partial \mathbf{h}_i}{\partial \mathbf{pos}_i} = \mathbf{1}
\end{equation}

Więc przez chain rule \textbf{oba dostają ten sam gradient}:
\begin{equation}
\frac{\partial \mathcal{L}}{\partial \mathbf{e}_i}
= \frac{\partial \mathcal{L}}{\partial \mathbf{h}_i} \cdot 1
= \frac{\partial \mathcal{L}}{\partial \mathbf{h}_i}, \qquad
\frac{\partial \mathcal{L}}{\partial \mathbf{pos}_i}
= \frac{\partial \mathcal{L}}{\partial \mathbf{h}_i} \cdot 1
= \frac{\partial \mathcal{L}}{\partial \mathbf{h}_i}
\end{equation}

Ale potem gradient \textbf{rozchodzi się} w dwie różne gałęzie:

\begin{center}
\begin{tabular}{cp{10.5cm}}
\toprule
\textbf{Gałąź} & \textbf{Co się dzieje z gradientem?} \\
\midrule
$\mathbf{pos}_i$ &
To jest \textbf{parametr sieci} (liść w grafie obliczeń).
Gradient \textbf{bezpośrednio} go aktualizuje:
$\mathbf{pos}_i \leftarrow \mathbf{pos}_i - \eta \cdot \frac{\partial \mathcal{L}}{\partial \mathbf{pos}_i}$ \\
\midrule
$\mathbf{e}_i$ &
To \textbf{nie jest} parametr -- to wynik obliczenia
$\mathbf{e}_i = \mathbf{W}_{\text{patch}} \cdot \mathbf{p}_i + \mathbf{b}_{\text{patch}}$.
Gradient płynie \textbf{dalej wstecz}:
$\frac{\partial \mathcal{L}}{\partial \mathbf{W}_{\text{patch}}}
= \frac{\partial \mathcal{L}}{\partial \mathbf{e}_i} \cdot \mathbf{p}_i^\top$
$\Rightarrow$ aktualizuje $\mathbf{W}_{\text{patch}}$ i $\mathbf{b}_{\text{patch}}$. \\
\bottomrule
\end{tabular}
\end{center}

Schematycznie -- gradient rozchodzi się w dwie gałęzie.

\textbf{Uwaga:} $\mathcal{L}$ to \textbf{pełna funkcja straty LeJEPA} (szczegółowo opisana
w Sekcji~\ref{sec:fulloss}, rów.~\ref{eq:lejepa}):
\[
\mathcal{L} = \lambda \cdot \text{SIGReg}(\mathbf{z}) + (1-\lambda) \cdot \|\text{centroid} - \mathbf{z}_v\|^2
\]
Pierwszy człon wymusza rozkład $\mathcal{N}(\mathbf{0}, \mathbf{I})$,
drugi -- żeby widoki tego samego obrazu miały podobne embeddingi.

\begin{center}
$\mathcal{L}$ (loss) \\[0.3em]
$\downarrow$ \small{gradient $\frac{\partial \mathcal{L}}{\partial \mathbf{h}_i}$} \\[0.3em]
$\mathbf{h}_i = \mathbf{e}_i + \mathbf{pos}_i$ \\[0.5em]
\begin{tabular}{c@{\hspace{2cm}}c}
$\swarrow$ \textbf{Gałąź lewa} & $\searrow$ \textbf{Gałąź prawa} \\[0.5em]
$\mathbf{e}_i$ & $\mathbf{pos}_i$ \\
\small{(to nie jest parametr!} & \small{(to JEST parametr)} \\
\small{gradient płynie dalej $\downarrow$)} & \small{$\Downarrow$ aktualizuj bezpośrednio:} \\[0.3em]
$\downarrow$ & $\mathbf{pos}_i \leftarrow \mathbf{pos}_i - \eta \cdot \frac{\partial \mathcal{L}}{\partial \mathbf{h}_i}$ \\[0.8em]
$\mathbf{W}_{\text{patch}} \cdot \mathbf{p}_i + \mathbf{b}_{\text{patch}}$ & \\[0.3em]
$\swarrow$ \hspace{1cm} $\searrow$ & \\[0.3em]
$\mathbf{W}_{\text{patch}} \leftarrow$ \small{aktual.}
\hspace{0.5cm} $\mathbf{b}_{\text{patch}} \leftarrow$ \small{aktual.} & \\
\end{tabular}
\end{center}

\begin{keyinsight}[Dlaczego uczą się różnych rzeczy, skoro gradient jest ten sam?]
Bo mają \textbf{różne źródła}:
\begin{itemize}
  \item $\mathbf{e}_i$ zależy od \textbf{pikseli patcha} $\mathbf{p}_i$
        -- więc $\mathbf{W}_{\text{patch}}$ uczy się wyciągać \textit{cechy wizualne}
        (krawędzie, kolory, tekstury).
  \item $\mathbf{pos}_i$ zależy \textbf{tylko od numeru pozycji} -- nie widzi pikseli,
        więc może kodować wyłącznie \textit{``gdzie na obrazie''}.
\end{itemize}
Ten sam gradient, ale dwa \textbf{różne} cele uczenia -- bo informacja
o treści (piksele) i o pozycji (numer) są \textit{rozdzielone architekturalnie}.
\end{keyinsight}

\end{tcolorbox}

\begin{figure}[H]
\centering
\includegraphics[width=\textwidth]{figures/vit_positions.pdf}
\caption{\textbf{Lewo}: Wizualizacja jednego wymiaru embeddingu pozycyjnego na siatce $14 \times 14$.
Bliskie pozycje mają podobne wartości — sieć ``nauczyła się'' geometrii obrazu.
\textbf{Prawo}: Podobieństwo cosinusowe między wybranymi pozycjami —
bliskie pozycje (zielony) vs.\ dalekie (czerwony).}
\label{fig:positions}
\end{figure}

\subsection{Krok 4: Token [CLS]}

Przed wejściem do Transformera dodajemy \textbf{specjalny token} [CLS]
(class token) — dodatkowy uczony wektor $\mathbf{h}_{\text{CLS}}^{(0)} \in \mathbb{R}^{384}$.

Teraz mamy $196 + 1 = \mathbf{197}$ tokenów, każdy o wymiarze $384$.

\begin{tcolorbox}[colback=blue!3, colframe=blue!60!black, breakable,
  title=\textbf{Co to jest token [CLS] i po co go dodajemy?}]

\textbf{Problem}: mamy 196 patchy, każdy z embeddingiem w $\mathbb{R}^{384}$.
Potrzebujemy \textbf{jednego} wektora, który opisuje \textbf{cały obraz}.
Jak go uzyskać?

\textbf{Naiwne podejście}: uśrednić wszystkie 196 patchy: $\bar{\mathbf{h}} = \frac{1}{196}\sum_{i=1}^{196} \mathbf{h}_i$.
Ale to traci informację -- patch z narzędziem dostaje taką samą wagę jak patch z tłem.

\textbf{Rozwiązanie}: dodajemy \textbf{sztuczny token nr 0} -- [CLS] (od ``classification''):

\begin{center}
\renewcommand{\arraystretch}{1.3}
\begin{tabular}{cl}
\toprule
\textbf{Token} & \textbf{Czym jest} \\
\midrule
$\mathbf{h}_1, \ldots, \mathbf{h}_{196}$ & Embeddingi 196 patchy (fragmenty obrazu) \\
$\mathbf{h}_{\text{CLS}}$ & \textbf{Sztuczny token} -- nie odpowiada żadnemu patchowi! \\
\bottomrule
\end{tabular}
\end{center}

\textbf{Kluczowe właściwości [CLS]:}

\begin{enumerate}[leftmargin=2em]
\item \textbf{Nie ma „swoich'' pikseli} -- nie pochodzi z żadnego fragmentu obrazu.
      Na starcie to po prostu \textbf{uczony wektor} (384 losowych liczb, które się zmieniają podczas treningu).

\item \textbf{Uczestniczy w attention} jak każdy inny token --
      ma swoje $\mathbf{Q}_{\text{CLS}}, \mathbf{K}_{\text{CLS}}, \mathbf{V}_{\text{CLS}}$.

\item \textbf{[CLS] pyta, patche odpowiadają}:
      w mechanizmie attention [CLS] liczy $\mathbf{Q}_{\text{CLS}} \cdot \mathbf{K}_j$
      dla każdego patcha $j$. Patche z ważną informacją (narzędzie, tęczówka)
      dostaną \textbf{duże wagi} $\alpha_{\text{CLS},j}$, a tło -- małe.

\item \textbf{Po 12 warstwach} [CLS] ``zebrał'' informację ze wszystkich patchy,
      ale \textbf{ważoną} przez attention -- ważniejsze patche miały większy wpływ.
\end{enumerate}

\textbf{Analogia}: [CLS] to \textbf{reporter} na konferencji prasowej.
196 patchy to mówcy. Reporter (CLS) słucha wszystkich, ale notuje głównie
to, co mówią najważniejsi (narzędzie, tęczówka). Na końcu reporter pisze
\textbf{jedno podsumowanie} (embedding $\mathbf{z}$) -- to jest wynik ViT.

\medskip

\textbf{Ale po co [CLS] w ogóle jest? Co wnosi?}

[CLS] \textbf{nic nie dodaje} do obrazu -- to działa \textbf{odwrotnie}!
To patche modyfikują [CLS], nie [CLS] modyfikuje patche.

\textbf{Problem}: ViT na wyjściu daje \textbf{197 wektorów}
(196 patchy + [CLS]). Ale do klasyfikacji czy SIGReg potrzebujemy
\textbf{jednego} wektora opisującego cały obraz.
Jak skompresować 197 wektorów do 1?

\begin{center}
\renewcommand{\arraystretch}{1.4}
\begin{tabular}{lcp{6cm}}
\toprule
\textbf{Metoda} & \textbf{Wynik} & \textbf{Problem} \\
\midrule
Uśrednij 196 patchy & $\bar{\mathbf{h}} = \frac{1}{196}\sum_i \mathbf{h}_i$
  & Każdy patch ma wagę $0.5\%$ -- tło = narzędzie. Tracimy informację o ważności. \\
Weź losowy patch & $\mathbf{h}_{42}$
  & Opisuje tylko $\frac{1}{196}$ obrazu. Reszta jest ignorowana. \\
\textbf{[CLS] + attention} & $\mathbf{h}_{\text{CLS}}^{(12)}$
  & \textbf{Brak!} Sieć sama uczy się, ile uwagi dać każdemu patchowi. \\
\bottomrule
\end{tabular}
\end{center}

[CLS] to \textbf{``pusty kubek''}, który przez 12 warstw attention
\textbf{napełnia się} najważniejszą informacją z patchy:

\begin{center}
\begin{tabular}{lcl}
Warstwa 0: & $\mathbf{h}_{\text{CLS}}^{(0)}$ & = losowy wektor (pusty kubek) \\
Warstwa 1: & $\mathbf{h}_{\text{CLS}}^{(1)}$ & = kubek + trochę info z patchy \\
Warstwa 6: & $\mathbf{h}_{\text{CLS}}^{(6)}$ & = kubek + dużo info (krawędzie, kolory) \\
Warstwa 12: & $\mathbf{h}_{\text{CLS}}^{(12)}$ & = \textbf{pełne podsumowanie obrazu} $= \mathbf{z}$
\end{tabular}
\end{center}

\textbf{Kluczowy punkt}: [CLS] \textbf{nie zmienia} patchy -- to patche zmieniają [CLS].
Każda warstwa attention pozwala [CLS] ``wchłonąć'' więcej informacji
z ważnych regionów obrazu.

\medskip

\textbf{Co konkretnie zawiera $\mathbf{z} \in \mathbb{R}^{384}$?}

To \textbf{384 liczb} opisujących \textbf{semantykę} całego obrazu.
Ale te liczby \textbf{nie mają ludzkich nazw} -- nie ma
``cecha nr 47 = kolor tęczówki'' ani ``cecha nr 200 = typ narzędzia''.
To \textbf{abstrakcyjne cechy}, które sieć sama wymyśliła podczas treningu.

Czego \textbf{nie wiemy}: co oznacza każda z 384 liczb z osobna.

Czego \textbf{wiemy}: dwa obrazy o \textbf{podobnej semantyce}
(np.\ ta sama faza operacji) mają \textbf{bliskie} wektory $\mathbf{z}$:
\[
\text{podobne obrazy} \;\Rightarrow\; \|\mathbf{z}_A - \mathbf{z}_B\| \text{ jest \textbf{mały}}
\]
\[
\text{różne obrazy} \;\Rightarrow\; \|\mathbf{z}_A - \mathbf{z}_B\| \text{ jest \textbf{duży}}
\]

To dlatego potem działa \textbf{linear probe} -- wystarczy prosta linia
(granica decyzyjna) w $\mathbb{R}^{384}$, żeby oddzielić klasy,
bo \textbf{sieć już ułożyła} podobne obrazy blisko siebie.

\medskip

\textbf{Przykład} (uproszczony do 3 wymiarów zamiast 384):
\begin{center}
\renewcommand{\arraystretch}{1.3}
\begin{tabular}{lc}
\toprule
\textbf{Obraz} & $\mathbf{z}$ (embedding) \\
\midrule
Incision, klatka 1 & $(0.8, \; -0.3, \; 1.2)$ \\
Incision, klatka 2 & $(0.7, \; -0.2, \; 1.1)$ \quad $\leftarrow$ bliski! \\
Phaco, klatka 1 & $(-1.5, \; 0.9, \; -0.4)$ \quad $\leftarrow$ daleki! \\
\bottomrule
\end{tabular}
\end{center}

Dwa kadry incision mają prawie identyczne $\mathbf{z}$,
a phaco jest daleko -- mimo że \textbf{piksele} mogą wyglądać zupełnie inaczej
(inny kąt kamery, inne oświetlenie). To jest siła embeddingów:
\textbf{kodują znaczenie, nie wygląd}.

\medskip

\textbf{Czy [CLS] ma własne macierze $\mathbf{W}_Q, \mathbf{W}_K, \mathbf{W}_V$?}

\textbf{Nie!} [CLS] używa \textbf{tych samych} macierzy co wszystkie 196 patchy.
To jest cała idea self-attention -- jedna wspólna trójka macierzy
$(\mathbf{W}_Q, \mathbf{W}_K, \mathbf{W}_V)$ obsługuje \textbf{wszystkich} 197 tokenów:

\[
\mathbf{Q}_{42} = \mathbf{W}_Q \cdot \mathbf{h}_{42}, \qquad
\mathbf{Q}_{\text{CLS}} = \mathbf{W}_Q \cdot \mathbf{h}_{\text{CLS}}
\qquad \longleftarrow \text{ta sama } \mathbf{W}_Q\text{!}
\]

Różnica nie jest w macierzach, tylko we \textbf{wejściu}:
$\mathbf{h}_{42}$ zawiera cechy patcha nr 42 (piksele),
a $\mathbf{h}_{\text{CLS}}$ zawiera dotychczas zebraną informację o obrazie.
Po mnożeniu przez tę samą $\mathbf{W}_Q$ dostają \textbf{różne} pytania Q,
bo wejścia są różne.

\end{tcolorbox}

\subsection{Krok 5: Bloki Transformera (encoder)}

Serce ViT to $L = 12$ identycznych bloków (w ViT-Small).
Każdy blok składa się z dwóch części:

\subsubsection{(a) Self-Attention: ``kto na kogo patrzy''}

Każdy token tworzy trzy wektory przez mnożenie macierzowe:

\begin{equation}
\mathbf{Q}_i = \mathbf{W}_Q \mathbf{h}_i, \quad
\mathbf{K}_i = \mathbf{W}_K \mathbf{h}_i, \quad
\mathbf{V}_i = \mathbf{W}_V \mathbf{h}_i
\end{equation}

\begin{center}
\renewcommand{\arraystretch}{1.5}
\begin{tabular}{cl}
\toprule
\textbf{Wektor} & \textbf{Rola} \\
\midrule
$\mathbf{Q}_i$ (Query) & ``Czego szukam?'' — pytanie tokenu $i$ \\
$\mathbf{K}_j$ (Key) & ``Co mam do zaoferowania?'' — opis tokenu $j$ \\
$\mathbf{V}_j$ (Value) & ``Jaka jest moja treść?'' — informacja tokenu $j$ \\
\bottomrule
\end{tabular}
\end{center}

% ============ SZCZEGÓŁOWE WYJAŚNIENIE MACIERZY W_Q, W_K, W_V ============
\begin{tcolorbox}[colback=blue!3, colframe=blue!60!black, breakable,
  title=\textbf{Co to są macierze $\mathbf{W}_Q$, $\mathbf{W}_K$, $\mathbf{W}_V$?}]

\textbf{Jeden token} $\mathbf{h}_i \in \mathbb{R}^{384}$ zawiera \textit{wszystko} o patchu $i$:
kolor, teksturę, kształt, pozycję\ldots To jest 384-wymiarowy opis ``wszystkiego naraz''.

Problem: żeby mechanizm attention mógł działać, potrzebujemy \textbf{trzech różnych perspektyw}
na ten sam token -- pytanie, odpowiedź, treść. Gdybyśmy używali $\mathbf{h}_i$ bezpośrednio
do wszystkich trzech ról, nie moglibyśmy oddzielić ``czego szukam'' od ``co oferuję''.

\textbf{Rozwiązanie}: mnożymy $\mathbf{h}_i$ przez trzy \textit{różne} macierze,
które ``wyciągają'' różne aspekty informacji:

\vspace{6pt}

\textbf{Wymiary macierzy} (ViT-Small z 6 głowicami, $d = 384/6 = 64$ na głowicę):

\begin{center}
\renewcommand{\arraystretch}{1.4}
\begin{tabular}{lccl}
\toprule
\textbf{Macierz} & \textbf{Wymiar} & \textbf{Operacja} & \textbf{Efekt} \\
\midrule
$\mathbf{W}_Q$ & $64 \times 384$ & $\mathbf{W}_Q \cdot \mathbf{h}_i = \mathbf{Q}_i$ & $\mathbb{R}^{384} \to \mathbb{R}^{64}$ \\
$\mathbf{W}_K$ & $64 \times 384$ & $\mathbf{W}_K \cdot \mathbf{h}_i = \mathbf{K}_i$ & $\mathbb{R}^{384} \to \mathbb{R}^{64}$ \\
$\mathbf{W}_V$ & $64 \times 384$ & $\mathbf{W}_V \cdot \mathbf{h}_i = \mathbf{V}_i$ & $\mathbb{R}^{384} \to \mathbb{R}^{64}$ \\
\bottomrule
\end{tabular}
\end{center}

Każda macierz \textbf{kompresuje} 384 wymiarów do 64 wymiarów,
ale każda robi to \textbf{inaczej} -- wyciągając inne cechy!

\end{tcolorbox}

% ============ DLACZEGO W_Q "PYTA", A W_K "ODPOWIADA"? ============
\begin{tcolorbox}[breakable, colback=red!3, colframe=red!60!black,
  title=\textbf{Ale dlaczego $\mathbf{W}_Q$ ``pyta'', a $\mathbf{W}_K$ ``odpowiada''?}]

To kluczowe pytanie! Odpowiedź jest zaskakująca:

\textbf{Nic w samych macierzach nie sprawia, że jedna jest ``pytająca'' a druga ``odpowiadająca''.}
Na początku treningu wszystkie trzy ($\mathbf{W}_Q, \mathbf{W}_K, \mathbf{W}_V$)
to \textbf{losowe macierze} -- identyczne co do natury.

To, co nadaje im rolę, to ich \textbf{pozycja w obliczeniach} (architektura):

\begin{center}
\begin{tabular}{l}
$\underbrace{\mathbf{Q}_i \cdot \mathbf{K}_j}_{\text{iloczyn skalarny}}
\;\to\; \alpha_{ij} \;\text{(waga attention: \textbf{KTO} jest ważny)}$ \\[10pt]
$\underbrace{\sum_j \alpha_{ij} \cdot \mathbf{V}_j}_{\text{ważona suma}}
\;\to\; \text{wynik (\textbf{TREŚĆ} do przekazania dalej)}$
\end{tabular}
\end{center}

gdzie:
\begin{itemize}[leftmargin=2em]
\item $i$ = token, który \textbf{pyta} (``ja'') -- np.\ patch nr 42 (fragment narzędzia),
\item $j$ = token, który jest \textbf{odpytywany} (``inni'') -- przebiega po \textit{wszystkich}
      197 tokenach ($j = 1, 2, \ldots, 197$),
\item $\alpha_{ij}$ = ile uwagi token $i$ poświęca tokenowi $j$
      (po softmax: $\sum_j \alpha_{ij} = 1$),
\item $\mathbf{V}_j$ = treść, którą token $j$ ``przekazuje'', gdy zostanie wybrany.
\end{itemize}
Innymi słowy: dla \textbf{każdego} tokenu $i$, obliczamy jego podobieństwo
do \textbf{każdego} tokenu $j$ (włącznie z samym sobą, $j = i$).

\textbf{Dlaczego to wystarczy?} Bo gradient ``widzi'' różne ścieżki:

\begin{itemize}[leftmargin=2em]
\item \textbf{$\mathbf{W}_Q$ i $\mathbf{W}_K$} uczą się ``współpracować'' --
      bo ich wyniki ($\mathbf{Q}$ i $\mathbf{K}$) są mnożone skalarnie ($\mathbf{Q} \cdot \mathbf{K}$).
      Gradient wymusza: duży iloczyn skalarny dla \textit{ważnych} par tokenów,
      mały dla \textit{nieważnych}.

\item \textbf{$\mathbf{W}_V$} uczy się czegoś \textbf{zupełnie innego} --
      bo jej wynik nie wpływa na to, \textit{kto} jest ważny,
      tylko \textit{jaką informację} przekazać, gdy token zostanie ``wybrany''.
\end{itemize}

\medskip

\textbf{Analogia}: Trzy osoby na budowie -- na początku identyczne.
Ale jeden dostał \textbf{łopatę} (= pozycja Q·K w równaniu),
drugi \textbf{wiertło} (= pozycja K·Q),
trzeci \textbf{pędzel} (= pozycja V w sumie ważonej).
\textbf{Narzędzie} (pozycja w architekturze) determinuje ich specjalizację!

\medskip

\textbf{Dowód}: gdybyśmy zamienili $\mathbf{W}_Q$ z $\mathbf{W}_K$
(ale nie zmienili architektury), sieć po treningu dałaby
\textbf{identyczne wyniki} -- bo $\mathbf{Q}_i \cdot \mathbf{K}_j = \mathbf{K}_j \cdot \mathbf{Q}_i$
(iloczyn skalarny jest przemienny).
Natomiast zamiana $\mathbf{W}_Q$ z $\mathbf{W}_V$ \textbf{zepsułaby} sieć,
bo $\mathbf{V}$ jest używane \textit{inaczej} w obliczeniach.

\end{tcolorbox}

% ============ MULTI-HEAD: DLACZEGO 6 GŁOWIC ============
\begin{tcolorbox}[breakable, colback=green!3, colframe=green!60!black,
  title=\textbf{Dlaczego 6 głowic (Multi-Head Attention)?}]

Jedna głowica (jedna trójka $\mathbf{W}_Q, \mathbf{W}_K, \mathbf{W}_V$) liczy Q, K i V
-- ale ponieważ ma \textbf{jeden} zestaw macierzy, potrafi wykrywać
\textbf{tylko jeden typ relacji} między tokenami.
Np.\ jeśli ta głowica nauczyła się, że $\mathbf{Q} \cdot \mathbf{K}$ daje duży wynik
dla patchy o podobnym \textit{kolorze}, to nie potrafi \textbf{jednocześnie}
szukać patchy \textit{bliskich przestrzennie} -- bo ma tylko jedną parę
$(\mathbf{W}_Q, \mathbf{W}_K)$.

\textbf{Rozwiązanie: 6 niezależnych głowic}, każda z własnym zestawem macierzy:

\begin{center}
\renewcommand{\arraystretch}{1.3}
\begin{tabular}{clc}
\toprule
\textbf{Głowica} & \textbf{Czego się uczy szukać (przykład)} & \textbf{Macierze} \\
\midrule
1 & ``Kto ma podobną \textit{teksturę}?'' & $\mathbf{W}_Q^{(1)}, \mathbf{W}_K^{(1)}, \mathbf{W}_V^{(1)}$ \\
2 & ``Kto jest \textit{blisko} przestrzennie?'' & $\mathbf{W}_Q^{(2)}, \mathbf{W}_K^{(2)}, \mathbf{W}_V^{(2)}$ \\
3 & ``Kto ma ten sam \textit{kolor}?'' & $\mathbf{W}_Q^{(3)}, \mathbf{W}_K^{(3)}, \mathbf{W}_V^{(3)}$ \\
4 & ``Gdzie jest \textit{krawędź} narzędzia?'' & $\mathbf{W}_Q^{(4)}, \mathbf{W}_K^{(4)}, \mathbf{W}_V^{(4)}$ \\
5 & ``Kto wygląda jak \textit{tęczówka}?'' & $\mathbf{W}_Q^{(5)}, \mathbf{W}_K^{(5)}, \mathbf{W}_V^{(5)}$ \\
6 & ``Kto jest w \textit{tle} (nieistotny)?'' & $\mathbf{W}_Q^{(6)}, \mathbf{W}_K^{(6)}, \mathbf{W}_V^{(6)}$ \\
\bottomrule
\end{tabular}
\end{center}

Wyniki 6 głowic są \textbf{sklejane} (konkatenacja) i przepuszczane przez jedną macierz wyjściową:
\begin{equation}
\text{MultiHead}(\mathbf{h}) = \mathbf{W}_O \cdot
\underbrace{\text{Concat}\!\left(
\text{head}_1, \;
\text{head}_2, \;
\ldots, \;
\text{head}_6
\right)}_{6 \times 64 = 384}
\end{equation}
gdzie $\mathbf{W}_O \in \mathbb{R}^{384 \times 384}$ -- jeszcze jedna uczona macierz.

\medskip

\textbf{Dlaczego akurat 6?} Bo $384 / 6 = 64$, a $64$ wymiary na głowicę to sprawdzona wartość:

\begin{center}
\renewcommand{\arraystretch}{1.3}
\begin{tabular}{lccc}
\toprule
\textbf{Wariant ViT} & \textbf{$d_{\text{model}}$} & \textbf{Głowice} & \textbf{$d_{\text{head}}$} \\
\midrule
ViT-Tiny & 192 & 3 & $192/3 = 64$ \\
\textbf{ViT-Small} & \textbf{384} & \textbf{6} & $\mathbf{384/6 = 64}$ \\
ViT-Base & 768 & 12 & $768/12 = 64$ \\
ViT-Large & 1024 & 16 & $1024/16 = 64$ \\
\bottomrule
\end{tabular}
\end{center}

Zawsze $\mathbf{d_{\text{head}} = 64}$ -- to \textbf{stała}!
Większe modele mają więcej głowic (więcej ``pytań'' równolegle),
a nie większe głowice.

\medskip

\textbf{Ale dlaczego 6 głowic nie uczą się tego samego?}

Nic nie \textit{gwarantuje}, że się nie nauczą -- ale w praktyce uczą się
\textbf{różnych} relacji z dwóch powodów:

\begin{enumerate}[leftmargin=2em]
\item \textbf{Losowa inicjalizacja}: każda głowica startuje z \textit{innymi} losowymi macierzami.
      Gradient descent to optymalizacja \textbf{lokalna} -- różne punkty startowe
      prowadzą do różnych rozwiązań (lokalnych minimów).

\item \textbf{Gradient ``nagradza'' za nowe informacje}:
      jeśli głowica 1 już nauczyła się wykrywać \textit{kolor},
      to loss jest częściowo zaspokojony w tym kierunku.
      Gradient dla pozostałych głowic popycha je w stronę
      \textbf{innych} wzorców (tekstura, pozycja\ldots),
      bo \textit{tam} jest jeszcze pole do poprawy lossu.
\end{enumerate}

\textbf{Analogia}: zespół 6 pracowników z losowo przydzielonymi zadaniami.
Jeśli pracownik 1 już robi zadanie A dobrze, szef (gradient) nie nagradza
pracownika 2 za \textit{powtórzenie} A -- nagroda idzie do tego, kto weźmie się za B.

\medskip

\textbf{Uwaga}: w praktyce głowice \textbf{nie zawsze} uczą się idealnie
różnych rzeczy -- część jest redundantna.
Badania (Michel et al., 2019) pokazują, że w wytrenowanym Transformerze
można \textbf{usunąć} nawet 40\% głowic bez dużej utraty jakości (\textit{head pruning}).
To potwierdza, że pewna nadmiarowość jest normalna.

\end{tcolorbox}

% ============ MINI-PRZYKŁAD ============
\begin{tcolorbox}[breakable, colback=yellow!5, colframe=orange!70!black,
  title=\textbf{Mini-przykład: $\mathbf{h}_i \in \mathbb{R}^{4} \to$ Q, K, V $\in \mathbb{R}^{2}$}]

Niech token $\mathbf{h}_i$ ma 4 cechy: $\mathbf{h}_i = \begin{bmatrix} 0.5 \\ 1.0 \\ -0.3 \\ 0.8 \end{bmatrix}$

Trzy macierze (uczone podczas treningu, tu uproszczone):

\[
\mathbf{W}_Q = \begin{bmatrix} 1 & 0 & 0 & 0 \\ 0 & 1 & 0 & 0 \end{bmatrix}, \quad
\mathbf{W}_K = \begin{bmatrix} 0 & 0 & 1 & 0 \\ 0 & 0 & 0 & 1 \end{bmatrix}, \quad
\mathbf{W}_V = \begin{bmatrix} 0.5 & 0.5 & 0 & 0 \\ 0 & 0 & 0.5 & 0.5 \end{bmatrix}
\]

\textbf{Obliczamy:}
\begin{align}
\mathbf{Q}_i &= \mathbf{W}_Q \cdot \mathbf{h}_i
= \begin{bmatrix} 1 & 0 & 0 & 0 \\ 0 & 1 & 0 & 0 \end{bmatrix}
\begin{bmatrix} 0.5 \\ 1.0 \\ -0.3 \\ 0.8 \end{bmatrix}
= \begin{bmatrix} 0.5 \\ 1.0 \end{bmatrix}
\quad \text{\small(wyciąga cechy 1--2)} \nonumber\\[6pt]
\mathbf{K}_i &= \mathbf{W}_K \cdot \mathbf{h}_i
= \begin{bmatrix} 0 & 0 & 1 & 0 \\ 0 & 0 & 0 & 1 \end{bmatrix}
\begin{bmatrix} 0.5 \\ 1.0 \\ -0.3 \\ 0.8 \end{bmatrix}
= \begin{bmatrix} -0.3 \\ 0.8 \end{bmatrix}
\quad \text{\small(wyciąga cechy 3--4)} \nonumber\\[6pt]
\mathbf{V}_i &= \mathbf{W}_V \cdot \mathbf{h}_i
= \begin{bmatrix} 0.5 & 0.5 & 0 & 0 \\ 0 & 0 & 0.5 & 0.5 \end{bmatrix}
\begin{bmatrix} 0.5 \\ 1.0 \\ -0.3 \\ 0.8 \end{bmatrix}
= \begin{bmatrix} 0.75 \\ 0.25 \end{bmatrix}
\quad \text{\small(uśrednia pary cech)} \nonumber
\end{align}

\textbf{Obserwacja}: ten sam token $\mathbf{h}_i$ dał \textbf{trzy różne} wektory!
\begin{itemize}[leftmargin=2em]
\item $\mathbf{Q}_i = (0.5, \; 1.0)$ -- ``to czego szukam'' (np. cechy krawędziowe),
\item $\mathbf{K}_i = (-0.3, \; 0.8)$ -- ``to co oferuję'' (np. cechy kolorowe),
\item $\mathbf{V}_i = (0.75, \; 0.25)$ -- ``moja faktyczna treść'' (uśrednione cechy).
\end{itemize}

\end{tcolorbox}

% ============ ANALOGIA ============
\begin{tcolorbox}[breakable, colback=green!3, colframe=green!60!black,
  title=\textbf{Analogia: trzy różne okulary}]

Wyobraź sobie, że $\mathbf{h}_i$ to \textbf{pełny opis} osoby (wzrost, waga, kolor oczu,
wykształcenie, dochód, hobby\ldots).

\begin{itemize}[leftmargin=2em]
\item $\mathbf{W}_Q$ = ``okulary pytające'' -- pokazują cechy, których ta osoba \textit{szuka}
      u innych (np. wykształcenie, hobby).
\item $\mathbf{W}_K$ = ``okulary reklamowe'' -- pokazują cechy, które ta osoba \textit{oferuje}
      (np. zawód, umiejętności).
\item $\mathbf{W}_V$ = ``okulary treściowe'' -- pokazują \textit{konkretną informację},
      którą ta osoba przekaże, gdy ktoś ją ``wybierze''.
\end{itemize}

\textbf{Kluczowy punkt}: wszystkie trzy macierze $\mathbf{W}_Q, \mathbf{W}_K, \mathbf{W}_V$
są \textbf{parametrami uczonymi} -- na początku treningu zawierają losowe wartości.
Sieć sama uczy się, jakie ``okulary'' są najlepsze!

\end{tcolorbox}

% ============ PEŁNA PROCEDURA Q*K ============
\begin{tcolorbox}[breakable, colback=blue!3, colframe=blue!60!black,
  title=\textbf{Pełna procedura: od $\mathbf{h}$ do wag attention (mini-przykład z 3 tokenami)}]

Mamy 3 tokeny z $d=2$ (wymiar wektora Q i K \textbf{wewnątrz jednej głowicy},
w prawdziwym ViT-Small $d = 384/6 = 64$; tu używamy $d=2$, żeby można było liczyć ręcznie).
Po projekcji przez $\mathbf{W}_Q$ i $\mathbf{W}_K$:

\begin{center}
\renewcommand{\arraystretch}{1.3}
\begin{tabular}{lcc}
\toprule
\textbf{Token} & $\mathbf{Q}_i$ & $\mathbf{K}_i$ \\
\midrule
$i=1$ (narzędzie) & $(1.0, \; 0.2)$ & $(0.1, \; 0.9)$ \\
$i=2$ (tęczówka) & $(0.3, \; 0.8)$ & $(0.7, \; 0.5)$ \\
$i=3$ (tło) & $(0.1, \; 0.1)$ & $(0.2, \; 0.3)$ \\
\bottomrule
\end{tabular}
\end{center}

\textbf{Krok 1}: Macierz podobieństw $S_{ij} = \mathbf{Q}_i \cdot \mathbf{K}_j$:

\[
S = \begin{bmatrix}
Q_1 \cdot K_1 & Q_1 \cdot K_2 & Q_1 \cdot K_3 \\
Q_2 \cdot K_1 & Q_2 \cdot K_2 & Q_2 \cdot K_3 \\
Q_3 \cdot K_1 & Q_3 \cdot K_2 & Q_3 \cdot K_3
\end{bmatrix}
= \begin{bmatrix}
0.28 & 0.80 & 0.26 \\
0.75 & 0.61 & 0.30 \\
0.10 & 0.12 & 0.05
\end{bmatrix}
\]

Np. $S_{12} = Q_1 \cdot K_2 = 1.0 \times 0.7 + 0.2 \times 0.5 = 0.80$ -- \textbf{wysoka wartość!}
Token 1 (narzędzie) ``szuka'' czegoś, co token 2 (tęczówka) ``oferuje''.

\textbf{Krok 2}: Dzielimy przez $\sqrt{d} = \sqrt{2} \approx 1.41$:
\[
\tilde{S} = S / \sqrt{2} \approx \begin{bmatrix} 0.20 & 0.57 & 0.18 \\ 0.53 & 0.43 & 0.21 \\ 0.07 & 0.08 & 0.04 \end{bmatrix}
\]

\textbf{Krok 3}: Softmax (wiersz po wierszu).

\textbf{Wzór na softmax} -- zamienia dowolne liczby rzeczywiste na prawdopodobieństwa (nieujemne, sumujące się do 1):

\begin{equation}
\text{softmax}(x_1, x_2, \ldots, x_n)_k = \frac{e^{x_k}}{\sum_{m=1}^{n} e^{x_m}}
\end{equation}

\textbf{Co robi?} Bierze wektor dowolnych wartości i zamienia go na rozkład prawdopodobieństwa:
\begin{itemize}[leftmargin=2em]
\item duże $x_k$ $\to$ duże $e^{x_k}$ $\to$ duży udział (blisko 1),
\item małe $x_k$ $\to$ małe $e^{x_k}$ $\to$ mały udział (blisko 0),
\item zawsze: $\sum_k \text{softmax}(x)_k = 1$ (suma wag = 100\%).
\end{itemize}

\textbf{Przykład na wierszu 1} macierzy $\tilde{S}$: \; $(\tilde{s}_{11},\; \tilde{s}_{12},\; \tilde{s}_{13}) = (0.20,\; 0.57,\; 0.18)$

\[
e^{0.20} = 1.22, \quad e^{0.57} = 1.77, \quad e^{0.18} = 1.20
\quad \Rightarrow \quad \text{suma} = 4.19
\]
\[
\alpha_{11} = \frac{1.22}{4.19} = 0.29, \quad
\alpha_{12} = \frac{1.77}{4.19} = \mathbf{0.42}, \quad
\alpha_{13} = \frac{1.20}{4.19} = 0.29
\]

Sprawdzenie: $0.29 + 0.42 + 0.29 = 1.00$ \; \checkmark

Stosujemy softmax do każdego wiersza:
\[
\alpha = \text{softmax}(\tilde{S}) \approx \begin{bmatrix}
0.29 & \mathbf{0.42} & 0.29 \\
\mathbf{0.39} & 0.35 & 0.26 \\
0.34 & 0.34 & 0.33
\end{bmatrix}
\]

\textbf{Interpretacja}: Token 1 zwraca $42\%$ uwagi na token 2, bo
$\mathbf{W}_Q$ wyciągnęła z narzędzia pytanie ``gdzie tęczówka?'',
a $\mathbf{W}_K$ wyciągnęła z tęczówki odpowiedź ``tutaj jestem!''.

Token 3 (tło) nie szuka niczego konkretnego $\rightarrow$ uwaga prawie równomierna (33\%).
\end{tcolorbox}

% ============ WIZUALIZACJA: CO W ROBI ============
\begin{tcolorbox}[breakable, colback=gray!5, colframe=gray!60!black,
  title=\textbf{Wizualizacja: jak $\mathbf{W}$ transformuje przestrzeń}]

Mnożenie wektora przez macierz $\mathbf{W}$ to \textbf{transformacja liniowa} --
obrót, rozciągnięcie i/lub kompresja przestrzeni:

\begin{center}
\begin{tabular}{ccc}
$\mathbf{h}_i \in \mathbb{R}^{384}$ & $\xrightarrow{\quad \mathbf{W}_Q \quad}$ & $\mathbf{Q}_i \in \mathbb{R}^{64}$ \\[4pt]
\small{384 cech (``wszystko'')} & \small{projekcja} & \small{64 cechy (``pytanie'')} \\[10pt]
\multicolumn{3}{c}{\textit{Co robi ta projekcja geometrycznie?}} \\[6pt]
\multicolumn{3}{l}{\textbf{1.} Wybiera 64 ``najważniejszych'' kierunków z 384-wymiarowej przestrzeni} \\
\multicolumn{3}{l}{\textbf{2.} Obraca je tak, żeby iloczyn skalarny Q $\cdot$ K mierzył ``to co trzeba''} \\
\multicolumn{3}{l}{\textbf{3.} Różne macierze $\mathbf{W}_Q$ vs $\mathbf{W}_K$ = różne ``co trzeba''}
\end{tabular}
\end{center}

\textbf{Ile parametrów?} Każda macierz $\mathbf{W}$ to $64 \times 384 = 24\,576$ wag.
Dla 3 macierzy: $3 \times 24\,576 = 73\,728$.
Ale mamy 6 głowic, każda z własnymi macierzami: $6 \times 73\,728 = 442\,368$ parametrów
-- sam attention to $\sim 0.44$M parametrów na blok!

\end{tcolorbox}

Teraz budujemy wzór na wagę uwagi (attention) krok po kroku.

\medskip

\textbf{Krok 1: Ile token $i$ ``interesuje się'' tokenem $j$?}

Liczymy \textbf{iloczyn skalarny} Query tokenu $i$ z Key tokenu $j$:
\begin{equation}
s_{ij} = \mathbf{Q}_i \cdot \mathbf{K}_j = \sum_{m=1}^{d} Q_i[m] \cdot K_j[m]
\end{equation}

Iloczyn skalarny mierzy \textbf{podobieństwo kierunków}:
\begin{itemize}[leftmargin=2em]
  \item $s_{ij}$ duże i dodatnie $\Rightarrow$ pytanie $i$ pasuje do opisu $j$ (token $j$ jest ``interesujący''),
  \item $s_{ij} \approx 0$ $\Rightarrow$ brak związku,
  \item $s_{ij}$ ujemne $\Rightarrow$ ``odpychają się'' (token $j$ jest nieistotny dla $i$).
\end{itemize}

\textbf{Krok 2: Normalizacja przez $\sqrt{d}$.}

$d$ to \textbf{wymiar wektorów Q/K w jednej głowie attention}.
ViT-Small ma $6$ głów, a pełny wymiar to $384$, więc każda głowa operuje na:
\begin{equation}
d = \frac{384}{6} = 64, \qquad \sqrt{d} = \sqrt{64} = 8
\end{equation}

\textbf{Po co dzielenie?} Iloczyn skalarny to suma $d$ składników.
Jeśli składowe $Q$ i $K$ mają wariancję $\approx 1$,
to wariancja sumy rośnie proporcjonalnie do $d$:
\begin{equation}
\text{Var}(s_{ij}) = \text{Var}\!\left(\sum_{m=1}^{d} Q_i[m] \cdot K_j[m]\right) \approx d
\end{equation}

Więc $s_{ij}$ ma odchylenie standardowe $\approx \sqrt{d}$.
Dzielenie przez $\sqrt{d}$ sprowadza skalę z powrotem do $\approx 1$:
\begin{equation}
\tilde{s}_{ij} = \frac{s_{ij}}{\sqrt{d}} = \frac{\mathbf{Q}_i \cdot \mathbf{K}_j}{\sqrt{d}}
\quad \Rightarrow \quad \text{Var}(\tilde{s}_{ij}) \approx 1
\end{equation}

Bez tego dzielenia: $s_{ij}$ miałoby wartości rzędu $\pm\sqrt{64} = \pm 8$,
a $e^8 \approx 2981$ vs $e^{-8} \approx 0.0003$
-- softmax dałby prawie \textit{one-hot} (jeden token dostaje $\approx 100\%$ uwagi).
Po dzieleniu: $\tilde{s}_{ij}$ ma wartości rzędu $\pm 1$, a $e^1 \approx 2.7$ vs $e^{-1} \approx 0.37$
-- softmax daje ``miękkie'' wagi i sieć może patrzeć na \textit{wielu} tokenów.

\textbf{Krok 3: Zamiana na prawdopodobieństwa (softmax).}

Mamy wyniki $\tilde{s}_{i1}, \tilde{s}_{i2}, \ldots, \tilde{s}_{i,197}$ -- jak bardzo
token $i$ interesuje się każdym z $197$ tokenów.
Chcemy zamienić je na \textbf{wagi sumujące się do 1} (żeby wynik był ważoną średnią).

Funkcja \textbf{softmax} robi dokładnie to:
\begin{equation}
\text{softmax}(x_j) = \frac{e^{x_j}}{\sum_{k} e^{x_k}}
\end{equation}

Dlaczego $e^x$, a nie np.\ samo $x$?
\begin{itemize}[leftmargin=2em]
  \item $e^x > 0$ zawsze -- wagi są nieujemne (prawdopodobieństwa),
  \item $e^x$ rośnie \textit{wykładniczo} -- duże $\tilde{s}_{ij}$ dominują
        (sieć ``skupia uwagę'' na najważniejszych tokenach),
  \item dzielenie przez sumę gwarantuje $\sum_j \alpha_{ij} = 1$.
\end{itemize}

\textbf{Krok 4: Składamy wszystko razem.}

Podstawiamy $x_j = \tilde{s}_{ij} = \mathbf{Q}_i \cdot \mathbf{K}_j / \sqrt{d}$ do softmax:
\begin{equation}
\boxed{
\alpha_{ij} = \frac{e^{\,\mathbf{Q}_i \cdot \mathbf{K}_j \,/\, \sqrt{d}}}
{\sum_{k=1}^{197} e^{\,\mathbf{Q}_i \cdot \mathbf{K}_k \,/\, \sqrt{d}}}
}
\end{equation}

\textbf{Przykład liczbowy} (uproszczony, 3 tokeny, $d=2$):
\begin{align}
\mathbf{Q}_1 &= (1,\; 0), \quad
\mathbf{K}_1 = (1,\; 0), \quad
\mathbf{K}_2 = (0,\; 1), \quad
\mathbf{K}_3 = (-1,\; 0) \nonumber\\[0.3em]
s_{11} &= 1 \cdot 1 + 0 \cdot 0 = 1, \quad
s_{12} = 1 \cdot 0 + 0 \cdot 1 = 0, \quad
s_{13} = 1 \cdot ({-1}) + 0 \cdot 0 = {-1} \nonumber\\[0.3em]
\tilde{s}_{11} &= 1/\sqrt{2} \approx 0.71, \quad
\tilde{s}_{12} = 0/\sqrt{2} = 0, \quad
\tilde{s}_{13} = {-1}/\sqrt{2} \approx {-0.71} \nonumber\\[0.3em]
e^{0.71} &\approx 2.03, \quad e^{0} = 1, \quad e^{-0.71} \approx 0.49
\quad \Rightarrow \quad \text{suma} = 3.52 \nonumber\\[0.3em]
\alpha_{11} &= 2.03 / 3.52 \approx \mathbf{0.58}, \quad
\alpha_{12} = 1 / 3.52 \approx \mathbf{0.28}, \quad
\alpha_{13} = 0.49 / 3.52 \approx \mathbf{0.14}
\end{align}

Token 1 patrzy głównie na siebie (58\%), trochę na token 2 (28\%), prawie ignoruje token 3 (14\%)
-- bo Query 1 jest najbardziej ``zgodne'' z Key 1.

\medskip

\textbf{Krok 5: Ważona suma wartości.}

Nowa wartość tokenu $i$:
\begin{equation}
\mathbf{h}_i' = \sum_{j=1}^{197} \alpha_{ij}\,\mathbf{V}_j
= \alpha_{i1}\,\mathbf{V}_1 + \alpha_{i2}\,\mathbf{V}_2 + \cdots + \alpha_{i,197}\,\mathbf{V}_{197}
\end{equation}

Token $i$ \textbf{zbiera informację} od wszystkich tokenów,
ale \textit{więcej} od tych, na które ``patrzy'' (wysokie $\alpha_{ij}$).

\begin{keyinsight}[Intuicja: self-attention to ``rozmowa'']
Każdy token ``rozmawia'' ze wszystkimi innymi:
\begin{itemize}
  \item Token patcha z tęczówką pyta (Q): ``Kto jeszcze jest tęczówką?''
  \item Inne patche tęczówki odpowiadają (K): ``Ja!'' $\Rightarrow$ wysoki $\alpha_{ij}$
  \item Token zbiera ich treść (V) i aktualizuje swoją reprezentację
  \item \textbf{[CLS]} pyta WSZYSTKICH i zbiera globalne podsumowanie
\end{itemize}
\end{keyinsight}

\subsubsection{(b) MLP: przetwarzanie lokalne}

Po attention każdy token przechodzi przez sieć MLP \textbf{niezależnie od innych tokenów}.
O ile Self-Attention odpowiada na pytanie ``\textit{kto na kogo patrzy}'',
MLP odpowiada na pytanie ``\textit{co to wszystko razem znaczy}''.

\subsubsection{Wzór MLP -- krok po kroku}

Pełny wzór MLP w jednej linii:
\begin{equation}
\boxed{
\text{MLP}(\mathbf{h}) = \mathbf{W}_2\,\text{GELU}(\mathbf{W}_1 \mathbf{h} + \mathbf{b}_1) + \mathbf{b}_2
}
\label{eq:mlp}
\end{equation}
gdzie $\mathbf{W}_1 \in \mathbb{R}^{1536 \times 384}$,
$\mathbf{W}_2 \in \mathbb{R}^{384 \times 1536}$,
$\mathbf{b}_1 \in \mathbb{R}^{1536}$,
$\mathbf{b}_2 \in \mathbb{R}^{384}$.

Rozpiszmy to na trzy oddzielne kroki:

\textbf{Krok 1 -- Rozszerzenie liniowe} ($384 \to 1536$, czyli $4\times$ więcej wymiarów):
\begin{equation}
\mathbf{z} = \mathbf{W}_1 \mathbf{h} + \mathbf{b}_1 \in \mathbb{R}^{1536}
\end{equation}

Każdy element $z_j$ to \textbf{ważona suma} wszystkich $384$ wymiarów wejścia:
\[
z_j = \sum_{i=1}^{384} W_{1,ji} \cdot h_i + b_{1,j} \qquad \text{dla } j = 1, \ldots, 1536
\]
Jeden wiersz macierzy $\mathbf{W}_1$ definiuje jeden ``kandydat na cechę'' --
inną liniową kombinację wejściowych wymiarów.
Mamy $1536$ takich kandydatów, czyli sieć tworzy $1536$ różnych ``hipotez''
o tym, co wejściowy token oznacza.

\textbf{Krok 2 -- GELU (nieliniowość)} ($1536 \to 1536$, ten sam wymiar):
\begin{equation}
\tilde{\mathbf{z}} = \text{GELU}(\mathbf{z}) \in \mathbb{R}^{1536}
\end{equation}
GELU działa \textbf{element po elemencie}: $\tilde{z}_j = \text{GELU}(z_j)$ dla każdego $j$.
\begin{itemize}[leftmargin=2em]
\item Jeśli $z_j > 0$: cecha ``przepuszczona'' prawie bez zmian ($\tilde{z}_j \approx z_j$).
\item Jeśli $z_j \ll 0$: cecha ``wyciszona'' ($\tilde{z}_j \approx 0$).
\item To jest \textbf{selekcja cech}: z~$1536$ kandydatów GELU wybiera te, które są aktywne
      (wartość dodatnia), a resztę wycisza. Typowo około 40--60\% neuronów jest wyciszonych.
\end{itemize}

\textbf{Dlaczego GELU jest konieczne?} Bez nieliniowości MLP byłby złożeniem dwóch transformacji liniowych:
\[
\mathbf{W}_2(\mathbf{W}_1 \mathbf{h} + \mathbf{b}_1) + \mathbf{b}_2
= \underbrace{(\mathbf{W}_2 \mathbf{W}_1)}_{\mathbf{W}'} \mathbf{h}
+ \underbrace{(\mathbf{W}_2 \mathbf{b}_1 + \mathbf{b}_2)}_{\mathbf{b}'}
\]
Złożenie dwóch macierzy to wciąż jedna macierz -- cała ukryta warstwa byłaby zbędna.
GELU ``łamie'' tę liniowość i umożliwia tworzenie \textbf{nieliniowych kombinacji cech}.

\textbf{Krok 3 -- Kompresja liniowa} ($1536 \to 384$, powrót do oryginalnego wymiaru):
\begin{equation}
\mathbf{o} = \mathbf{W}_2 \tilde{\mathbf{z}} + \mathbf{b}_2 \in \mathbb{R}^{384}
\end{equation}

Każdy element wyjścia $o_i$ to ważona suma \textbf{przefiltrowanych} cech:
\[
o_i = \sum_{j=1}^{1536} W_{2,ij} \cdot \tilde{z}_j + b_{2,i} \qquad \text{dla } i = 1, \ldots, 384
\]
Ponieważ GELU wyzerowała część $\tilde{z}_j$, ta suma efektywnie bierze pod uwagę
tylko \textbf{aktywne} cechy -- różny podzbiór dla różnych tokenów.

\begin{center}
\includegraphics[width=0.98\textwidth]{figures/mlp_architecture.pdf}
\captionof{figure}{Schemat MLP: rozszerzenie $4\times$ daje ``przestrzeń roboczą'' na cechy pośrednie,
GELU filtruje je, a kompresja $4\times$ pakuje wynik z powrotem do $384$ wymiarów.}
\label{fig:mlp_arch}
\end{center}

\subsubsection{Przykład liczbowy MLP}

Dla czytelności użyjemy $d = 4$ (zamiast $384$) i warstwy ukrytej $d_h = 8$ (zamiast $1536$).

Niech wejście po Self-Attention: $\mathbf{h} = (1.2, \; -0.8, \; 0.5, \; -1.5)$.

\textbf{Krok 1}: Mnożymy przez macierz $\mathbf{W}_1 \in \mathbb{R}^{8 \times 4}$ i dodajemy bias $\mathbf{b}_1$:
\[
\mathbf{z} = \mathbf{W}_1 \mathbf{h} + \mathbf{b}_1
\]
Przykładowe obliczenie dla $z_1$ (pierwszy wiersz $\mathbf{W}_1 = [0.6, \; -0.3, \; 0.8, \; 0.1]$):
\begin{align}
z_1 &= 0.6 \cdot 1.2 + (-0.3) \cdot (-0.8) + 0.8 \cdot 0.5 + 0.1 \cdot (-1.5) + 0.1 \nonumber \\
    &= 0.72 + 0.24 + 0.40 - 0.15 + 0.1 = 1.31 \nonumber
\end{align}
Analogicznie dla pozostałych 7 elementów:
\[
\mathbf{z} = (1.31, \; -0.37, \; 0.06, \; 0.33, \; 1.18, \; 0.84, \; -0.22, \; 1.44)
\]

\textbf{Krok 2}: Stosujemy GELU element po elemencie:
\begin{center}
\renewcommand{\arraystretch}{1.4}
\begin{tabular}{c|r|r|l}
\toprule
$j$ & $z_j$ & $\tilde{z}_j = \text{GELU}(z_j)$ & \textbf{Efekt} \\
\midrule
1 & $+1.31$ & $+1.19$ & przepuszczony \\
2 & $-0.37$ & $-0.13$ & \textcolor{red}{stłumiony} \\
3 & $+0.06$ & $+0.03$ & na granicy \\
4 & $+0.33$ & $+0.22$ & przepuszczony \\
5 & $+1.18$ & $+1.06$ & przepuszczony \\
6 & $+0.84$ & $+0.69$ & przepuszczony \\
7 & $-0.22$ & $-0.09$ & \textcolor{red}{stłumiony} \\
8 & $+1.44$ & $+1.34$ & przepuszczony \\
\bottomrule
\end{tabular}
\end{center}

GELU wyciszyła elementy $z_2$ i $z_7$ (ujemne wartości $\to$ prawie zero).
Pozostałe $6$ z $8$ cech zostało ``przepuszczonych'' -- to jest \textbf{selekcja cech}.

\textbf{Krok 3}: Kompresja $\mathbf{W}_2 \tilde{\mathbf{z}} + \mathbf{b}_2$ z powrotem do $\mathbb{R}^4$:
\[
\mathbf{o} = \mathbf{W}_2 \tilde{\mathbf{z}} + \mathbf{b}_2 = (0.83, \; 0.34, \; 0.89, \; 0.62)
\]

\textbf{Wynik MLP}: $(1.2, \; -0.8, \; 0.5, \; -1.5) \xrightarrow{\text{MLP}} (0.83, \; 0.34, \; 0.89, \; 0.62)$

Każda wartość wyjściowa to nieliniowa kombinacja cech wejściowych,
gdzie GELU zdecydowała, \textit{które} cechy pośrednie brały udział w obliczeniu.

\begin{center}
\includegraphics[width=0.98\textwidth]{figures/mlp_step_by_step.pdf}
\captionof{figure}{Wizualizacja MLP krok po kroku.
Od lewej: wejście $\mathbf{h}$ ($d=4$), po rozszerzeniu $\mathbf{W}_1 \mathbf{h} + \mathbf{b}_1$ ($d=8$),
po GELU (wartości ujemne wyciszone -- selekcja cech),
wyjście $\mathbf{W}_2 \tilde{\mathbf{z}} + \mathbf{b}_2$ ($d=4$, kompresja).
Kolor niebieski = wartość dodatnia, czerwony = ujemna.}
\label{fig:mlp_steps}
\end{center}

\subsubsection{Co GELU robi z neuronami ukrytej warstwy?}

\begin{center}
\includegraphics[width=0.98\textwidth]{figures/mlp_gelu_effect.pdf}
\captionof{figure}{Lewo: transformacja GELU -- każdy neuron ukrytej warstwy (punkt)
jest przepuszczany (prawa strona) lub wyciszany (lewa strona, region czerwony).
Prawo: rozkład wartości $1000$ neuronów przed i po GELU -- GELU ``ściąga''
ujemne wartości do zera, tworząc efektywną selekcję cech.}
\label{fig:mlp_gelu}
\end{center}

\begin{tcolorbox}[
  colback=lejepaBlue!5,
  colframe=lejepaBlue!80,
  fonttitle=\bfseries,
  title={Dlaczego takie wymiary? Rozszerzenie $4\times$ i kompresja},
  breakable,
]

\textbf{Skąd $1536 = 4 \times 384$?}

MLP działa jak \textbf{odwrócony bottleneck} (inverted bottleneck):

\begin{center}
\begin{tabular}{ccccc}
$\mathbf{h} \in \mathbb{R}^{384}$
& $\xrightarrow{\;\mathbf{W}_1\;}$
& $\mathbb{R}^{1536}$
& $\xrightarrow{\;\mathbf{W}_2\;}$
& $\mathbf{h}_{\text{out}} \in \mathbb{R}^{384}$ \\
\small{wejście} & \small{rozszerzenie $4\times$} & \small{ukryta warstwa} & \small{kompresja $4\times$} & \small{wyjście}
\end{tabular}
\end{center}

\textbf{Dlaczego nie zostać przy $384$?}
Attention zbiera informację \textit{między} tokenami (``kto jest obok kogo'').
Ale potem każdy token musi tę informację \textbf{przetworzyć lokalnie}
-- np.\ ``widzę krawędź narzędzia + czerwony kolor + bliskość tęczówki $\Rightarrow$
to jest końcówka phaco''.

Takie \textbf{nieliniowe kombinowanie cech} wymaga dużo parametrów.
Rozszerzenie do $1536$ wymiarów daje sieci ``przestrzeń roboczą'',
w której może tworzyć skomplikowane cechy pośrednie.

\end{tcolorbox}

% ============ GELU: SZCZEGÓŁOWE WYJAŚNIENIE ============
\begin{tcolorbox}[breakable, colback=blue!3, colframe=blue!60!black,
  title=\textbf{GELU: czym jest i jak działa?}]

Funkcja GELU (Gaussian Error Linear Unit) to \textbf{gładka wersja ReLU}:

\begin{equation}
\text{GELU}(x) = x \cdot \Phi(x) \qquad \text{gdzie } \Phi(x) = \frac{1}{2}\left[1 + \text{erf}\!\left(\frac{x}{\sqrt{2}}\right)\right]
\end{equation}

$\Phi(x)$ to \textbf{dystrybuanta rozkładu normalnego} -- prawdopodobieństwo,
że zmienna losowa z $\mathcal{N}(0,1)$ jest $\leq x$.

\textbf{Intuicja}: GELU mnoży wartość $x$ przez ``prawdopodobieństwo, że $x$ jest ważne'':
\begin{itemize}[leftmargin=2em]
\item Jeśli $x \gg 0$: \; $\Phi(x) \approx 1$ \; $\Rightarrow$ \; $\text{GELU}(x) \approx x$ \quad (przepuść bez zmian)
\item Jeśli $x \ll 0$: \; $\Phi(x) \approx 0$ \; $\Rightarrow$ \; $\text{GELU}(x) \approx 0$ \quad (wycisz)
\item Jeśli $x \approx 0$: \; $\Phi(x) \approx 0.5$ \; $\Rightarrow$ \; $\text{GELU}(x) \approx 0.5x$ \quad (\textbf{łagodne tłumienie})
\end{itemize}

\textbf{Porównanie wartości:}
\begin{center}
\renewcommand{\arraystretch}{1.3}
\begin{tabular}{r|ccc}
$x$ & $\text{ReLU}(x)$ & $\Phi(x)$ & $\text{GELU}(x) = x\Phi(x)$ \\
\hline
$-3$ & $0$ & $0.001$ & $-0.004$ \\
$-1$ & $0$ & $0.159$ & $-0.159$ \\
$0$ & $0$ & $0.500$ & $0$ \\
$+1$ & $1$ & $0.841$ & $0.841$ \\
$+3$ & $3$ & $0.999$ & $2.996$ \\
\end{tabular}
\end{center}

\textbf{Dlaczego GELU a nie ReLU?}
\begin{itemize}[leftmargin=2em]
\item \textbf{ReLU}: ostro zeruje wszystko $<0$ (pochodna = 0 dla $x<0$ $\to$ ``martwe neurony'')
\item \textbf{GELU}: łagodnie tłumi -- wartości bliskie zeru są \textit{częściowo} przepuszczane,
      co daje lepszy przepływ gradientu.
\end{itemize}

\begin{center}
\includegraphics[width=0.95\textwidth]{figures/gelu_activation.pdf}
\captionof{figure}{Lewo: porównanie GELU z ReLU -- GELU łagodnie tłumi wartości ujemne zamiast
ostro je zerować. Prawo: efekt GELU na rozkład wartości -- wartości ujemne są ``spychane''
w kierunku zera, ale nie wycinane.}
\label{fig:gelu}
\end{center}

\end{tcolorbox}

\begin{tcolorbox}[
  colback=lejepaBlue!5,
  colframe=lejepaBlue!80,
  fonttitle=\bfseries,
  title={Dlaczego akurat rozszerzenie $4\times$?},
  breakable,
]

\textbf{Dlaczego akurat $4\times$?}
To wartość ustalona empirycznie przez Vaswani et~al.\ (2017).
Testowano $1\times$, $2\times$, $4\times$, $8\times$:
\begin{itemize}[leftmargin=2em]
  \item $1\times$ lub $2\times$: za mało pojemności -- sieć nie potrafi tworzyć złożonych cech,
  \item $4\times$: dobry kompromis wydajność/koszt,
  \item $8\times$: marginalna poprawa, ale $2\times$ więcej parametrów i obliczeń.
\end{itemize}
Wartość $4\times$ stała się \textbf{standardem} w prawie wszystkich Transformerach
(ViT, BERT, GPT, LLaMA, \ldots).

\textbf{Ile to parametrów?}
\begin{align}
\mathbf{W}_1&: \; 1536 \times 384 = 589\,824 \text{ wag} + 1536 \text{ biasów} \nonumber\\
\mathbf{W}_2&: \; 384 \times 1536 = 589\,824 \text{ wag} + 384 \text{ biasy} \nonumber\\
&\text{Razem na 1 blok MLP:} \approx \mathbf{1.18\text{M}} \text{ parametrów}
\end{align}

Przy $12$ blokach: $12 \times 1.18\text{M} \approx 14.2\text{M}$
-- to \textbf{ponad połowa} wszystkich $\sim 22$M parametrów ViT-Small!

\end{tcolorbox}

% -- LAYERNORM (przed pierwszym użyciem w równaniu rezydualnym) --
\begin{tcolorbox}[colback=gray!5, colframe=gray!60!black, title=\textbf{Co to jest LayerNorm?}, breakable]

Zanim pokażemy pełne równanie bloku Transformera, musimy wyjaśnić operację
$\text{LayerNorm}$, która pojawia się w nim dwukrotnie.

\subsubsection{Po co normalizacja w Transformerze?}

\textbf{Problem}: po wielu warstwach wartości w wektorze $\mathbf{h}$ mogą stać się bardzo duże
(np.\ rzędu tysięcy) lub bardzo małe (rzędu $10^{-6}$). To destabilizuje trening --
softmax i GELU działają dobrze tylko dla wartości w ``rozsądnym'' zakresie.

Konkretnie, bez normalizacji:
\begin{itemize}[leftmargin=2em]
\item \textbf{Softmax w attention} -- jeśli wartości $\mathbf{Q}\mathbf{K}^\top$ są rzędu tysięcy,
      softmax daje rozkład prawie jedynkowo skoncentrowany na jednym tokenie
      (tzw.\ \textit{attention collapse}), co niszczy zdolność do łączenia informacji.
\item \textbf{GELU w MLP} -- dla $|x| \gg 3$ GELU zachowuje się prawie liniowo,
      więc warstwa MLP traci swoją nieliniowość i staje się zwykłą transformacją afiniczną.
\item \textbf{Gradienty} -- duże wartości aktywacji powodują duże gradienty (\textit{exploding gradients}),
      a bardzo małe wartości prowadzą do zaniku gradientów (\textit{vanishing gradients}).
      W obu przypadkach trening staje się niestabilny.
\end{itemize}

\textbf{Rozwiązanie}: LayerNorm \textbf{normalizuje} wektor do średniej $\approx 0$
i wariancji $\approx 1$ przed każdą operacją, utrzymując wartości w zakresie,
w~którym softmax, GELU i gradienty działają prawidłowo.

\subsubsection{Wzór krok po kroku}

Dany wektor $\mathbf{h} = (h_1, h_2, \ldots, h_d) \in \mathbb{R}^d$ (w ViT-Small $d = 384$):
\begin{itemize}[leftmargin=2em]
\item $\mathbf{h}$ -- cały wektor embeddingu \textbf{jednego tokenu} (wszystkie 384 liczb naraz),
\item $h_i$ -- \textbf{i-ta wartość} w tym wektorze (jedna liczba, np.\ $h_1$ = pierwsza, $h_{384}$ = ostatnia).
\end{itemize}

\textbf{Krok 1 -- Średnia} (centrowanie):
\[
\mu = \frac{1}{d}\sum_{i=1}^{d} h_i
\]
Obliczamy jedną liczbę $\mu$ -- średnią po \textit{wszystkich} $d$ wymiarach tokenu.
Intuicja: $\mu$ mierzy ``ogólny poziom'' aktywacji -- czy wartości są globalnie przesunięte
w górę czy w dół.

\textbf{Krok 2 -- Wariancja} (jak bardzo wartości ``rozbiegają się'' od średniej):
\[
\sigma^2 = \frac{1}{d}\sum_{i=1}^{d} (h_i - \mu)^2
\]
Jedna liczba $\sigma^2$ mierzy ``rozrzut'' wartości wokół średniej.
Jeśli $\sigma^2$ jest duże, wartości mocno się różnią; jeśli małe -- są blisko siebie.

\textbf{Krok 3 -- Normalizacja} (centruj + podziel przez odchylenie):
\[
\hat{h}_i = \frac{h_i - \mu}{\sqrt{\sigma^2 + \epsilon}}
\]
\begin{itemize}[leftmargin=2em]
\item Odejmujemy $\mu$ -- \textbf{centrujemy} wektor (średnia $\to 0$).
\item Dzielimy przez $\sqrt{\sigma^2 + \epsilon}$ -- \textbf{skalujemy} do wariancji $\approx 1$.
\item $\epsilon = 10^{-5}$ -- mała stała zapobiegająca dzieleniu przez zero
      (gdyby $\sigma^2 = 0$, czyli wszystkie $h_i$ były identyczne).
\end{itemize}

Po tym kroku: $\hat{\mathbf{h}}$ ma średnią $= 0$ i wariancję $= 1$ (znormalizowany wektor).

\textbf{Krok 4 -- Skalowanie i przesunięcie} (uczone parametry $\gamma$, $\beta$):
\begin{equation}
\text{LayerNorm}(\mathbf{h})_i = \gamma_i \cdot \hat{h}_i + \beta_i
\label{eq:layernorm_elementwise}
\end{equation}

Parametry $\gamma_i, \beta_i \in \mathbb{R}$ (po jednym na każdy z $d$ wymiarów) są \textbf{uczone} --
sieć sama decyduje, jaka skala i przesunięcie są optymalne dla każdego wymiaru.

Inicjalizacja: $\gamma_i = 1$, $\beta_i = 0$ dla wszystkich $i$
-- na starcie treningu LayerNorm po prostu zwraca $\hat{\mathbf{h}}$ bez zmian.

\textbf{Wzór wektorowy} (cała operacja w jednej linii):
\begin{equation}
\boxed{
\text{LayerNorm}(\mathbf{h}) = \boldsymbol{\gamma} \odot \frac{\mathbf{h} - \mu \cdot \mathbf{1}}{\sqrt{\sigma^2 + \epsilon}} + \boldsymbol{\beta}
}
\label{eq:layernorm_vector}
\end{equation}

gdzie $\odot$ oznacza mnożenie element-po-elemencie (Hadamard product),
$\boldsymbol{\gamma} = (\gamma_1, \ldots, \gamma_d)$,
$\boldsymbol{\beta} = (\beta_1, \ldots, \beta_d)$,
$\mathbf{1} = (1, \ldots, 1) \in \mathbb{R}^d$.

\subsubsection{Dlaczego uczone $\gamma$ i $\beta$?}

Na pierwszy rzut oka uczone $\gamma$ i $\beta$ wydają się sprzeczne z celem normalizacji
-- skoro właśnie wymusiliśmy średnią $0$ i wariancję $1$, po co pozwalać sieci to zmienić?

\textbf{Odpowiedź}: normalizacja do $\hat{\mathbf{h}}$ jest potrzebna, żeby ustabilizować
\textit{przepływ gradientów} przez sieć. Ale \textit{optymalna reprezentacja} dla konkretnej
warstwy niekoniecznie ma średnią $0$ i wariancję $1$ w każdym wymiarze.

Parametry $\gamma$, $\beta$ rozwiązują oba problemy jednocześnie:
\begin{itemize}[leftmargin=2em]
\item \textbf{Stabilizacja}: krok normalizacji (kroki 1--3) gwarantuje, że wartości
      wchodzące do skalowania są zawsze w kontrolowanym zakresie, niezależnie od tego,
      jak duże lub małe były oryginalne $h_i$.
\item \textbf{Ekspresywność}: $\gamma_i$ i $\beta_i$ pozwalają sieci przywrócić
      \textit{dowolną} skalę i przesunięcie, jeśli to konieczne.
      W szczególności, jeśli $\gamma_i = \sigma$ i $\beta_i = \mu$,
      LayerNorm staje się operacją tożsamościową (nic nie zmienia).
      Ale w praktyce sieć uczy się innych, optymalnych wartości.
\end{itemize}

\subsubsection{Ile parametrów dodaje LayerNorm?}

W ViT-Small ($d = 384$) każda warstwa LayerNorm ma:
\[
\underbrace{384}_{\gamma} + \underbrace{384}_{\beta} = 768 \text{ parametrów}
\]
Każdy blok Transformera zawiera \textbf{2 LayerNormy} (przed Self-Attention i przed MLP), więc:
\[
12 \text{ bloków} \times 2 \times 768 = 18{,}432 \text{ parametrów}
\]
To mniej niż $0.1\%$ wszystkich $\sim$22M parametrów ViT-Small -- niemal zerowy koszt
w zamian za stabilny trening.

\subsubsection{Przykład liczbowy}

Niech $\mathbf{h} = (2.0, \; 4.0, \; 6.0, \; 8.0)$ (uproszczenie -- $d = 4$ zamiast 384).

\textbf{Kroki 1--3 (normalizacja):}

\begin{center}
\renewcommand{\arraystretch}{1.4}
\begin{tabular}{ll}
\toprule
\textbf{Krok} & \textbf{Obliczenie} \\
\midrule
Średnia $\mu$ & $\frac{2+4+6+8}{4} = 5.0$ \\[4pt]
Wariancja $\sigma^2$ & $\frac{(2{-}5)^2 + (4{-}5)^2 + (6{-}5)^2 + (8{-}5)^2}{4} = \frac{9+1+1+9}{4} = 5.0$ \\[4pt]
Odchylenie $\sqrt{\sigma^2}$ & $\sqrt{5.0} \approx 2.236$ \\[4pt]
$\hat{h}_1 = \frac{2.0 - 5.0}{2.236}$ & $= -1.34$ \\[2pt]
$\hat{h}_2 = \frac{4.0 - 5.0}{2.236}$ & $= -0.45$ \\[2pt]
$\hat{h}_3 = \frac{6.0 - 5.0}{2.236}$ & $= +0.45$ \\[2pt]
$\hat{h}_4 = \frac{8.0 - 5.0}{2.236}$ & $= +1.34$ \\
\bottomrule
\end{tabular}
\end{center}

\textbf{Wynik normalizacji}: $(2, 4, 6, 8) \to (-1.34, \; -0.45, \; +0.45, \; +1.34)$

Sprawdzenie: średnia $= \frac{-1.34 -0.45 +0.45 +1.34}{4} = 0$ \checkmark, \;
wariancja $= \frac{1.34^2 + 0.45^2 + 0.45^2 + 1.34^2}{4} = \frac{1.80 + 0.20 + 0.20 + 1.80}{4} = 1.0$ \checkmark

\textbf{Krok 4 (skalowanie i przesunięcie):}

Na starcie treningu $\gamma = (1,1,1,1)$, $\beta = (0,0,0,0)$, więc wynik $= \hat{\mathbf{h}}$
(tożsamość). Po kilku epokach sieć może nauczyć się np.\
$\boldsymbol{\gamma} = (2.0, \; 0.5, \; 1.0, \; 1.5)$,
$\boldsymbol{\beta} = (0.1, \; -0.3, \; 0.0, \; 0.2)$:

\begin{center}
\renewcommand{\arraystretch}{1.4}
\begin{tabular}{lll}
\toprule
\textbf{Wymiar} & \textbf{Obliczenie} & \textbf{Wynik} \\
\midrule
$i=1$ & $2.0 \times (-1.34) + 0.1$ & $= -2.58$ \\
$i=2$ & $0.5 \times (-0.45) + (-0.3)$ & $= -0.52$ \\
$i=3$ & $1.0 \times (+0.45) + 0.0$ & $= +0.45$ \\
$i=4$ & $1.5 \times (+1.34) + 0.2$ & $= +2.21$ \\
\bottomrule
\end{tabular}
\end{center}

Zauważ, że po zastosowaniu $\gamma$ i $\beta$ średnia i wariancja \textbf{nie są} już
$0$ i $1$ -- i to jest zamierzone. Normalizacja w krokach 1--3 stabilizuje gradienty,
a uczone $\gamma$, $\beta$ przywracają sieci pełną ekspresywność.

\subsubsection{Gdzie LayerNorm działa w Transformerze?}

W ViT z \textit{Pre-Norm} (standard od 2020 r.) LayerNorm jest stosowany \textbf{przed}
każdą operacją, a nie po niej:

\begin{center}
\renewcommand{\arraystretch}{1.3}
\begin{tabular}{cll}
\toprule
\textbf{Nr} & \textbf{LayerNorm} & \textbf{Co normalizuje?} \\
\midrule
1 & Przed Self-Attention & Wejście do $Q, K, V$ -- stabilizuje iloczyny skalarowe $QK^\top$ \\
2 & Przed MLP & Wejście do warstw liniowych -- stabilizuje GELU \\
\bottomrule
\end{tabular}
\end{center}

Każda z tych dwóch normalizacji ma \textbf{własne, niezależne} parametry $\gamma$ i $\beta$.
Dzięki temu sieć może nauczyć się różnych skal dla attention (gdzie ważna jest
``podobieństwo między tokenami'') i MLP (gdzie ważne jest ``przetwarzanie cech'').

Ponadto, w ViT istnieje jeszcze \textbf{trzeci LayerNorm} -- zastosowany na samym końcu,
po ostatnim (12-tym) bloku Transformera, bezpośrednio przed ekstrakcją tokenu [CLS]
jako końcowego embeddingu. Stabilizuje on skalę wyjściowej reprezentacji.

\subsubsection{Pre-Norm vs Post-Norm}

Oryginalny Transformer (Vaswani et~al., 2017) stosował LayerNorm \textbf{po} dodaniu residuala
(\textit{Post-Norm}):
\[
\mathbf{h}' = \text{LayerNorm}(\mathbf{h} + \text{Self-Attention}(\mathbf{h}))
\]

Współczesne ViT (od 2020 r.) stosują LayerNorm \textbf{przed} operacją (\textit{Pre-Norm}):
\[
\mathbf{h}' = \mathbf{h} + \text{Self-Attention}(\text{LayerNorm}(\mathbf{h}))
\]

\begin{center}
\renewcommand{\arraystretch}{1.3}
\begin{tabular}{lcc}
\toprule
& \textbf{Post-Norm} & \textbf{Pre-Norm} \\
\midrule
Stabilność treningu & trudniejsza & \textbf{łatwiejsza} \\
Potrzeba warm-up? & tak, konieczny & mniej wrażliwy \\
Przepływ gradientów & gradient przechodzi przez LN & \textbf{gradient ma ścieżkę ``na skróty''} \\
Standard w ViT? & nie & \textbf{tak} \\
\bottomrule
\end{tabular}
\end{center}

Kluczowa przewaga Pre-Norm: w połączeniu rezydualnym $\mathbf{h}' = \mathbf{h} + f(\text{LN}(\mathbf{h}))$
gradient z lossu przepływa \textbf{bezpośrednio} przez ``$+\;\mathbf{h}$'' (ścieżka rezydualna)
bez przechodzenia przez LayerNorm. To zapobiega zanikowi gradientu w głębokich sieciach.

\subsubsection{Dlaczego LayerNorm a nie BatchNorm?}

\begin{center}
\renewcommand{\arraystretch}{1.3}
\begin{tabular}{lcc}
\toprule
& \textbf{BatchNorm} & \textbf{LayerNorm} \\
\midrule
Normalizuje po & próbkach w batchu & wymiarach jednego tokenu \\
Statystyki ($\mu, \sigma^2$) & z batcha ($B$ próbek) & z jednego tokenu ($d$ wymiarów) \\
Zależy od batch size? & tak & \textbf{nie} \\
Działa przy batch=1? & źle & \textbf{dobrze} \\
Wymaga running mean/var? & tak (inferencja) & \textbf{nie} \\
Standard w Transformerach? & nie & \textbf{tak} \\
\bottomrule
\end{tabular}
\end{center}

Kluczowe różnice:
\begin{itemize}[leftmargin=2em]
\item \textbf{BatchNorm} oblicza $\mu$ i $\sigma^2$ \textbf{po batchu} -- uśrednia wartości
      wymiaru $i$ po wszystkich $B$ próbkach. To powoduje, że wynik dla jednego obrazu
      zależy od tego, jakie inne obrazy są w batchu -- niedeterminizm.
\item \textbf{LayerNorm} oblicza $\mu$ i $\sigma^2$ \textbf{wewnątrz jednego tokenu} --
      uśrednia po $d = 384$ wymiarach tego samego wektora.
      Wynik jest w~pełni \textbf{deterministyczny}: ten sam obraz zawsze
      daje ten sam wynik, niezależnie od reszty batcha.
\item W Transformerach sekwencja tokenów ma \textbf{zmienną długość}
      (różne obrazy mogą mieć różną liczbę patchy po maskingu).
      BatchNorm wymagałby uśredniania po tokenach z różnych pozycji,
      co nie ma sensu semantycznego. LayerNorm tego problemu nie ma.
\end{itemize}

\subsubsection{Podsumowanie LayerNorm}

\begin{center}
\renewcommand{\arraystretch}{1.3}
\begin{tabular}{ll}
\toprule
\textbf{Aspekt} & \textbf{Szczegóły} \\
\midrule
Wejście & $\mathbf{h} \in \mathbb{R}^d$ (jeden token, $d = 384$ w ViT-Small) \\
Wyjście & $\text{LN}(\mathbf{h}) \in \mathbb{R}^d$ (ten sam rozmiar) \\
Uczone parametry & $\boldsymbol{\gamma}, \boldsymbol{\beta} \in \mathbb{R}^d$ ($2d = 768$ skalarów) \\
Inicjalizacja & $\boldsymbol{\gamma} = \mathbf{1}$, $\boldsymbol{\beta} = \mathbf{0}$ (tożsamość) \\
Gdzie w bloku & Przed Self-Attention i przed MLP (Pre-Norm) \\
Koszt & $<0.1\%$ parametrów, praktycznie zerowy koszt obliczeniowy \\
\bottomrule
\end{tabular}
\end{center}

\end{tcolorbox}

\subsubsection{Pełny blok z residualami i normalizacją}

Każdy z 12 bloków Transformera wykonuje dwie operacje -- Self-Attention i MLP --
ale \textbf{nie zastępuje} wejścia wynikiem. Zamiast tego \textbf{dodaje} wynik do wejścia:

\begin{equation}
\boxed{
\begin{aligned}
\mathbf{h}' &= \mathbf{h} + \text{Self-Attention}(\text{LayerNorm}(\mathbf{h})) \\
\mathbf{h}^{(\ell+1)} &= \mathbf{h}' + \text{MLP}(\text{LayerNorm}(\mathbf{h}'))
\end{aligned}
}
\end{equation}

To ``$+$'' w równaniu to \textbf{połączenie rezydualne} (residual connection) --
najważniejszy trick, który umożliwia trenowanie głębokich sieci.

% -- SCHEMAT BLOKOWY --
\begin{tcolorbox}[breakable, colback=blue!3, colframe=blue!60!black, title=\textbf{Schemat przepływu przez jeden blok Transformera}]

\begin{center}
\begin{tabular}{c}
\texttt{Wejście: h} \\[4pt]
$\downarrow$ \\[2pt]
\fbox{\texttt{LayerNorm}} \\[2pt]
$\downarrow$ \\[2pt]
\fbox{\texttt{Self-Attention}} \\[2pt]
$\downarrow$ \quad wynik = $\Delta_1$ \\[2pt]
\texttt{h' = h + } $\Delta_1$ \quad $\longleftarrow$ \textit{residual: dodaj do wejścia!} \\[8pt]
$\downarrow$ \\[2pt]
\fbox{\texttt{LayerNorm}} \\[2pt]
$\downarrow$ \\[2pt]
\fbox{\texttt{MLP}} \\[2pt]
$\downarrow$ \quad wynik = $\Delta_2$ \\[2pt]
\texttt{h\textsuperscript{(next)} = h' + } $\Delta_2$ \quad $\longleftarrow$ \textit{residual: dodaj do wejścia!} \\[4pt]
$\downarrow$ \\[2pt]
\texttt{Wyjście: h\textsuperscript{(next)}} \\
\end{tabular}
\end{center}

\textbf{Kluczowa obserwacja:} Attention i MLP nie produkują nowej wartości
od zera -- produkują jedynie \textbf{poprawkę} ($\Delta$), która jest
\textbf{dodawana} do oryginalnego sygnału.

\end{tcolorbox}

% -- CHAIN RULE OD PODSTAW --
\begin{tcolorbox}[breakable, colback=blue!3, colframe=blue!60!black, title=\textbf{Wstawka: reguła łańcuchowa (chain rule) -- od jednej funkcji do 12 warstw}]

Zanim pokażemy gradient w sieci, zbudujmy intuicję \textbf{od zera}.

\medskip
\textbf{Krok 0: Jedna funkcja}

Masz $y = f(x)$. Pochodna mówi ``jak zmiana $x$ wpływa na $y$'':
\[
\frac{\partial y}{\partial x} = f'(x)
\]
To znasz -- nic nowego.

\medskip
\textbf{Krok 1: Dwie funkcje złożone}

Teraz $y = g\!\big(f(x)\big)$ -- najpierw $f$, potem $g$. Pytamy: ``jak zmiana $x$ wpływa na $y$?''

Odpowiedź: zmiana $x$ najpierw zmienia $f$, a potem zmiana $f$ zmienia $g$. \textbf{Mnożymy} te dwa efekty:
\[
\frac{\partial y}{\partial x}
= \frac{\partial g}{\partial f} \cdot \frac{\partial f}{\partial x}
\]

\textit{Analogia}: Jeśli 1 obrót pedału kręci łańcuchem 3 razy ($\partial f/\partial x = 3$),
a 1 obrót łańcucha kręci kołem 2 razy ($\partial g/\partial f = 2$),
to 1 obrót pedału kręci kołem $3 \times 2 = 6$ razy.

\medskip
\textbf{Krok 2: Trzy funkcje = trzy czynniki}

$y = h\!\big(g\!\big(f(x)\big)\big)$ -- trzy ``warstwy''. Chain rule daje \textbf{iloczyn trzech} pochodnych:
\[
\frac{\partial y}{\partial x}
= \frac{\partial h}{\partial g} \cdot \frac{\partial g}{\partial f} \cdot \frac{\partial f}{\partial x}
\]

Wzorzec: \textbf{ile funkcji złożonych, tyle czynników w iloczynie}.

\medskip
\textbf{Krok 3: Symbol $\prod$ -- zwięzły zapis iloczynu}

Tak jak $\sum$ oznacza ``dodawaj kolejne elementy'',
$\prod$ oznacza ``\textbf{mnóż} kolejne elementy'':

\begin{center}
\renewcommand{\arraystretch}{1.5}
\begin{tabular}{lcl}
$\displaystyle\sum_{i=1}^{3} a_i$ & $=$ & $a_1 + a_2 + a_3$ \quad (suma) \\[6pt]
$\displaystyle\prod_{i=1}^{3} a_i$ & $=$ & $a_1 \cdot a_2 \cdot a_3$ \quad (iloczyn) \\
\end{tabular}
\end{center}

Więc chain rule dla $L$ złożonych funkcji:
\[
\frac{\partial y}{\partial x} = \prod_{\ell=1}^{L} \frac{\partial f_\ell}{\partial f_{\ell-1}}
\quad = \quad
\underbrace{\frac{\partial f_L}{\partial f_{L-1}} \cdot \frac{\partial f_{L-1}}{\partial f_{L-2}}
\cdot \;\ldots\; \cdot \frac{\partial f_1}{\partial x}}_{L \text{ czynników}}
\]

\end{tcolorbox}

% -- DLACZEGO RESIDUALNE --
\begin{tcolorbox}[breakable, colback=red!3, colframe=red!60!black, title=\textbf{Problem: dlaczego bez residuali sieć nie działa?}]

\textbf{Bez połączeń rezydualnych} (naiwne podejście) każda warstwa \textit{zastępuje} wejście:
\[
\mathbf{h}^{(\ell+1)} = f_\ell(\mathbf{h}^{(\ell)})
\]

Rozpiszmy to dla 3 warstw, żeby zobaczyć jak chain rule się składa:
\begin{align*}
\mathbf{h}^{(1)} &= f_0(\mathbf{h}^{(0)}) \\
\mathbf{h}^{(2)} &= f_1(\mathbf{h}^{(1)}) = f_1\!\big(f_0(\mathbf{h}^{(0)})\big) \\
\mathbf{h}^{(3)} &= f_2(\mathbf{h}^{(2)}) = f_2\!\big(f_1\!\big(f_0(\mathbf{h}^{(0)})\big)\big)
\end{align*}

Gradient lossu $\mathcal{L}$ po wejściu $\mathbf{h}^{(0)}$ -- stosujemy chain rule (mnożymy pochodne):
\[
\frac{\partial \mathcal{L}}{\partial \mathbf{h}^{(0)}}
= \frac{\partial \mathcal{L}}{\partial \mathbf{h}^{(3)}}
\cdot \underbrace{\frac{\partial f_2}{\partial \mathbf{h}^{(2)}}}_{\text{warstwa 3}}
\cdot \underbrace{\frac{\partial f_1}{\partial \mathbf{h}^{(1)}}}_{\text{warstwa 2}}
\cdot \underbrace{\frac{\partial f_0}{\partial \mathbf{h}^{(0)}}}_{\text{warstwa 1}}
\]

Dla 12 warstw -- 12 czynników, zapisujemy zwięźle jako $\prod$:
\[
\frac{\partial \mathcal{L}}{\partial \mathbf{h}^{(0)}}
= \frac{\partial \mathcal{L}}{\partial \mathbf{h}^{(12)}} \cdot
\prod_{\ell=0}^{11} \frac{\partial f_\ell}{\partial \mathbf{h}^{(\ell)}}
\]

\textbf{Problem:} to jest \textbf{iloczyn} 12 liczb. Jeśli każda jest $< 1$, wynik znika:
\[
\underbrace{0.5 \times 0.5 \times \ldots \times 0.5}_{12 \text{ razy}} = 0.5^{12} = 0.000244
\quad \rightarrow \quad \text{gradient \textbf{zanika!}}
\]
Pierwsze warstwy prawie się nie uczą. A jeśli każda $> 1$:
\[
2.0^{12} = 4096 \quad \rightarrow \quad \text{gradient \textbf{eksploduje!}}
\]
\end{tcolorbox}

% -- Z RESIDUALAMI --
\begin{tcolorbox}[breakable, colback=green!3, colframe=green!60!black, title=\textbf{Rozwiązanie: połączenie rezydualne jako ``autostrada'' dla gradientu}]

\textbf{Z połączeniem rezydualnym} warstwa \textit{dodaje} poprawkę zamiast zastępować:
\[
\mathbf{h}^{(\ell+1)} = \mathbf{h}^{(\ell)} + f_\ell(\mathbf{h}^{(\ell)})
\]

\textbf{Krok A: Pochodna jednej warstwy residualnej.}
Mamy sumę dwóch wyrazów: $\mathbf{h}^{(\ell)}$ (przechodzi ``na skróty'') i $f_\ell(\mathbf{h}^{(\ell)})$ (poprawka).
Pochodna sumy = suma pochodnych:
\[
\frac{\partial \mathbf{h}^{(\ell+1)}}{\partial \mathbf{h}^{(\ell)}}
= \underbrace{\frac{\partial \mathbf{h}^{(\ell)}}{\partial \mathbf{h}^{(\ell)}}}_{= \; 1 \;\text{(skip connection!)}}
+ \frac{\partial f_\ell}{\partial \mathbf{h}^{(\ell)}}
= 1 + \frac{\partial f_\ell}{\partial \mathbf{h}^{(\ell)}}
\]

\textbf{Krok B: Składamy 3 warstwy} (chain rule -- mnożymy pochodne):
\begin{align*}
\frac{\partial \mathbf{h}^{(3)}}{\partial \mathbf{h}^{(0)}}
&= \frac{\partial \mathbf{h}^{(3)}}{\partial \mathbf{h}^{(2)}}
\cdot \frac{\partial \mathbf{h}^{(2)}}{\partial \mathbf{h}^{(1)}}
\cdot \frac{\partial \mathbf{h}^{(1)}}{\partial \mathbf{h}^{(0)}} \\[6pt]
&= \left(1 + \frac{\partial f_2}{\partial \mathbf{h}^{(2)}}\right)
\cdot \left(1 + \frac{\partial f_1}{\partial \mathbf{h}^{(1)}}\right)
\cdot \left(1 + \frac{\partial f_0}{\partial \mathbf{h}^{(0)}}\right)
\end{align*}

\textbf{Krok C: Uogólnienie na 12 warstw} -- zapisujemy zwięźle jako $\prod$:
\[
\boxed{
\frac{\partial \mathcal{L}}{\partial \mathbf{h}^{(0)}}
= \prod_{\ell=0}^{11} \left(1 + \frac{\partial f_\ell}{\partial \mathbf{h}^{(\ell)}}\right)
}
\]

\textbf{Krok D: Dlaczego to działa?}
Nawet jeśli $\frac{\partial f_\ell}{\partial \mathbf{h}^{(\ell)}} \approx 0$ (warstwa ``nic nie robi''):
\[
\prod_{\ell=0}^{11} (1 + 0) = 1^{12} = 1 \quad \text{-- gradient \textbf{przepływa bez strat!}}
\]

\vspace{6pt}
\begin{center}
\renewcommand{\arraystretch}{1.3}
\begin{tabular}{|l|c|c|}
\hline
& \textbf{Bez residuali} & \textbf{Z residualami} \\
\hline
Wzór na gradient & $\displaystyle\prod \frac{\partial f_\ell}{\partial h}$ & $\displaystyle\prod \!\left(1 + \frac{\partial f_\ell}{\partial h}\right)$ \\[8pt]
\hline
Jeśli $\frac{\partial f_\ell}{\partial h} = 0.5$ & $0.5^{12} = 0.00024$ & $1.5^{12} = 129.7$ \\
\hline
Jeśli $\frac{\partial f_\ell}{\partial h} = 0$ & $0^{12} = 0$ (martwa sieć) & $1^{12} = 1$ (OK!) \\
\hline
\end{tabular}
\end{center}
\end{tcolorbox}

% -- ANALOGIA --
\begin{tcolorbox}[breakable, colback=yellow!5, colframe=orange!70!black, title=\textbf{Analogia: autostrada z $12$ stacjami}]

Wyobraź sobie \textbf{autostradę} prowadzącą sygnał od wejścia do wyjścia:

\begin{center}
\begin{tabular}{ccccccc}
$\mathbf{h}^{(0)}$ & $\xrightarrow{\text{autostrada}}$ & $\mathbf{h}^{(1)}$
& $\xrightarrow{\text{autostrada}}$ & $\cdots$
& $\xrightarrow{\text{autostrada}}$ & $\mathbf{h}^{(12)}$ \\[4pt]
& $\uparrow$ \small{+$\Delta_1$} & & $\uparrow$ \small{+$\Delta_2$} & & $\uparrow$ \small{+$\Delta_{12}$} & \\[2pt]
& \fbox{\small{Blok 1}} & & \fbox{\small{Blok 2}} & & \fbox{\small{Blok 12}} & \\
\end{tabular}
\end{center}

\begin{itemize}[leftmargin=2em]
\item \textbf{Autostrada}: oryginalny sygnał \textbf{zawsze przepływa} od $\mathbf{h}^{(0)}$ do $\mathbf{h}^{(12)}$.
\item \textbf{Stacje} (bloki): każda dodaje swoją poprawkę $\Delta_\ell$, ale \textbf{nie blokuje} ruchu.
\item Nawet jeśli stacja ``zepsuje się'' ($\Delta_\ell \approx 0$), sygnał jedzie dalej.
\item Gradient wraca tą samą autostradą -- ma \textbf{gwarantowaną drogę} do pierwszych warstw.
\end{itemize}

\textbf{Bez residuali}: każda stacja to \textbf{bramka} -- sygnał musi przejść przez każdą.
Jeśli jedna bramka blokuje, wszystko staje.
\end{tcolorbox}

% -- PODSUMOWANIE PEŁNEGO BLOKU --
\begin{tcolorbox}[breakable, colback=blue!3, colframe=blue!70!black, title=\textbf{Podsumowanie: co robi jeden blok Transformera}]
\begin{enumerate}[leftmargin=2em]
\item \textbf{LayerNorm} -- stabilizuje wartości
\item \textbf{Self-Attention} -- tokeny ``rozmawiają'' ze sobą $\rightarrow$ poprawka $\Delta_1$
\item \textbf{Residual} -- dodaj $\Delta_1$ do wejścia: $\mathbf{h}' = \mathbf{h} + \Delta_1$
\item \textbf{LayerNorm} -- ponowna stabilizacja
\item \textbf{MLP} -- każdy token przetwarzany osobno $\rightarrow$ poprawka $\Delta_2$
\item \textbf{Residual} -- dodaj $\Delta_2$: $\mathbf{h}^{(\ell+1)} = \mathbf{h}' + \Delta_2$
\end{enumerate}
Ten cykl powtarza się \textbf{12 razy} -- na końcu token [CLS] zawiera
\textbf{bogate, wielopoziomowe} cechy wizualne.
\end{tcolorbox}

\begin{figure}[H]
\centering
\includegraphics[width=\textwidth]{figures/vit_pipeline.pdf}
\caption{Pełny pipeline ViT-Small. Lewo: przepływ danych od obrazu do embeddingu.
Prawo: szczegóły self-attention (Q, K, V) i MLP wewnątrz bloku.
Na dole: notatka o inicjalizacji losowej.}
\label{fig:pipeline}
\end{figure}

\subsection{Krok 6: Wyjście — embedding}

Po $12$ blokach token [CLS] ``widział'' cały obraz przez wielokrotne attention.
Jego wartość to \textbf{globalny embedding obrazu}:

\begin{equation}
\boxed{
\mathbf{z} = f_\theta(\mathbf{x}) = \mathbf{h}_{\text{CLS}}^{(12)} \in \mathbb{R}^{384}
}
\end{equation}

Co oznacza ten wzór?
\begin{itemize}[leftmargin=2em]
\item $\mathbf{x}$ -- wejściowy obraz (np.\ $224 \times 224$ pikseli),
\item $f_\theta$ -- cała sieć ViT z parametrami $\theta$ (wszystkie $\mathbf{W}_Q, \mathbf{W}_K, \mathbf{W}_V,
      \mathbf{W}_{\text{patch}}, \mathbf{pos}, \ldots$),
\item $\mathbf{h}_{\text{CLS}}^{(12)}$ -- wartość tokenu [CLS] \textbf{po przejściu przez 12 bloków}
      (nie oryginalna wartość, tylko po 12-krotnym attention + MLP),
\item $\mathbf{z} \in \mathbb{R}^{384}$ -- \textbf{finalny embedding}: 384 liczb opisujących cały obraz.
\end{itemize}

Ten wektor $\mathbf{z}$ to wynik ViT -- jest podawany do LeJEPA (prediction loss + SIGReg)
podczas treningu, a po treningu używany do zadań downstream (klasyfikacja, k-NN).

\textbf{Pozostałe 196 tokenów} (patche) po 12 warstwach też mają bogate reprezentacje,
ale \textbf{nie są używane} -- tylko [CLS] jest ``oficjalnym'' wyjściem ViT.

\begin{keyinsight}[Po co nam embedding $\mathbf{z}$?]
Embedding $\mathbf{z}$ to \textbf{jedyne}, co zostaje z całego ViT po treningu ---
384 liczby, które kodują sens obrazu. Cała reszta (patche, attention, MLP)
to tylko \textit{maszyneria} produkująca ten wektor. Po co?

\begin{enumerate}[leftmargin=2em]
  \item \textbf{Downstream}: Po treningu ViT jest \textit{zamrożony}.
        Nowy obraz wchodzi, $\mathbf{z}$ wychodzi, a mały klasyfikator
        (linear probe / k-NN) podejmuje decyzję \textit{wyłącznie} na podstawie $\mathbf{z}$.
        Jeśli $\mathbf{z}$ jest dobry --- klasyfikacja działa.
        Jeśli $\mathbf{z}$ jest zły --- żaden klasyfikator nie pomoże.

  \item \textbf{Podobieństwo}: Dwa obrazy o podobnym znaczeniu
        (np.\ dwie klatki z fazy phaco) dadzą wektory $\mathbf{z}$ \textit{bliskie sobie}
        w przestrzeni $\mathbb{R}^{384}$; obrazy różne --- daleko od siebie.
        To pozwala np.\ wyszukiwać podobne klatki (retrieval) przez zwykłe
        liczenie odległości $\|\mathbf{z}_1 - \mathbf{z}_2\|$.

  \item \textbf{SIGReg}: Cały sens regularyzacji SIGReg polega na wymuszeniu,
        żeby zbiór wektorów $\{\mathbf{z}_1, \ldots, \mathbf{z}_N\}$ z batcha
        miał rozkład $\mathcal{N}(\mathbf{0}, \mathbf{I})$.
        Gdyby embeddingi się \textit{skolapsowały} ($\mathbf{z} = \text{const}$),
        żaden downstream by nie działał.
        Gdyby były \textit{anizotropowe} (rozciągnięte wzdłuż jednej osi),
        zadania wymagające separacji w ``ściśniętym'' kierunku byłyby skazane na porażkę.
        Izotropowy Gauss gwarantuje, że $\mathbf{z}$ jest \textbf{równie użyteczny}
        w każdym kierunku przestrzeni --- niezależnie od tego, jakie zadanie downstream dostaniemy.
\end{enumerate}

Krótko: $\mathbf{z}$ to \textbf{produkt końcowy}, a cały LeJEPA + SIGReg istnieje po to,
żeby ten produkt był jak najlepszej jakości.
\end{keyinsight}

\subsection{Trening od zera: od szumu do sensownych cech}

Gdy inicjalizujemy ViT od zera (\textbf{random init}):

\begin{enumerate}
  \item \textbf{Epoka 0}: Wszystkie macierze ($\mathbf{W}_Q, \mathbf{W}_K, \mathbf{W}_V, \mathbf{W}_{\text{patch}}, \ldots$)
        mają \textbf{losowe wartości}.
        Attention jest \textbf{równomierny} — każdy token ``patrzy'' na wszystkich po równo.
        Embedding = \textbf{losowy szum} (bezużyteczny).

  \item \textbf{Epoki 1--20}: Sieć uczy się podstaw:
  \begin{itemize}
    \item $\mathbf{W}_{\text{patch}}$: jakie cechy wyciągać z pikseli (krawędzie, kolory),
    \item Embeddingi pozycyjne: geometria siatki patchy,
    \item Attention: na co warto patrzeć (a na co nie).
  \end{itemize}

  \item \textbf{Epoki 20--100}: Cechy stają się semantyczne:
  \begin{itemize}
    \item Attention skupia się na istotnych regionach (narzędzie, tęczówka),
    \item {[CLS]} zbiera sensowne podsumowanie sceny,
    \item Embedding $\mathbf{z}$ zaczyna \textbf{separować} różne typy scen.
  \end{itemize}
\end{enumerate}

\begin{figure}[H]
\centering
\includegraphics[width=\textwidth]{figures/vit_attention.pdf}
\caption{\textbf{Lewo}: Mapa attention [CLS] — czerwone patche to te, na które [CLS] ``patrzy'' najsilniej.
\textbf{Środek}: Macierz attention (fragment) — wiersz = kto pyta, kolumna = kto odpowiada.
\textbf{Prawo}: Krzywe treningu od zera — loss maleje, attention staje się ostrzejszy,
jakość embeddingów rośnie.}
\label{fig:attention}
\end{figure}

\begin{warningbox}[Dlaczego self-supervised (bez etykiet)?]
W LeJEPA ViT nie ma etykiet (``to jest incision'', ``to jest phaco'').
Zamiast tego uczy się przez \textbf{predykcję widoków}:
\begin{itemize}
  \item Dostaje dwa ``widoki'' tego samego kadru (różne augmentacje),
  \item Musi nauczyć się, że te widoki \textbf{powinny mieć podobny embedding},
  \item To zmusza ViT do wyciągania \textbf{semantycznych cech} (nie pikseli),
  \item SIGReg dodatkowo wymusza, żeby embeddingi miały rozkład $\mathcal{N}(\mathbf{0}, \mathbf{I})$.
\end{itemize}
\end{warningbox}


\clearpage
%% ============================================================
\section{Teoria wariancji: od podstaw}
\label{sec:wariancja}
%% ============================================================

Zanim przejdziemy do rozkładu Gaussa, musimy zrozumieć pojęcia,
z których jest zbudowany: \textbf{wariancja}, \textbf{kowariancja}
i \textbf{macierz kowariancji}.

\subsection{Wariancja: jak bardzo dane się rozrzucają?}

\begin{definition}[Średnia (wartość oczekiwana)]
Dla $N$ pomiarów $x_1, x_2, \ldots, x_N$:
\begin{equation}
\mu = \frac{1}{N}\sum_{i=1}^{N} x_i
\end{equation}
To ``środek ciężkości'' danych.
\end{definition}

\begin{definition}[Wariancja]
\begin{equation}
\boxed{
\sigma^2 = \frac{1}{N}\sum_{i=1}^{N} (x_i - \mu)^2
}
\end{equation}
Wariancja to \textbf{średni kwadrat odległości} od średniej.
\end{definition}

\textbf{Dlaczego kwadrat?} Bo odchylenia w górę ($x_i > \mu$) i w dół ($x_i < \mu$)
by się nawzajem skasowały. Kwadrat sprawia, że każde odchylenie ``liczy się'' pozytywnie.

\begin{figure}[H]
\centering
\includegraphics[width=\textwidth]{figures/variance_explanation.pdf}
\caption{\textbf{Lewo}: Mała wariancja — punkty skupione blisko średniej.
\textbf{Środek}: Duża wariancja — punkty rozrzucone daleko.
\textbf{Prawo}: Wizualizacja wzoru — fioletowe strzałki to odchylenia $(x_i - \mu)$,
wariancja to średnia ich kwadratów.}
\label{fig:variance}
\end{figure}

\begin{definition}[Odchylenie standardowe]
\begin{equation}
\sigma = \sqrt{\sigma^2}
\end{equation}
To ``typowa odległość'' punktu od średniej, w oryginalnych jednostkach.
\end{definition}

\begin{remark}[Przykład liczbowy]
Dane: $x = \{2, 3, 5, 7, 8\}$.
\begin{align}
\mu &= \frac{2+3+5+7+8}{5} = 5 \\
\sigma^2 &= \frac{(2-5)^2 + (3-5)^2 + (5-5)^2 + (7-5)^2 + (8-5)^2}{5}
= \frac{9 + 4 + 0 + 4 + 9}{5} = \frac{26}{5} = 5.2 \\
\sigma &= \sqrt{5.2} \approx 2.28
\end{align}
Interpretacja: typowy punkt odchyla się o $\approx 2.3$ od średniej $5$.
\end{remark}

\subsection{Kowariancja: czy dwie zmienne ``chodzą razem''?}

Gdy mamy \textbf{dwie} zmienne $z_1$ i $z_2$ (np.\ dwa wymiary embeddingu),
chcemy wiedzieć: \textit{czy gdy $z_1$ rośnie, $z_2$ też rośnie?}

\begin{definition}[Kowariancja]
\begin{equation}
\boxed{
\mathrm{Cov}(z_1, z_2) = \sigma_{12} = \frac{1}{N}\sum_{i=1}^{N}
(z_{1,i} - \mu_1)(z_{2,i} - \mu_2)
}
\end{equation}
\end{definition}

\textbf{Interpretacja znaku}:
\begin{itemize}
  \item $\sigma_{12} > 0$: $z_1$ rośnie $\Rightarrow$ $z_2$ też rośnie (\textbf{korelacja dodatnia}),
  \item $\sigma_{12} = 0$: brak związku liniowego (\textbf{niezależne} lub ortogonalne),
  \item $\sigma_{12} < 0$: $z_1$ rośnie $\Rightarrow$ $z_2$ maleje (\textbf{korelacja ujemna}).
\end{itemize}

\begin{remark}
Wariancja to specjalny przypadek kowariancji zmiennej z samą sobą:
$\mathrm{Cov}(z_1, z_1) = \sigma_1^2$ (wariancja $z_1$).
\end{remark}

\begin{figure}[H]
\centering
\includegraphics[width=\textwidth]{figures/covariance_types.pdf}
\caption{Trzy typy kowariancji. Czerwone strzałki: kierunki główne (wektory własne).
\textbf{Lewo}: $\sigma_{12} > 0$ — punkty ciągną się od lewego-dołu do prawego-góry.
\textbf{Środek}: $\sigma_{12} = 0$ — okrągła chmurka, brak związku.
\textbf{Prawo}: $\sigma_{12} < 0$ — od lewego-góry do prawego-dołu.}
\label{fig:covariance}
\end{figure}

\subsection{Macierz kowariancji: pełny obraz w $K$ wymiarach}

Gdy mamy $K$ zmiennych ($K$ wymiarów embeddingu), potrzebujemy
\textbf{jednej struktury}, która zbiera \textit{wszystkie} wariancje i kowariancje.

\begin{definition}[Macierz kowariancji]
Dla wektora $\mathbf{z} = (z_1, \ldots, z_K)^\top$:
\begin{equation}
\boxed{
\boldsymbol{\Sigma} = \begin{pmatrix}
\sigma_1^2 & \sigma_{12} & \cdots & \sigma_{1K} \\
\sigma_{12} & \sigma_2^2 & \cdots & \sigma_{2K} \\
\vdots & \vdots & \ddots & \vdots \\
\sigma_{1K} & \sigma_{2K} & \cdots & \sigma_K^2
\end{pmatrix}
\in \mathbb{R}^{K \times K}
}
\end{equation}
\begin{itemize}
  \item \textbf{Diagonala}: wariancje $\sigma_k^2$ — rozrzut wzdłuż każdej osi,
  \item \textbf{Pozadiagonala}: kowariancje $\sigma_{ij}$ — powiązania między osiami,
  \item Macierz jest \textbf{symetryczna}: $\sigma_{ij} = \sigma_{ji}$.
\end{itemize}
\end{definition}

\textbf{Przykład 2D}:
\begin{equation}
\boldsymbol{\Sigma} = \begin{pmatrix}
\sigma_1^2 & \sigma_{12} \\
\sigma_{12} & \sigma_2^2
\end{pmatrix}
= \begin{pmatrix}
\text{rozrzut w }z_1 & \text{związek }z_1 \leftrightarrow z_2 \\
\text{związek }z_1 \leftrightarrow z_2 & \text{rozrzut w }z_2
\end{pmatrix}
\end{equation}

\subsection{Wartości własne: co mówią o kształcie danych?}

Macierz kowariancji $\boldsymbol{\Sigma}$ można rozłożyć na \textbf{wartości własne}
$\lambda_1, \ldots, \lambda_K$ i \textbf{wektory własne} $\mathbf{v}_1, \ldots, \mathbf{v}_K$:

\begin{equation}
\boldsymbol{\Sigma}\mathbf{v}_k = \lambda_k \mathbf{v}_k
\end{equation}

\textbf{Co to znaczy?}
\begin{itemize}
  \item $\mathbf{v}_k$: \textbf{kierunek} $k$-tej ``osi'' danych (po obrocie do naturalnych osi),
  \item $\lambda_k$: \textbf{wariancja wzdłuż} tego kierunku — ile ``rozrzutu'' jest w tym kierunku,
  \item $\sqrt{\lambda_k}$: długość $k$-tej półosi elipsy (elipsoidy).
\end{itemize}

\begin{figure}[H]
\centering
\includegraphics[width=\textwidth]{figures/eigenvalues_shape.pdf}
\caption{Wartości własne definiują kształt chmury punktów.
\textbf{Lewo}: $\lambda_1 = \lambda_2 = 1$ $\Rightarrow$ okrąg (\textbf{izotropia!}).
\textbf{Środek}: $\lambda_1 = 4, \lambda_2 = 0.25$ $\Rightarrow$ elipsa wzdłuż osi.
\textbf{Prawo}: $\lambda_1 = 3, \lambda_2 = 0.5$ $\Rightarrow$ obrócona elipsa.
Czerwone strzałki: wektory własne (kierunki), ich długość $\propto \sqrt{\lambda_k}$.}
\label{fig:eigenvalues}
\end{figure}

\subsection{Izotropia = wszystkie wartości własne równe}

\begin{keyinsight}[Kluczowe połączenie z LeJEPA]
\textbf{Izotropowy} rozkład = macierz kowariancji to wielokrotność jednostkowej:
\begin{equation}
\boldsymbol{\Sigma} = \mathbf{I}_K
\quad \iff \quad
\lambda_1 = \lambda_2 = \cdots = \lambda_K = 1
\end{equation}

Co to oznacza geometrycznie:
\begin{itemize}
  \item Dane rozrzucone \textbf{jednakowo} we wszystkich kierunkach,
  \item Chmura punktów to \textbf{hipersfera}, nie elipsoida,
  \item Żaden wymiar embeddingu nie jest ``ważniejszy'' od innego,
  \item \textbf{Kowariancje zerowe}: wymiary są niezależne ($\sigma_{ij} = 0$ dla $i \neq j$).
\end{itemize}

Dlatego LeJEPA wymusza $\boldsymbol{\Sigma} = \mathbf{I}$ za pomocą SIGReg:
żeby \textit{żadne} przyszłe zadanie downstream nie było ``oślepione''
brakiem informacji w którymkolwiek kierunku.
\end{keyinsight}


\clearpage
%% ============================================================
\section{Wyprowadzenie: rozkład Gaussa od 1D do 3D}
\label{sec:wyprowadzenie}
%% ============================================================

Zaczynamy od jednego wymiaru i krok po kroku budujemy intuicję aż do 3D.

% --- 1D ---
\subsection{Punkt wyjścia: Gauss w 1D}

Rozkład normalny jednej zmiennej $z \in \mathbb{R}$ o średniej $\mu$ i wariancji $\sigma^2$:

\begin{equation}
\boxed{
p(z) = \frac{1}{\sqrt{2\pi\sigma^2}}\;\exp\!\left(-\frac{(z-\mu)^2}{2\sigma^2}\right)
}
\tag{1D}
\label{eq:1d}
\end{equation}

\textbf{Skąd ten wzór?} Rozbijamy go na kawałki:

\begin{center}
\renewcommand{\arraystretch}{1.8}
\begin{tabular}{cp{10.5cm}}
\toprule
\textbf{Kawałek} & \textbf{Co robi?} \\
\midrule
$(z - \mu)^2$ &
Kwadrat odległości od średniej. Im dalej od $\mu$, tym większa wartość. \\
$\dfrac{(z-\mu)^2}{2\sigma^2}$ &
Normalizuje odległość przez wariancję.
\textbf{Czym jest $\sigma^2$?} To miara \textbf{rozrzutu} wartości wokół średniej $\mu$
-- jak daleko na lewo i prawo od $\mu$ ``sięga'' rozkład.
Odchylenie standardowe $\sigma = \sqrt{\sigma^2}$ ma tę samą jednostkę co dane
i wyznacza \textbf{charakterystyczną szerokość} rozkładu:
\newline\newline
\begin{tabular}{@{}ll@{}}
$\bullet$ $\sim$68\% wartości leży w $[\mu - \sigma, \;\; \mu + \sigma]$ & (``1 sigma'') \\
$\bullet$ $\sim$95\% wartości leży w $[\mu - 2\sigma, \; \mu + 2\sigma]$ & (``2 sigma'') \\
$\bullet$ $\sim$99.7\% wartości leży w $[\mu - 3\sigma, \; \mu + 3\sigma]$ & (``3 sigma'')
\end{tabular}
\newline\newline
Przykład: jeśli $\mu = 0$ i $\sigma = 1$, to $\sim$68\% wartości leży w $[-1, +1]$.
Jeśli $\sigma = 2$, rozkład jest $2\times$ szerszy: $\sim$68\% w~$[-2, +2]$.
Jeśli $\sigma = 0.5$, rozkład jest $2\times$ węższy: $\sim$68\% w~$[-0.5, +0.5]$.
\newline\newline
\textbf{Dlaczego $2$ w~mianowniku?} Dwójka gwarantuje, że parametr $\sigma^2$ we wzorze
jest \textit{dokładnie równy} wariancji rozkładu, tj.\ $\text{Var}[Z] = \int (z-\mu)^2 p(z)\,dz = \sigma^2$.
Gdybyśmy napisali $\exp(-(z-\mu)^2/\sigma^2)$ bez dwójki,
otrzymalibyśmy rozkład o wariancji $\sigma^2/2$, nie~$\sigma^2$
-- parametr $\sigma^2$ nie odpowiadałby rzeczywistemu rozrzutowi danych. \\
$\exp\!\left(-\dfrac{(z-\mu)^2}{2\sigma^2}\right)$ &
Zamienia odległość na \textbf{prawdopodobieństwo}. Minus w wykładniku = im dalej od $\mu$, tym \textit{mniejsze} $p(z)$. Funkcja $e^{-x}$ maleje szybko! \\
$\dfrac{1}{\sqrt{2\pi\sigma^2}}$ &
\textbf{Stała normalizacyjna} — gwarantuje, że $\int_{-\infty}^{\infty} p(z)\,dz = 1$.
Bez niej to nie byłby rozkład prawdopodobieństwa. \\
\bottomrule
\end{tabular}
\end{center}

\begin{remark}
Przypadek standardowy ($\mu = 0$, $\sigma^2 = 1$):
\begin{equation}
p(z) = \frac{1}{\sqrt{2\pi}}\;e^{-z^2/2}
\end{equation}
To jest ten rozkład, którego chcemy dla embeddingów w LeJEPA!
\end{remark}

\begin{figure}[H]
\centering
\includegraphics[width=\textwidth]{figures/gauss_decomposition.pdf}
\caption{Dekompozycja wzoru 1D Gaussa.
Zielona linia: kwadrat odległości $z^2/2$ rośnie od środka.
Czerwona: eksponenta $e^{-z^2/2}$ zamienia odległość na malejącą wagę.
Niebieska: wynik po przeskalowaniu stałą $1/\sqrt{2\pi}$.}
\label{fig:decomposition}
\end{figure}

\begin{figure}[H]
\centering
\includegraphics[width=\textwidth]{figures/gauss_1d.pdf}
\caption{\textbf{Lewo}: wpływ wariancji $\sigma^2$ — im większa, tym szerszy i niższy rozkład
(ale pole pod krzywą zawsze $= 1$).
\textbf{Prawo}: wpływ średniej $\mu$ — przesuwa ``dzwonek'' w prawo lub lewo.}
\label{fig:gauss1d}
\end{figure}

% --- 2D ---
\subsection{Wyprowadzenie: Gauss w 2D}

Mamy wektor $\mathbf{z} = \begin{pmatrix} z_1 \\ z_2 \end{pmatrix} \in \mathbb{R}^2$.
Chcemy znaleźć $p(z_1, z_2)$.

\subsubsection{Krok 1: Zakładamy niezależność}

Jeśli $z_1$ i $z_2$ są \textbf{niezależne}, to ich łączne prawdopodobieństwo jest iloczynem:

\begin{align}
p(z_1, z_2) &= p(z_1) \cdot p(z_2) \nonumber\\
&= \frac{1}{\sqrt{2\pi\sigma_1^2}}\exp\!\left(-\frac{(z_1-\mu_1)^2}{2\sigma_1^2}\right)
\cdot
\frac{1}{\sqrt{2\pi\sigma_2^2}}\exp\!\left(-\frac{(z_2-\mu_2)^2}{2\sigma_2^2}\right)
\label{eq:2d_step1}
\end{align}

\subsubsection{Krok 2: Łączymy stałe normalizacyjne}

\begin{equation}
\frac{1}{\sqrt{2\pi\sigma_1^2}} \cdot \frac{1}{\sqrt{2\pi\sigma_2^2}}
= \frac{1}{2\pi\,\sigma_1\sigma_2}
\label{eq:2d_step2}
\end{equation}

Bo $\sqrt{a} \cdot \sqrt{b} = \sqrt{ab}$, więc
$\sqrt{2\pi\sigma_1^2} \cdot \sqrt{2\pi\sigma_2^2} = \sqrt{(2\pi)^2 \sigma_1^2\sigma_2^2} = 2\pi\,\sigma_1\sigma_2$.

\subsubsection{Krok 3: Łączymy wykładniki}

Właściwość eksponenty: $e^a \cdot e^b = e^{a+b}$, więc:

\begin{equation}
\exp\!\left(-\frac{(z_1-\mu_1)^2}{2\sigma_1^2}\right)
\cdot
\exp\!\left(-\frac{(z_2-\mu_2)^2}{2\sigma_2^2}\right)
= \exp\!\left(-\frac{(z_1-\mu_1)^2}{2\sigma_1^2} - \frac{(z_2-\mu_2)^2}{2\sigma_2^2}\right)
\label{eq:2d_step3}
\end{equation}

\subsubsection{Krok 4: Zapisujemy w notacji macierzowej}

Definiujemy:
\begin{equation}
\boldsymbol{\mu} = \begin{pmatrix} \mu_1 \\ \mu_2 \end{pmatrix}, \qquad
\boldsymbol{\Sigma} = \begin{pmatrix} \sigma_1^2 & 0 \\ 0 & \sigma_2^2 \end{pmatrix}
\quad \text{(macierz kowariancji, diagonalna bo niezależne)}
\end{equation}

Wtedy:
\begin{equation}
\boldsymbol{\Sigma}^{-1} = \begin{pmatrix} 1/\sigma_1^2 & 0 \\ 0 & 1/\sigma_2^2 \end{pmatrix},
\qquad
\det(\boldsymbol{\Sigma}) = \sigma_1^2 \cdot \sigma_2^2
\end{equation}

Wykładnik możemy zapisać jako \textbf{formę kwadratową}:
\begin{align}
\frac{(z_1-\mu_1)^2}{\sigma_1^2} + \frac{(z_2-\mu_2)^2}{\sigma_2^2}
&= \begin{pmatrix} z_1-\mu_1 & z_2-\mu_2 \end{pmatrix}
\begin{pmatrix} 1/\sigma_1^2 & 0 \\ 0 & 1/\sigma_2^2 \end{pmatrix}
\begin{pmatrix} z_1-\mu_1 \\ z_2-\mu_2 \end{pmatrix} \nonumber\\
&= (\mathbf{z} - \boldsymbol{\mu})^\top \boldsymbol{\Sigma}^{-1} (\mathbf{z} - \boldsymbol{\mu})
\label{eq:quadratic}
\end{align}

\subsubsection{Krok 5: Wynik — Gauss 2D (niezależne)}

Składamy wszystko:

\begin{equation}
\boxed{
p(\mathbf{z}) = p(z_1, z_2) =
\frac{1}{2\pi\,\sigma_1\sigma_2}
\;\exp\!\left(
-\frac{1}{2}(\mathbf{z}-\boldsymbol{\mu})^\top \boldsymbol{\Sigma}^{-1} (\mathbf{z}-\boldsymbol{\mu})
\right)
}
\tag{2D-diag}
\label{eq:2d_diag}
\end{equation}

\subsubsection{Krok 6: Uogólnienie — z korelacją}

Co jeśli $z_1$ i $z_2$ \textit{nie} są niezależne? Wtedy macierz kowariancji ma elementy pozadiagonalne:

\begin{equation}
\boldsymbol{\Sigma} = \begin{pmatrix}
\sigma_1^2 & \rho\,\sigma_1\sigma_2 \\
\rho\,\sigma_1\sigma_2 & \sigma_2^2
\end{pmatrix}
\end{equation}

gdzie $\rho \in [-1,1]$ to \textbf{współczynnik korelacji} Pearsona.

Wzór ma \textit{identyczną strukturę}, zmienia się tylko $\boldsymbol{\Sigma}$:

\begin{equation}
\boxed{
p(\mathbf{z}) =
\frac{1}{2\pi\sqrt{\det(\boldsymbol{\Sigma})}}
\;\exp\!\left(
-\frac{1}{2}(\mathbf{z}-\boldsymbol{\mu})^\top \boldsymbol{\Sigma}^{-1} (\mathbf{z}-\boldsymbol{\mu})
\right)
}
\tag{2D}
\label{eq:2d_full}
\end{equation}

\begin{keyinsight}[Sprawdzenie: stała normalizacyjna]
Dla 2D:
$\dfrac{1}{\sqrt{(2\pi)^2 \det(\boldsymbol{\Sigma})}} = \dfrac{1}{2\pi\sqrt{\det(\boldsymbol{\Sigma})}}$.

\medskip
Gdy $\rho = 0$ (niezależne): $\det(\boldsymbol{\Sigma}) = \sigma_1^2\sigma_2^2$,
więc $\sqrt{\det(\boldsymbol{\Sigma})} = \sigma_1\sigma_2$ — zgadza się z \eqref{eq:2d_diag}!
\end{keyinsight}

\subsubsection{Rozpiszmy wykładnik z korelacją (do ćwiczenia)}

\begin{align}
\boldsymbol{\Sigma}^{-1} &= \frac{1}{\sigma_1^2\sigma_2^2(1-\rho^2)}
\begin{pmatrix}
\sigma_2^2 & -\rho\,\sigma_1\sigma_2 \\
-\rho\,\sigma_1\sigma_2 & \sigma_1^2
\end{pmatrix}
\label{eq:sigma_inv_2d}
\end{align}

Więc wykładnik:
\begin{align}
(\mathbf{z}-\boldsymbol{\mu})^\top \boldsymbol{\Sigma}^{-1} (\mathbf{z}-\boldsymbol{\mu})
&= \frac{1}{1-\rho^2}\left[
\frac{(z_1-\mu_1)^2}{\sigma_1^2}
- \frac{2\rho(z_1-\mu_1)(z_2-\mu_2)}{\sigma_1\sigma_2}
+ \frac{(z_2-\mu_2)^2}{\sigma_2^2}
\right]
\label{eq:exponent_2d}
\end{align}

\begin{remark}
Gdy $\rho = 0$: czynnik $\frac{1}{1-\rho^2} = 1$ i wyraz mieszany znika — wracamy do przypadku niezależnego.
\end{remark}

\begin{figure}[H]
\centering
\includegraphics[width=\textwidth]{figures/gauss_2d_types.pdf}
\caption{Trzy typy rozkładu Gaussa 2D.
\textbf{Lewo}: izotropowy ($\boldsymbol{\Sigma} = \mathbf{I}$) — izolinie to okręgi.
\textbf{Środek}: diagonalny anizotropowy ($\sigma_1^2 \neq \sigma_2^2$) — elipsy wzdłuż osi.
\textbf{Prawo}: z korelacją ($\rho = 0.8$) — obrócone elipsy.
Czerwone strzałki: wektory własne $\boldsymbol{\Sigma}$ (kierunki główne).}
\label{fig:gauss2d_types}
\end{figure}

\begin{figure}[H]
\centering
\includegraphics[width=\textwidth]{figures/gauss_isolines.pdf}
\caption{Izolinie gęstości 2D — kształt mówi wszystko o macierzy kowariancji.
\textbf{Okręgi} = izotropowy. \textbf{Elipsy wzdłuż osi} = diagonalny.
\textbf{Obrócone elipsy} = korelacja.
Wartości własne $\lambda_1, \lambda_2$ określają długości półosi.}
\label{fig:isolines}
\end{figure}

\bigskip

% --- 3D ---
\subsection{Wyprowadzenie: Gauss w 3D}

Teraz $\mathbf{z} = \begin{pmatrix} z_1 \\ z_2 \\ z_3 \end{pmatrix} \in \mathbb{R}^3$.
Procedura jest identyczna!

\subsubsection{Krok 1: Definiujemy parametry}

\begin{equation}
\boldsymbol{\mu} = \begin{pmatrix} \mu_1 \\ \mu_2 \\ \mu_3 \end{pmatrix}, \qquad
\boldsymbol{\Sigma} = \begin{pmatrix}
\sigma_1^2 & \sigma_{12} & \sigma_{13} \\
\sigma_{12} & \sigma_2^2 & \sigma_{23} \\
\sigma_{13} & \sigma_{23} & \sigma_3^2
\end{pmatrix}
\end{equation}

gdzie $\sigma_{ij} = \mathrm{Cov}(z_i, z_j)$ to kowariancje (korelacje przeskalowane).

\subsubsection{Krok 2: Wzór ogólny}

\begin{equation}
\boxed{
p(\mathbf{z}) =
\frac{1}{(2\pi)^{3/2}\sqrt{\det(\boldsymbol{\Sigma})}}
\;\exp\!\left(
-\frac{1}{2}(\mathbf{z}-\boldsymbol{\mu})^\top \boldsymbol{\Sigma}^{-1} (\mathbf{z}-\boldsymbol{\mu})
\right)
}
\tag{3D}
\label{eq:3d_full}
\end{equation}

\subsubsection{Krok 3: Przypadek niezależny ($\boldsymbol{\Sigma}$ diagonalna)}

\begin{equation}
\boldsymbol{\Sigma} = \begin{pmatrix}
\sigma_1^2 & 0 & 0 \\
0 & \sigma_2^2 & 0 \\
0 & 0 & \sigma_3^2
\end{pmatrix}
\;\Rightarrow\;
\boldsymbol{\Sigma}^{-1} = \begin{pmatrix}
1/\sigma_1^2 & 0 & 0 \\
0 & 1/\sigma_2^2 & 0 \\
0 & 0 & 1/\sigma_3^2
\end{pmatrix}
\end{equation}

Wtedy $\det(\boldsymbol{\Sigma}) = \sigma_1^2\,\sigma_2^2\,\sigma_3^2$ i:

\begin{align}
p(z_1,z_2,z_3) &=
\frac{1}{(2\pi)^{3/2}\,\sigma_1\sigma_2\sigma_3}
\;\exp\!\left(
-\frac{(z_1-\mu_1)^2}{2\sigma_1^2}
-\frac{(z_2-\mu_2)^2}{2\sigma_2^2}
-\frac{(z_3-\mu_3)^2}{2\sigma_3^2}
\right) \nonumber\\
&= \underbrace{\frac{e^{-(z_1-\mu_1)^2/(2\sigma_1^2)}}{\sqrt{2\pi}\,\sigma_1}}_{p(z_1)}
\cdot
\underbrace{\frac{e^{-(z_2-\mu_2)^2/(2\sigma_2^2)}}{\sqrt{2\pi}\,\sigma_2}}_{p(z_2)}
\cdot
\underbrace{\frac{e^{-(z_3-\mu_3)^2/(2\sigma_3^2)}}{\sqrt{2\pi}\,\sigma_3}}_{p(z_3)}
\label{eq:3d_indep}
\end{align}

\begin{keyinsight}[Wzorzec: iloczyn niezależnych = suma w wykładniku]
Niezależność w 3D oznacza, że łączna gęstość \textbf{faktoryzuje się} na iloczyn trzech 1D Gaussów.
W wykładniku: iloczyn $e^a \cdot e^b \cdot e^c = e^{a+b+c}$.
\end{keyinsight}

\subsubsection{Krok 4: Przypadek izotropowy (LeJEPA!)}

Izotropowy = niezależne \textit{i} jednakowa wariancja: $\sigma_1^2 = \sigma_2^2 = \sigma_3^2 = 1$, $\boldsymbol{\mu} = \mathbf{0}$:

\begin{equation}
\boxed{
p(\mathbf{z}) = \frac{1}{(2\pi)^{3/2}}
\;\exp\!\left(-\frac{z_1^2 + z_2^2 + z_3^2}{2}\right)
= \frac{1}{(2\pi)^{3/2}}
\;\exp\!\left(-\frac{\|\mathbf{z}\|^2}{2}\right)
}
\tag{3D-iso}
\label{eq:3d_iso}
\end{equation}

Izolinie gęstości ($p(\mathbf{z}) = \mathrm{const}$) to \textbf{sfery}:
$\|\mathbf{z}\|^2 = z_1^2 + z_2^2 + z_3^2 = r^2$.

\begin{figure}[H]
\centering
\includegraphics[width=\textwidth]{figures/gauss_3d_surfaces.pdf}
\caption{Powierzchnie gęstości 3D (wyświetlamy $p(z_1,z_2)$ dla 2 zmiennych).
\textbf{Lewo}: izotropowy — symetryczny ``dzwonek''.
\textbf{Prawo}: anizotropowy ($\sigma_1^2=3, \sigma_2^2=0.3$) — wydłużony grzbiet.
W 3D pełnym izolinie to sfery (izotropowy) vs.\ elipsoidy (anizotropowy).}
\label{fig:gauss3d}
\end{figure}

% --- General K-D ---
\subsection{Uogólnienie: Gauss w $K$ wymiarach}

Wzorzec jest jasny — w $K$ wymiarach ($K = 384$ dla ViT-Small):

\begin{equation}
\boxed{
p(\mathbf{z}) =
\frac{1}{(2\pi)^{K/2}\sqrt{\det(\boldsymbol{\Sigma})}}
\;\exp\!\left(
-\frac{1}{2}(\mathbf{z}-\boldsymbol{\mu})^\top \boldsymbol{\Sigma}^{-1} (\mathbf{z}-\boldsymbol{\mu})
\right)
}
\tag{$K$D}
\label{eq:kd}
\end{equation}

Dla izotropowego ($\boldsymbol{\mu} = \mathbf{0}$, $\boldsymbol{\Sigma} = \mathbf{I}_K$):

\begin{equation}
\boxed{
p(\mathbf{z}) = \frac{1}{(2\pi)^{K/2}}
\;\exp\!\left(-\frac{\|\mathbf{z}\|^2}{2}\right)
= \frac{1}{(2\pi)^{K/2}}
\;\exp\!\left(-\frac{1}{2}\sum_{k=1}^{K} z_k^2\right)
}
\tag{$K$D-iso}
\label{eq:kd_iso}
\end{equation}

\begin{center}
\renewcommand{\arraystretch}{1.5}
\begin{tabular}{cccc}
\toprule
\textbf{Wymiar} & \textbf{Stała normalizacyjna} & \textbf{Wykładnik (iso)} & \textbf{Izolinie} \\
\midrule
1D & $\dfrac{1}{\sqrt{2\pi}}$ & $-\dfrac{z^2}{2}$ & punkty $z = \pm r$ \\[0.7em]
2D & $\dfrac{1}{2\pi}$ & $-\dfrac{z_1^2+z_2^2}{2}$ & okręgi \\[0.7em]
3D & $\dfrac{1}{(2\pi)^{3/2}}$ & $-\dfrac{z_1^2+z_2^2+z_3^2}{2}$ & sfery \\[0.7em]
$K$D & $\dfrac{1}{(2\pi)^{K/2}}$ & $-\dfrac{\sum_k z_k^2}{2}$ & hipersfery \\
\bottomrule
\end{tabular}
\end{center}


\clearpage
%% ============================================================
\section{Kontekst: Co robi encoder w JEPA?}
%% ============================================================

Encoder $f_\theta: \mathbb{R}^D \to \mathbb{R}^K$ mapuje dane wejściowe (np.\ ramki wideo)
do embeddingów w przestrzeni $K$-wymiarowej. Te embeddingi muszą spełniać dwa warunki:

\begin{enumerate}[label=(\roman*)]
  \item \textbf{Predykcja}: embedding widoku $v$ powinien być przewidywalny z embeddingu widoku $v'$,
  \item \textbf{Brak degeneracji}: embeddingi nie mogą się ``zkolapsować'' do jednego punktu.
\end{enumerate}

\begin{keyinsight}[Fundamentalne pytanie]
Jaki \textbf{rozkład} $P_\theta$ powinny mieć embeddingi $\mathbf{z} = f_\theta(\mathbf{x})$,
aby encoder był najlepszy na \textit{dowolnym} przyszłym zadaniu downstream?
\end{keyinsight}

Odpowiedź LeJEPA: \textbf{izotropowy Gauss} $\mathcal{N}(\mathbf{0}, \mathbf{I}_K)$.


\clearpage
%% ============================================================
\section{Co to jest izotropowy rozkład Gaussa?}
%% ============================================================

\begin{definition}[Izotropowy rozkład Gaussa]
Wektor losowy $\mathbf{z} \in \mathbb{R}^K$ ma \textbf{izotropowy rozkład Gaussa},
jeśli:
\begin{equation}
\mathbf{z} \sim \mathcal{N}(\mathbf{0}, \mathbf{I}_K),
\quad \text{tzn.} \quad
p(\mathbf{z}) = \frac{1}{(2\pi)^{K/2}} \exp\!\left(-\frac{1}{2}\|\mathbf{z}\|^2\right)
\label{eq:isotropic}
\end{equation}
gdzie $\mathbf{I}_K$ jest macierzą jednostkową $K \times K$.
\end{definition}

\textbf{Izotropia} oznacza, że rozkład wygląda identycznie we \textit{wszystkich} kierunkach:
\begin{itemize}
  \item Macierz kowariancji: $\mathrm{Cov}(\mathbf{z}) = \mathbf{I}_K$
        (wszystkie wartości własne $\lambda_1 = \lambda_2 = \cdots = \lambda_K = 1$),
  \item Izolinie gęstości to \textbf{hipersferery} $\|\mathbf{z}\| = r$,
  \item Żaden wymiar nie jest ``ważniejszy'' od innego.
\end{itemize}

\begin{figure}[H]
\centering
\includegraphics[width=\textwidth]{figures/isotropic_gaussian_3d.pdf}
\caption{Gęstość izotropowego Gaussa $\mathcal{N}(\mathbf{0}, \mathbf{I}_2)$.
\textbf{Lewo}: powierzchnia 3D. \textbf{Prawo}: izolinie tworzą koncentryczne okręgi —
cecha izotropii.}
\label{fig:3d}
\end{figure}

\subsection{Izotropowy vs.\ anizotropowy vs.\ kolaps}

\begin{figure}[H]
\centering
\includegraphics[width=\textwidth]{figures/isotropic_vs_anisotropic_2d.pdf}
\caption{Trzy scenariusze embeddingów 2D. \textbf{Lewo}: izotropowy Gauss —
równomierne rozłożenie we wszystkich kierunkach.
\textbf{Środek}: anizotropowy — informacja skoncentrowana wzdłuż jednej osi.
\textbf{Prawo}: kolaps — brak jakiejkolwiek informacji.
Czerwone strzałki to wektory własne macierzy kowariancji.}
\label{fig:2d}
\end{figure}


\clearpage
%% ============================================================
\section{Dlaczego izotropowy Gauss jest optymalny?}
\label{sec:why}
%% ============================================================

\subsection{Przypadek 1: Linear probe (regresja liniowa)}

Standardowa ewaluacja foundation modeli: zamrażamy encoder, trenujemy liniowy klasyfikator
na embeddingach.

\begin{definition}[Linear probe — OLS z regularyzacją Tikhonova]
\begin{equation}
\hat{\beta} = \arg\min_{\beta \in \mathbb{R}^K}
\|\mathbf{y} - \mathbf{Z}\beta\|_2^2 + \lambda\|\beta\|_2^2
\label{eq:ols}
\end{equation}
gdzie $\mathbf{Z} \in \mathbb{R}^{N \times K}$ to macierz embeddingów,
$\mathbf{y} \in \mathbb{R}^N$ to etykiety, $\lambda \geq 0$ to siła regularyzacji.
\end{definition}

Rozważmy dwa rozkłady embeddingów o \textit{tej samej energii} (suma wariancji):
\begin{align}
\mathbf{Z}_{\text{iso}} &: \quad \mathrm{Cov}(\mathbf{z}) = \mathbf{I}_K
\quad (\text{wartości własne: } 1, 1, \ldots, 1) \\
\mathbf{Z}_{\text{aniso}} &: \quad \mathrm{Cov}(\mathbf{z}) = \mathrm{diag}(\lambda_1, \ldots, \lambda_K)
\quad (\text{co najmniej dwie różne } \lambda_k)
\end{align}
Oba rozkłady mają tę samą sumaryczną wariancję: $\sum_k \lambda_k = K$.

\begin{lemma}[Anizotropia wzmacnia bias — Lemat 1 w artykule]
\label{lem:bias}
Gdy $\lambda_K > \lambda_1$, \textbf{zawsze istnieje} zadanie downstream (etykiety $\mathbf{y}$),
dla którego $\mathbf{Z}_{\text{aniso}}$ daje \textbf{wyższy bias} estymatora $\hat{\beta}$
niż $\mathbf{Z}_{\text{iso}}$, przy $\lambda > 0$.
\end{lemma}

\textit{Intuicja}: Regularyzacja Tikhonova ``obcina'' komponenty proporcjonalnie do $\frac{\lambda_k}{\lambda_k + \lambda}$.
Gdy $\lambda_k$ są nierówne, małe $\lambda_k$ są obcinane agresywniej $\Rightarrow$ informacja w tych kierunkach jest tracona.

\begin{lemma}[Anizotropia wzmacnia wariancję — Lemat 2 w artykule]
\label{lem:var}
Wariancja sumaryczna estymatora jest większa dla rozkładu anizotropowego:
\begin{equation}
\mathrm{tr}\!\left(\mathrm{Var}(\hat{\beta}_{\text{aniso}})\right)
> \mathrm{tr}\!\left(\mathrm{Var}(\hat{\beta}_{\text{iso}})\right)
\end{equation}
przy $\lambda = 0$ (OLS bez regularyzacji).
\end{lemma}

\textit{Intuicja}: Kierunki z małą wariancją ($\lambda_k \ll 1$) mają mało próbek ``pokrywających'' ten wymiar,
więc estymacja $\hat{\beta}_k$ jest niestabilna.

\begin{figure}[H]
\centering
\includegraphics[width=\textwidth]{figures/bias_variance_illustration.pdf}
\caption{Ilustracja Lematu 1 i 2.
Czarna linia: prawdziwa granica decyzyjna.
Zielone linie: granice nauczone z losowych podzbiorów danych.
\textbf{Lewo}: izotropowe embeddingi $\Rightarrow$ nauczone granice blisko prawdziwej (niski bias)
i mało rozproszone (niska wariancja).
\textbf{Prawo}: anizotropowe $\Rightarrow$ granice przesunięte (wysoki bias)
i mocno rozproszone (wysoka wariancja).}
\label{fig:bias_var}
\end{figure}

\subsection{Przypadek 2: Nieliniowe metody (k-NN, kernel)}

Dla bardziej elastycznej ewaluacji (nie tylko linear probe), artykuł analizuje:

\begin{definition}[k-NN predykcja]
\begin{equation}
\hat{y}(\mathbf{q}) = \frac{1}{|\mathcal{N}_{r_0}(\mathbf{q})|}
\sum_{n \in \mathcal{N}_{r_0}(\mathbf{q})} y_n,
\quad \text{gdzie } \mathcal{N}_{r_0}(\mathbf{q}) = \{n : \|\mathbf{z}_n - \mathbf{q}\| \leq r_0\}
\end{equation}
\end{definition}

\begin{definition}[Kernel predykcja]
\begin{equation}
\hat{y}(\mathbf{q}) = \frac{\sum_{n=1}^N K_h(\mathbf{q} - \mathbf{z}_n) y_n}
{\sum_{n=1}^N K_h(\mathbf{q} - \mathbf{z}_n)}
\end{equation}
\end{definition}

\begin{theorem}[Optymalność izotropowego Gaussa — Tw.\ 1 w artykule]
\label{thm:optimal}
Zintegrowany błąd kwadratowy (Integrated Square Bias) wynosi:
\begin{align}
\mathrm{ISB}_{k\text{-NN}} &= \frac{r_0^4}{(K+2)^2}\,\tau_2^2\,J(p) + O(r_0^4) \label{eq:isb_knn} \\
\mathrm{ISB}_{\text{kernel}} &\leq \left(\frac{h^2 \mu_2(K)}{2}\right)^2
\left(2B^2 + 8L^2 J(p)\right) + o(h^4) \label{eq:isb_kernel}
\end{align}
gdzie $J(p) = \int \|\nabla^2 \log p(\mathbf{z})\|_F^2 \, p(\mathbf{z}) \, d\mathbf{z}$
jest \textbf{informacją Fishera drugiego rzędu}.

Wśród rozkładów o kowariancji typu $\kappa \mathbf{I}_K$ (izotropowych),
\textbf{izotropowy Gauss jest jedynym minimalizatorem} $J(p)$,
a więc jedynym minimalizatorem ISB.
\end{theorem}

\begin{keyinsight}[Podsumowanie: dlaczego Gauss?]
\begin{enumerate}
  \item \textbf{Izotropia} ($\mathrm{Cov} = \mathbf{I}$): minimalizuje bias i wariancję linear probe,
  \item \textbf{Gaussowość}: minimalizuje informację Fishera $J(p)$,
        co minimalizuje ISB dla k-NN i kernel,
  \item \textbf{Razem}: $\mathcal{N}(\mathbf{0}, \mathbf{I}_K)$ jest jedynym rozkładem
        optymalnym dla \textit{każdego} możliwego zadania downstream.
\end{enumerate}
\end{keyinsight}


\clearpage
%% ============================================================
\section{Jak to wymusić? SIGReg}
\label{sec:sigreg}
%% ============================================================

Skoro wiemy, że embeddingi powinny mieć rozkład $\mathcal{N}(\mathbf{0}, \mathbf{I}_K)$,
potrzebujemy mechanizmu, który to wymusi podczas treningu.

\subsection{Idea: test statystyczny jako loss}

Zamiast heurystyk (whitening, stop-gradient, teacher-student), LeJEPA używa
\textbf{testu hipotez}:
\begin{equation}
H_0: P_\theta = \mathcal{N}(\mathbf{0}, \mathbf{I}_K)
\quad \text{vs.} \quad
H_1: P_\theta \neq \mathcal{N}(\mathbf{0}, \mathbf{I}_K)
\end{equation}

Problem: testowanie w $K$ wymiarach ($K = 128, 384, \ldots$) jest obliczeniowo trudne.

\subsection{Sketching: redukcja do testów 1D}

Kluczowy trik: zamiast testować w $\mathbb{R}^K$, rzutujemy na losowe kierunki $\mathbf{a} \in \mathbb{S}^{K-1}$:

\begin{equation}
\mathbf{a}^\top \mathbf{z} \sim \mathcal{N}(0, 1) \quad \forall\, \mathbf{a}
\quad \iff \quad
\mathbf{z} \sim \mathcal{N}(\mathbf{0}, \mathbf{I}_K)
\end{equation}

To wynika z \textbf{Lematu Craméra-Wolda}: rozkład wielowymiarowy jest jednoznacznie
określony przez wszystkie swoje rzuty jednowymiarowe.

\paragraph{Dlaczego $\mathcal{N}(0,1)$ na rzutach gwarantuje izotropowość?}

Powyższy wzór ($\iff$) to potężne twierdzenie. Rozbijmy je na dwie strony:

\begin{tcolorbox}[
  colback=lejepaBlue!8,
  colframe=lejepaBlue!80,
  fonttitle=\bfseries,
  title={Strona $\Rightarrow$: jeśli $\mathbf{z} \sim \mathcal{N}(\mathbf{0}, \mathbf{I}_K)$, to rzuty są $\mathcal{N}(0,1)$},
  breakable,
]
Jeśli embeddingi mają izotropowy rozkład Gaussa, to rzut na \textit{dowolny} kierunek
$\mathbf{a}$ daje $\mathcal{N}(0,1)$. Dlaczego?

Rzut $u = \mathbf{a}^\top \mathbf{z}$ to kombinacja liniowa składowych Gaussa,
więc sam jest gaussowski. Wystarczy obliczyć jego średnią i~wariancję:
\begin{align}
\mathbb{E}[u] &= \mathbb{E}[\mathbf{a}^\top \mathbf{z}]
= \mathbf{a}^\top \underbrace{\mathbb{E}[\mathbf{z}]}_{= \,\mathbf{0}} = 0 \nonumber \\[4pt]
\text{Var}[u] &= \text{Var}[\mathbf{a}^\top \mathbf{z}]
= \mathbf{a}^\top \underbrace{\text{Cov}(\mathbf{z})}_{= \,\mathbf{I}_K} \mathbf{a}
= \mathbf{a}^\top \mathbf{a}
= \underbrace{\|\mathbf{a}\|^2}_{= \,1} = 1 \nonumber
\end{align}
Średnia $= 0$, wariancja $= 1$, rozkład gaussowski $\Rightarrow$ $u \sim \mathcal{N}(0,1)$.

\textbf{Kluczowy moment:} wariancja wyszła $\mathbf{a}^\top \mathbf{I}_K \mathbf{a} = \|\mathbf{a}\|^2 = 1$
\textit{niezależnie od kierunku} $\mathbf{a}$.
To jest właśnie izotropowość --- macierz $\mathbf{I}_K$ traktuje wszystkie kierunki jednakowo.
\end{tcolorbox}

\begin{tcolorbox}[
  colback=lejepaRed!8,
  colframe=lejepaRed!80,
  fonttitle=\bfseries,
  title={Co by się stało, gdyby embeddingi NIE były izotropowe?},
  breakable,
]
Załóżmy, że embeddingi mają anizotropową macierz kowariancji, np.:
\[
\boldsymbol{\Sigma} = \begin{bmatrix} 4 & 0 \\ 0 & 0.25 \end{bmatrix}
\quad\text{(rozciągnięte wzdłuż $z_1$, ściśnięte wzdłuż $z_2$)}
\]

Teraz wariancja rzutu \textbf{zależy od kierunku}:
\begin{itemize}[leftmargin=2em]
  \item Kierunek $\mathbf{a} = (1, 0)$ (wzdłuż $z_1$):
        $\text{Var}[u] = \mathbf{a}^\top \boldsymbol{\Sigma}\, \mathbf{a}
        = (1,\; 0) \begin{bmatrix} 4 & 0 \\ 0 & 0.25 \end{bmatrix} \begin{bmatrix} 1 \\ 0 \end{bmatrix} = 4$
        \quad $\neq 1$ !
  \item Kierunek $\mathbf{a} = (0, 1)$ (wzdłuż $z_2$):
        $\text{Var}[u] = 0.25$
        \quad $\neq 1$ !
  \item Kierunek $\mathbf{a} = (1/\sqrt{2},\; 1/\sqrt{2})$ (po przekątnej):
        $\text{Var}[u] = 2.125$
        \quad $\neq 1$ !
\end{itemize}

Różne kierunki dają \textit{różne} wariancje $\Rightarrow$ rzuty \textit{nie} są $\mathcal{N}(0,1)$
$\Rightarrow$ test Eppsa-Pulleya to wykryje i~da duży loss.
\end{tcolorbox}

\begin{tcolorbox}[
  colback=lejepaGreen!8,
  colframe=lejepaGreen!80,
  fonttitle=\bfseries,
  title={Strona $\Leftarrow$ (Lemat Craméra-Wolda): jeśli WSZYSTKIE rzuty są $\mathcal{N}(0,1)$, to $\mathbf{z} \sim \mathcal{N}(\mathbf{0}, \mathbf{I}_K)$},
  breakable,
]
To jest potężniejsze twierdzenie (i~trudniejsze do udowodnienia).
Mówi:

\textit{Jeśli rzut na \textbf{każdy} możliwy kierunek $\mathbf{a}$ daje
rozkład $\mathcal{N}(0,1)$, to jedynym rozkładem $K$-wymiarowym, który to spełnia,
jest $\mathcal{N}(\mathbf{0}, \mathbf{I}_K)$.}

Innymi słowy: nie ma ``podstępu'' --- nie istnieje żaden dziwny, nie-gaussowski rozkład
w~$\mathbb{R}^K$, który przypadkiem miałby wszystkie rzuty 1D gaussowskie.
Gaussowskość rzutów \textbf{jednoznacznie wymusza} gaussowskość całego rozkładu.

\textbf{Analogia:} Jeśli cień obiektu 3D jest okrągły z~\textit{każdej} strony
--- to obiekt \textit{musi} być kulą. Nie ma innego kształtu, który daje
okrągły cień ze~wszystkich kierunków.
\end{tcolorbox}

\begin{keyinsight}[Podsumowanie: dlaczego testujemy rzuty?]
\begin{enumerate}[leftmargin=2em, topsep=2pt]
  \item Chcemy: $\mathbf{z} \sim \mathcal{N}(\mathbf{0}, \mathbf{I}_K)$ (izotropowy Gauss w~$K$ wymiarach).
  \item Lemat Craméra-Wolda mówi: wystarczy sprawdzić, że
        $\mathbf{a}^\top \mathbf{z} \sim \mathcal{N}(0,1)$ dla każdego kierunku $\mathbf{a}$.
  \item Jeśli jakikolwiek kierunek daje wariancję $\neq 1$ lub rozkład nie-gaussowski
        --- embeddingi \textit{nie} są izotropowe.
  \item SIGReg testuje $M = 1024$ losowych kierunków testem Eppsa-Pulleya
        i~minimalizuje odchylenia od $\mathcal{N}(0,1)$.
  \item Efekt: sieć jest ``pchana'' w~stronę $\mathcal{N}(\mathbf{0}, \mathbf{I}_K)$
        --- izotropowego Gaussa.
\end{enumerate}
\end{keyinsight}

%% ============================================================
\subsection{Co to jest rzut $u = \mathbf{a}^\top \mathbf{z}$?}
\label{sec:rzut}
%% ============================================================

W~powyższym wzorze pojawiły się trzy symbole: $u$, $\mathbf{a}$ i~$\mathbf{z}$.
Wyjaśnijmy \textit{dokładnie}, co każdy z~nich oznacza.

\subsubsection{$\mathbf{z}$ --- wektor embeddingu}

$\mathbf{z}$ to \textbf{wektor embeddingu} --- wynik działania sieci neuronowej (enkodera)
na jednym obrazie. Ma $K$ wymiarów (liczb):
\[
\mathbf{z} = \begin{bmatrix} z_1 \\ z_2 \\ \vdots \\ z_K \end{bmatrix}
\in \mathbb{R}^K
\]

W~LeJEPA typowo $K = 128$ (po projektorze MLP) lub $K = 384$ (surowy ViT CLS token).
Każdy obraz w~batchu daje swój wektor $\mathbf{z}_j$, więc z~batcha $N$ obrazów
dostajemy $N$ wektorów: $\mathbf{z}_1, \mathbf{z}_2, \ldots, \mathbf{z}_N$.

\textbf{Prosty przykład} ($K=3$, żebyśmy mogli to narysować):
\[
\mathbf{z} = \begin{bmatrix} 0.5 \\ -1.2 \\ 0.8 \end{bmatrix}
\quad \leftarrow \text{trzy liczby opisujące jeden obraz}
\]

\subsubsection{$\mathbf{a}$ --- losowy kierunek jednostkowy}

$\mathbf{a}$ to \textbf{losowy wektor kierunkowy} o~długości 1.
Też ma $K$ wymiarów (tyle samo co $\mathbf{z}$!):
\[
\mathbf{a} = \begin{bmatrix} a_1 \\ a_2 \\ \vdots \\ a_K \end{bmatrix}
\in \mathbb{R}^K, \qquad \|\mathbf{a}\| = \sqrt{a_1^2 + a_2^2 + \cdots + a_K^2} = 1
\]

Warunek $\|\mathbf{a}\| = 1$ oznacza, że $\mathbf{a}$ leży na \textbf{sferze jednostkowej}
$\mathbb{S}^{K-1}$. Co to takiego?

%% --- Co to jest sfera jednostkowa? ---
\paragraph{Co to jest sfera jednostkowa $\mathbb{S}^{K-1}$?}

\begin{tcolorbox}[
  colback=lejepaBlue!8,
  colframe=lejepaBlue!80,
  fonttitle=\bfseries,
  title={Definicja: sfera jednostkowa},
  breakable,
]
Sfera jednostkowa $\mathbb{S}^{K-1}$ to zbiór \textbf{wszystkich wektorów w~$\mathbb{R}^K$
o~długości dokładnie~1}:
\[
\mathbb{S}^{K-1} = \left\{\, \mathbf{a} \in \mathbb{R}^K \;:\; \|\mathbf{a}\| = 1 \,\right\}
\]

\textbf{Dlaczego $K\!-\!1$, a~nie $K$?}
Indeks górny oznacza \textit{wymiarowość samej sfery}, nie przestrzeni, w~której żyje:
\begin{itemize}[leftmargin=2em]
  \item $K=2$: $\mathbb{S}^1$ = \textbf{okrąg} --- krzywa 1-wymiarowa
        (żyje w~$\mathbb{R}^2$, ale sam okrąg to krzywa --- 1D).
  \item $K=3$: $\mathbb{S}^2$ = \textbf{sfera} --- powierzchnia 2-wymiarowa
        (żyje w~$\mathbb{R}^3$, ale sama powierzchnia kuli jest 2D).
  \item $K=128$: $\mathbb{S}^{127}$ = \textbf{hipersfera} 127-wymiarowa w~$\mathbb{R}^{128}$.
\end{itemize}

Ogólna reguła: warunek $\|\mathbf{a}\| = 1$ ``zabiera'' jeden stopień swobody,
więc sfera w~$\mathbb{R}^K$ ma wymiar $K - 1$.
\end{tcolorbox}

\begin{figure}[H]
\centering
\includegraphics[width=\textwidth]{figures/unit_sphere.pdf}
\caption{\textbf{Lewo}: Dla $K=2$ sfera $\mathbb{S}^1$ to okrąg.
Każda strzałka to jeden losowy kierunek $\mathbf{a}$ --- wszystkie kończą się na okręgu
(bo $\|\mathbf{a}\|=1$).
\textbf{Środek}: Dla $K=3$ sfera $\mathbb{S}^2$ to powierzchnia kuli.
Kierunki $\mathbf{a}$ to strzałki od środka do powierzchni.
\textbf{Prawo}: Jak losujemy kierunek $\mathbf{a}$ --- trzy kroki.}
\label{fig:unit_sphere}
\end{figure}

\begin{figure}[H]
\centering
\includegraphics[width=\textwidth]{figures/unit_sphere_intuition.pdf}
\caption{\textbf{Lewo}: Zielone strzałki leżą na sferze ($\|\mathbf{a}\|=1$),
czerwone krzyżyki to wektory o~złej długości --- nie są na sferze.
\textbf{Prawo}: Dlaczego $\mathbb{S}^{K-1}$, a~nie $\mathbb{S}^K$
--- wymiarowość sfery jest o~1 mniejsza od wymiarowości przestrzeni.}
\label{fig:unit_sphere_intuition}
\end{figure}

%% --- Jak losujemy a? ---
\paragraph{Jak losujemy kierunek $\mathbf{a}$?}

Procedura jest prosta (3 kroki):
\begin{enumerate}[leftmargin=2em]
  \item \textbf{Losujemy} $K$ niezależnych liczb z~rozkładu $\mathcal{N}(0,1)$:
        \[
        \underbrace{v_1}_{\text{1 liczba}} ,\;\;
        \underbrace{v_2}_{\text{1 liczba}} ,\;\;
        \ldots ,\;\;
        \underbrace{v_K}_{\text{1 liczba}}
        \qquad\text{--- każda $v_k$ to osobna liczba z $\mathcal{N}(0,1)$}
        \]
        \textbf{Skąd się biorą te liczby?} Generuje je \textit{komputer} ---
        generator liczb pseudolosowych (w~PyTorch: \texttt{torch.randn(K)},
        w~NumPy: \texttt{np.random.randn(K)}).
        Rozkład $\mathcal{N}(0,1)$ oznacza, że:
        \begin{itemize}[leftmargin=2em, topsep=2pt]
          \item większość wylosowanych wartości będzie blisko~$0$ (np.\ między $-2$ a~$2$),
          \item wartości mogą być ujemne lub dodatnie (symetria wokół zera),
          \item bardzo duże $|v_k| > 3$ zdarzają się rzadko (poniżej $0.3\%$ szansy).
        \end{itemize}
        Te liczby są \textbf{losowe} --- za każdym razem inne.
        Nie są uczone, nie zależą od danych, nie mają gradientu.

        Razem tworzymy z~nich \textbf{wektor} o~$K$ elementach:
        $\mathbf{v} = (v_1, v_2, \ldots, v_K) \in \mathbb{R}^K$.
  \item \textbf{Obliczamy normę} (długość) tego wektora:
        $\|\mathbf{v}\| = \sqrt{v_1^2 + v_2^2 + \cdots + v_K^2}$ --- to jest \textit{jedna liczba}.
  \item \textbf{Normalizujemy} --- dzielimy każdy element wektora przez normę:
        $\mathbf{a} = \mathbf{v} \,/\, \|\mathbf{v}\|$,
        tzn.\ $a_k = v_k / \|\mathbf{v}\|$ dla $k = 1, \ldots, K$.
\end{enumerate}

\paragraph{Jak to wygląda w~praktyce? Konkretny przykład.}

W~kodzie to dosłownie \textbf{3 linijki}. Pokażmy dla $K = 5$
(w~LeJEPA będzie $K = 128$, ale zasada jest identyczna):

\begin{tcolorbox}[
  colback=black!3,
  colframe=black!50,
  fonttitle=\bfseries\ttfamily,
  title={Python / PyTorch},
  breakable,
]
\begin{verbatim}
import torch

K = 5
v = torch.randn(K)       # Krok 1: losujemy K liczb z N(0,1)
a = v / torch.norm(v)     # Kroki 2+3: normalizujemy
\end{verbatim}
\end{tcolorbox}

\textbf{Co dokładnie robi \texttt{torch.randn(5)}?}
Prosi komputer: ``daj mi 5 losowych liczb z~rozkładu Gaussa $\mathcal{N}(0,1)$''.
Za każdym uruchomieniem dostaniemy \textit{inne} liczby.

\begin{tcolorbox}[
  colback=lejepaGreen!8,
  colframe=lejepaGreen!80,
  fonttitle=\bfseries,
  title={Jak komputer generuje liczby z~rozkładu Gaussa?},
  breakable,
]
Komputer \textbf{nie} generuje Gaussa bezpośrednio. Robi to w~dwóch etapach:

\textbf{Etap~1: Liczby równomierne} $U_1, U_2 \in (0, 1)$.\\
To jest łatwe --- procesor ma wbudowany generator, który daje ``losowe'' liczby
równomiernie rozłożone między 0 a~1 (jak idealna kostka, ale ciągła).
Przykład: $U_1 = 0.374$, $U_2 = 0.821$.

\textbf{Etap~2: Transformacja Box-Mullera} --- klasyczny algorytm z~1958~roku
(G.E.P.~Box, M.E.~Muller, \textit{``A~Note on the Generation of Random Normal Deviates''},
The Annals of Mathematical Statistics).
Zamienia dwie liczby równomierne na dwie liczby gaussowskie:
\[
Z_1 = \sqrt{-2 \ln U_1} \cdot \cos(2\pi U_2), \qquad
Z_2 = \sqrt{-2 \ln U_1} \cdot \sin(2\pi U_2)
\]
Wynik: $Z_1$ i~$Z_2$ mają \textit{dokładnie} rozkład $\mathcal{N}(0,1)$.

\textbf{Dlaczego to działa?} Logarytm i~pierwiastek ``rozciągają'' równomierny rozkład
tak, że wartości blisko~$0$ stają się częste (środek dzwonu Gaussa),
a~wartości daleko od~$0$ stają się rzadkie (ogony dzwonu).
Cosinus i~sinus zapewniają symetrię (wartości ujemne i~dodatnie równie prawdopodobne).

\textbf{Przykład liczbowy:}
\[
U_1 = 0.374,\;\; U_2 = 0.821 \quad\Rightarrow\quad
Z_1 = \sqrt{-2 \ln 0.374} \cdot \cos(2\pi \cdot 0.821) = 1.40 \cdot 0.37 = 0.52
\]
Powtarzamy $K/2$ razy --- i~mamy $K$ liczb z~$\mathcal{N}(0,1)$.

\medskip
\textit{Uwaga:} W~praktyce PyTorch używa szybszego wariantu (algorytm Ziggurat),
ale idea jest ta sama: równomierne $\to$ przekształcenie $\to$ gaussowskie.
Jako użytkownik nie musisz o~tym myśleć --- \texttt{torch.randn(K)} robi to za Ciebie.
\end{tcolorbox}

Na przykład:

\medskip

\textbf{Uruchomienie 1:}
\begin{align}
\text{Krok 1:}\quad \mathbf{v} &= (\underset{\uparrow}{0.49},\;\; \underset{\uparrow}{-0.13},\;\; \underset{\uparrow}{0.65},\;\; \underset{\uparrow}{0.86},\;\; \underset{\uparrow}{-1.23})
\quad\leftarrow \text{5 losowych liczb} \nonumber \\
\text{Krok 2:}\quad \|\mathbf{v}\| &= \sqrt{0.49^2 + 0.13^2 + 0.65^2 + 0.86^2 + 1.23^2}
= \sqrt{2.83} = 1.683 \nonumber \\
\text{Krok 3:}\quad \mathbf{a} &= \mathbf{v} / 1.683
= (0.291,\;\; {-0.077},\;\; 0.386,\;\; 0.511,\;\; {-0.731}) \nonumber
\end{align}
Sprawdzenie: $\|\mathbf{a}\| = \sqrt{0.291^2 + 0.077^2 + 0.386^2 + 0.511^2 + 0.731^2} = 1.0$ \;\checkmark

\medskip

\textbf{Uruchomienie 2} (inne losowe liczby!):
\begin{align}
\text{Krok 1:}\quad \mathbf{v} &= (-1.07,\;\; 0.54,\;\; -0.31,\;\; 0.02,\;\; 1.51)
\quad\leftarrow \text{zupełnie inne 5 liczb} \nonumber \\
\text{Krok 2:}\quad \|\mathbf{v}\| &= 1.951 \nonumber \\
\text{Krok 3:}\quad \mathbf{a} &= (-0.548,\;\; 0.277,\;\; {-0.159},\;\; 0.010,\;\; 0.774) \nonumber
\end{align}

\begin{keyinsight}[Co się tu dzieje --- podsumowanie]
\begin{enumerate}[leftmargin=2em, topsep=2pt]
  \item Komputer ``rzuca kostką'' $K$ razy --- za każdym razem dostaje jedną losową
        liczbę z~$\mathcal{N}(0,1)$ (zwykle między $-2$ a~$2$).
  \item Te $K$ liczb składamy w~wektor $\mathbf{v}$.
  \item Wektor $\mathbf{v}$ ma jakąś długość (normę) --- dzielimy przez nią,
        żeby dostać wektor $\mathbf{a}$ o~długości dokładnie~1.
  \item Wynik: losowy kierunek $\mathbf{a}$ na sferze $\mathbb{S}^{K-1}$.
  \item Za każdym uruchomieniem kierunek jest \textit{inny} ---
        bo liczby $v_k$ są inne.
  \item W~SIGReg robimy to $M = 1024$ razy, dostając 1024 różnych kierunków.
\end{enumerate}
\end{keyinsight}

\begin{warningbox}[Uwaga: jaki to rozkład? Czy Gauss zawsze jest izotropowy?]
W~kroku~1 losujemy $K$ \textbf{oddzielnych, niezależnych} liczb, każdą z~\textit{jednowymiarowego}
Gaussa $\mathcal{N}(0,1)$. Każda $v_k$ to \textit{jedna} liczba (skalar).
Ale złożone razem w~wektor $\mathbf{v} = (v_1, \ldots, v_K)$ dają wektor o~$K$ wymiarach.

Ale ponieważ są niezależne --- to jest \textbf{dokładnie to samo}, co wylosowanie jednego
wektora z~$K$-wymiarowego Gaussa $\mathcal{N}(\mathbf{0}, \mathbf{I}_K)$:
\[
\underbrace{v_1 \sim \mathcal{N}(0,1), \;\; v_2 \sim \mathcal{N}(0,1), \;\; \ldots, \;\; v_K \sim \mathcal{N}(0,1)}_{\text{$K$ niezależnych losowań 1D}}
\quad\iff\quad
\underbrace{\mathbf{v} \sim \mathcal{N}(\mathbf{0}, \mathbf{I}_K)}_{\text{jedno losowanie $K$-wymiarowe}}
\]

To działa, bo macierz kowariancji $\mathbf{I}_K$ (macierz jednostkowa) oznacza:
każdy wymiar ma wariancję~1, i~wymiary są \textit{niezależne} (korelacja~= 0).

\bigskip

\textbf{Czy rozkład Gaussa zawsze jest izotropowy? NIE!}

Rozkład Gaussa wielowymiarowy $\mathcal{N}(\boldsymbol{\mu}, \boldsymbol{\Sigma})$
może mieć \textit{dowolną} macierz kowariancji $\boldsymbol{\Sigma}$:
\begin{itemize}[leftmargin=2em]
  \item $\boldsymbol{\Sigma} = \mathbf{I}$ --- \textbf{izotropowy}: okręgi/sfery,
        \textit{żaden kierunek nie jest uprzywilejowany}.
  \item $\boldsymbol{\Sigma} = \text{diag}(\sigma_1^2, \ldots, \sigma_K^2)$ --- \textbf{anizotropowy}:
        elipsy/elipsoidy, różne wariancje w~różnych kierunkach.
  \item $\boldsymbol{\Sigma}$ z~elementami pozadiagonalnymi $\neq 0$ --- \textbf{skorelowany}:
        elipsy obrócone pod kątem.
\end{itemize}

\textbf{Dlaczego tu potrzebujemy izotropowego?}
Bo chcemy, żeby losowy kierunek $\mathbf{a}$ był \textbf{równomierny} na sferze ---
żaden kierunek nie może być bardziej prawdopodobny niż inny.
Gdybyśmy użyli anizotropowego Gaussa (np.\ $\sigma_1^2 = 10$, $\sigma_2^2 = 0.1$),
to po normalizacji kierunki byłyby ``skupione'' wokół osi~$z_1$
--- a~my byśmy testowali głównie ten jeden kierunek, pomijając inne.

Izotropowy Gauss $\mathcal{N}(\mathbf{0}, \mathbf{I}_K)$ gwarantuje, że po normalizacji
$\mathbf{a} = \mathbf{v}/\|\mathbf{v}\|$ dostajemy \textbf{rozkład jednostajny na sferze}
$\mathbb{S}^{K-1}$.
\end{warningbox}

\textbf{Przykład} ($K=3$):
\[
\mathbf{v} = \begin{bmatrix} 1 \\ 1 \\ 1 \end{bmatrix}
\quad\Rightarrow\quad
\|\mathbf{v}\| = \sqrt{1^2 + 1^2 + 1^2} = \sqrt{3}
\quad\Rightarrow\quad
\mathbf{a} = \frac{\mathbf{v}}{\|\mathbf{v}\|} = \begin{bmatrix} 1/\sqrt{3} \\ 1/\sqrt{3} \\ 1/\sqrt{3} \end{bmatrix}
\approx \begin{bmatrix} 0.577 \\ 0.577 \\ 0.577 \end{bmatrix}
\]
Sprawdzenie: $\|\mathbf{a}\| = \sqrt{3 \cdot (1/\sqrt{3})^2} = \sqrt{3 \cdot 1/3} = 1$ \;\checkmark

%% --- Rola a w SIGReg ---
\paragraph{Rola $\mathbf{a}$ w~SIGReg --- kiedy losujemy?}

\begin{tcolorbox}[
  colback=lejepaBlue!8,
  colframe=lejepaBlue!80,
  fonttitle=\bfseries,
  title={Kiedy losujemy kierunki $\mathbf{a}$? Raz czy wiele razy?},
  breakable,
]
W~oryginalnym LeJEPA kierunki losujemy \textbf{raz, przed rozpoczęciem treningu},
i~zostają \textbf{takie same} przez cały trening --- przez wszystkie epoki i~wszystkie batche:

\begin{center}
\renewcommand{\arraystretch}{1.5}
\begin{tabular}{lcc}
\toprule
\textbf{Pytanie} & \textbf{Odpowiedź} \\
\midrule
Czy $\mathbf{a}$ zmienia się między epokami? & \textbf{Nie} --- te same kierunki \\
Czy $\mathbf{a}$ zmienia się między batchami? & \textbf{Nie} --- te same kierunki \\
Ile kierunków losujemy? & $M \approx 1024$ \\
Czy sieć ``uczy'' kierunki $\mathbf{a}$? & \textbf{Nie} --- brak gradientu \\
\bottomrule
\end{tabular}
\end{center}

\textbf{Dlaczego to wystarczy?}
\begin{itemize}[leftmargin=2em, topsep=2pt]
  \item $M = 1024$ losowych kierunków w~$\mathbb{R}^{128}$ daje bardzo dobre pokrycie sfery
        $\mathbb{S}^{127}$ --- wystarczająco gęste, żeby wykryć anizotropię w~każdym kierunku.
  \item Lemat Craméra-Wolda teoretycznie wymaga \textit{wszystkich} kierunków
        ($\infty$), ale w~praktyce 1024 to wystarczające przybliżenie.
  \item Stałe kierunki mają zaletę: loss SIGReg jest \textbf{deterministyczny}
        (przy tym samym batchu daje tę samą wartość) --- to stabilizuje trening.
\end{itemize}

\textbf{Analogia:} Wyobraź sobie, że chcesz sprawdzić, czy piłka jest idealnie okrągła.
Mierzysz ją linijką w~1024 losowych kierunkach.
Nie musisz zmieniać kierunków przy każdym pomiarze ---
te same 1024 kierunków wystarczą, żeby wykryć każdą deformację.

\medskip

\textbf{Kod:}
\begin{verbatim}
# Przed treningiem (raz!):
A = torch.randn(M, K)                 # M=1024 wektorów, każdy K-wymiarowy
A = A / torch.norm(A, dim=1, keepdim=True)  # normalizacja

# W pętli treningowej (każdy batch):
for epoch in range(100):
    for batch in dataloader:
        z = encoder(batch)        # embeddingi [N, K]
        u = z @ A.T               # rzuty [N, M] - używamy TYCH SAMYCH A!
        loss = epps_pulley(u)      # test na każdym z M kierunków
\end{verbatim}
\end{tcolorbox}

\textbf{Geometrycznie:} $\mathbf{a}$ to po prostu \textit{kierunek w~przestrzeni} ---
strzałka wskazująca ``w~którą stronę patrzymy'' na nasze dane.

\subsubsection{$u = \mathbf{a}^\top \mathbf{z}$ --- iloczyn skalarny (rzut)}

Teraz najważniejsze: \textbf{iloczyn skalarny} $\mathbf{a}^\top \mathbf{z}$:
\begin{equation}
u = \mathbf{a}^\top \mathbf{z}
= a_1 \cdot z_1 + a_2 \cdot z_2 + \cdots + a_K \cdot z_K
= \sum_{k=1}^{K} a_k z_k
\label{eq:projection}
\end{equation}

\textbf{To jest po prostu suma iloczynów element-po-elemencie.}
Wynik to \textit{jedna liczba} $u \in \mathbb{R}$ --- nie wektor!
Z~wektora o~$K$ wymiarach zrobiliśmy jedną liczbę.

\textbf{Pełny przykład liczbowy} ($K=3$):
\begin{align}
u &= \mathbf{a}^\top \mathbf{z}
= a_1 \cdot z_1 + a_2 \cdot z_2 + a_3 \cdot z_3 \nonumber \\
&= 0.577 \cdot 0.5 + 0.577 \cdot (-1.2) + 0.577 \cdot 0.8 \nonumber \\
&= 0.289 + (-0.693) + 0.462 \nonumber \\
&= 0.058 \label{eq:projection_example}
\end{align}

Z~wektora 3D $(0.5, -1.2, 0.8)$ dostaliśmy jedną liczbę: $u = 0.058$.

\begin{figure}[H]
\centering
\includegraphics[width=\textwidth]{figures/projection_example.pdf}
\caption{Przykład liczbowy: z~wektora $\mathbf{z} \in \mathbb{R}^3$ i~kierunku
$\mathbf{a} \in \mathbb{R}^3$ obliczamy jedną liczbę $u = \mathbf{a}^\top \mathbf{z}$.}
\label{fig:projection_example}
\end{figure}

\subsubsection{Geometryczna interpretacja --- rzut na prostą}

Iloczyn skalarny $u = \mathbf{a}^\top \mathbf{z}$ ma prostą interpretację geometryczną:

\textbf{$u$ to długość cienia} --- gdybyś postawił wektor $\mathbf{z}$ w~przestrzeni
i~poświecił na niego latarką z~kierunku prostopadłego do $\mathbf{a}$,
to cień $\mathbf{z}$ na osi $\mathbf{a}$ miałby długość $u$.

Formalnie: $u$ to \textbf{rzut ortogonalny} $\mathbf{z}$ na kierunek $\mathbf{a}$.
Punkt $u \cdot \mathbf{a}$ to najbliższy punkt na \textit{prostej wyznaczonej przez $\mathbf{a}$}
do~punktu~$\mathbf{z}$.

\begin{tcolorbox}[
  colback=lejepaGreen!8,
  colframe=lejepaGreen!80,
  fonttitle=\bfseries,
  title={Notacja: $\{\,t\,\mathbf{a} : t \in \mathbb{R}\,\}$ --- co to znaczy?},
  breakable,
]
Zapis $\{\,t\,\mathbf{a} : t \in \mathbb{R}\,\}$ to \textbf{notacja opisowa zbioru}
(ang.\ \textit{set-builder notation}). Czytamy ją tak:

\[
\underbrace{\{}_{\text{zbiór}}
\;\underbrace{t\,\mathbf{a}}_{\text{elementów postaci }t \cdot \mathbf{a}}
\;\underbrace{:}_{\text{takich, że}}
\;\underbrace{t \in \mathbb{R}}_{\text{$t$ jest dowolną liczbą rzeczywistą}}
\underbrace{\}}_{\text{koniec opisu}}
\]

Czyli: ``zbiór wszystkich punktów $t \cdot \mathbf{a}$, gdzie $t$ przebiega
od~$-\infty$ do~$+\infty$''. Geometrycznie to jest \textbf{prosta}
przechodząca przez początek układu współrzędnych w~kierunku $\mathbf{a}$:

\medskip
\begin{center}
\renewcommand{\arraystretch}{1.4}
\begin{tabular}{rcl}
$t = 0$ & $\Rightarrow$ & punkt $\mathbf{0} = (0, 0, \ldots)$ --- początek układu \\
$t = 1$ & $\Rightarrow$ & punkt $\mathbf{a}$ --- sam wektor kierunkowy \\
$t = 2$ & $\Rightarrow$ & punkt $2\mathbf{a}$ --- dwa razy dalej w~tym samym kierunku \\
$t = -1$ & $\Rightarrow$ & punkt $-\mathbf{a}$ --- w~przeciwnym kierunku \\
$t = 0.5$ & $\Rightarrow$ & punkt $0.5\,\mathbf{a}$ --- w~połowie drogi do $\mathbf{a}$ \\
\end{tabular}
\end{center}

\medskip
Wszystkie te punkty leżą na jednej prostej ---
a~rzut $u$ mówi, \textit{w~którym miejscu} na tej prostej ``ląduje'' cień wektora $\mathbf{z}$.
\end{tcolorbox}

\begin{figure}[H]
\centering
\includegraphics[width=\textwidth]{figures/projection_explanation.pdf}
\caption{\textbf{Lewo}: Embeddingi $\mathbf{z}_j$ w~2D.
\textbf{Środek}: Rzut na kierunek $\mathbf{a}$ --- każdy punkt ``spada'' na czerwoną prostą
(linia przerywana = prostopadła).
\textbf{Prawo}: Wynikowe wartości $u_j$ na osi liczbowej + histogram (wyjaśnienie niżej).}
\label{fig:projection_explanation}
\end{figure}

\paragraph{Po co histogram na prawym panelu?}

Na środkowym panelu zrzutowaliśmy $N = 8$ embeddingów na kierunek $\mathbf{a}$
i~dostaliśmy 8 liczb: $u_1, u_2, \ldots, u_8$.
Na prawym panelu robimy z~nimi dwie rzeczy:

\begin{enumerate}[leftmargin=2em]
  \item \textbf{Oś liczbowa} (górna część) --- pokazuje, \textit{gdzie} dokładnie
        wylądowała każda wartość $u_j$. Pomarańczowe kwadraty to poszczególne
        rzuty, podpisane liczbami (np.\ $u_3 = 0.8$, $u_5 = -1.1$).

  \item \textbf{Histogram} (dolna część) --- grupuje wartości $u_j$ w~``przedziały'' (bins)
        i~pokazuje, \textit{ile wartości} wpadło w~każdy przedział.
        Im wyższy słupek, tym więcej embeddingów ma rzut w~tym zakresie.

        \textbf{Po co?} Bo chcemy zobaczyć \textbf{kształt rozkładu} tych $N$ liczb.
        Histogram to najprostszy sposób na wizualizację rozkładu:
        \begin{itemize}[leftmargin=2em, topsep=2pt]
          \item Jeśli histogram wygląda jak \textbf{dzwonek} (dużo wartości blisko~0,
                mało daleko od~0) --- embeddingi w~tym kierunku wyglądają jak Gauss.
          \item Jeśli histogram jest \textbf{płaski}, \textbf{przesunięty} lub ma
                \textbf{dwa szczyty} --- embeddingi \textit{nie} są gaussowskie.
        \end{itemize}

  \item \textbf{Czarna krzywa} $\mathcal{N}(0,1)$ --- to nasz \textit{cel}.
        Pokazuje, jak powinien wyglądać idealny rozkład.
        Im bardziej histogram pokrywa się z~czarną krzywą, tym lepiej.
\end{enumerate}

\begin{keyinsight}[Histogram $\to$ test EP --- po co nam coś więcej niż histogram?]
Histogram to dobra \textit{wizualizacja} dla człowieka, ale zły \textit{loss} dla sieci:
\begin{itemize}[leftmargin=2em, topsep=2pt]
  \item Histogram jest ``schodkowy'' --- nie ma gładkiego gradientu
        (mała zmiana $u_j$ nie zmienia histogramu, dopóki punkt nie przeskoczy granicy binu).
  \item Wymaga sortowania danych --- $O(N \log N)$.
  \item Kształt zależy od wyboru binów (ile? jak szerokie?).
\end{itemize}
Dlatego SIGReg \textbf{nie} porównuje histogramów, lecz używa \textbf{funkcji charakterystycznych}
(sekcja~\ref{sec:sigreg}.4 dalej) --- mają gładki gradient, złożoność $O(N)$
i~nie wymagają żadnych arbitralnych wyborów.

\end{keyinsight}

\subsubsection{Od jednego kierunku do wielu --- dlaczego testujemy wiele $\mathbf{a}$?}

Jeden kierunek $\mathbf{a}$ daje nam jedną ``perspektywę'' na dane.
Ale różne kierunki mogą ujawnić różne problemy:
\begin{itemize}[leftmargin=2em]
\item Kierunek wzdłuż osi $z_1$ sprawdza, czy wariancja w~wymiarze 1 jest poprawna.
\item Kierunek pod kątem sprawdza, czy wymiary są nieskorelowane.
\item Tylko testując \textit{wiele} kierunków, możemy stwierdzić, że rozkład jest
      izotropowy w~\textit{wszystkich} wymiarach.
\end{itemize}

\begin{figure}[H]
\centering
\includegraphics[width=\textwidth]{figures/projection_directions.pdf}
\caption{\textbf{Górny rząd}: Embeddingi izotropowe --- rzuty we wszystkich kierunkach
wyglądają jak $\mathcal{N}(0,1)$ (wariancja $\approx 1$).
\textbf{Dolny rząd}: Embeddingi anizotropowe (rozciągnięte wzdłuż $z_1$) ---
różne kierunki dają \textit{różne} wariancje (od $0.25$ do $4.0$).
SIGReg to wykrywa i~koryguje.}
\label{fig:projection_directions}
\end{figure}

\subsubsection{Podsumowanie: od obrazów do liczb 1D}

Cały pipeline wygląda tak:

\begin{figure}[H]
\centering
\includegraphics[width=\textwidth]{figures/projection_pipeline.pdf}
\caption{Pipeline SIGReg: batch obrazów $\to$ enkoder $\to$ embeddingi $K$-wymiarowe $\to$
rzut na losowy kierunek $\mathbf{a}$ $\to$ $N$ liczb 1D $\to$ test Eppsa-Pulleya $\to$ loss.}
\label{fig:projection_pipeline}
\end{figure}

\begin{keyinsight}[Dlaczego rzutujemy zamiast testować w~$K$ wymiarach?]
Test w~$K = 128$ wymiarach wymaga \textbf{astronomicznie dużo} danych
(tzw.\ ``klątwa wymiarowości'').
Rzut $u = \mathbf{a}^\top \mathbf{z}$ redukuje problem do 1D, gdzie test Eppsa-Pulleya
działa doskonale nawet z~małym batchem ($N = 64$).
Lemat Craméra-Wolda gwarantuje, że jeśli \textit{wszystkie} rzuty 1D
są $\mathcal{N}(0,1)$, to cały rozkład $K$-wymiarowy jest $\mathcal{N}(\mathbf{0}, \mathbf{I}_K)$.
W~praktyce losujemy $M \approx 1024$ kierunków --- to wystarczy.
\end{keyinsight}

%% ============================================================
\subsection{Test Eppsa-Pulleya na rzutach 1D}
%% ============================================================

Mamy rzut $u = \mathbf{a}^\top \mathbf{z}$ --- jedną liczbę dla każdego embeddingu.
Z~batcha $N$ embeddingów dostajemy $N$ takich liczb: $u_1, u_2, \ldots, u_N$.

\textbf{Pytanie:} czy te $N$ liczb wyglądają jak próbka z~$\mathcal{N}(0,1)$?

Żeby to sprawdzić, porównujemy dwa obiekty --- \textbf{funkcje charakterystyczne} (CF).
Zanim przejdziemy do wzorów SIGReg, wyjaśnijmy dokładnie, czym jest CF
i~skąd się bierze --- od samych podstaw.

%% ----- Element 0: Motywacja --- po co nam CF? -----
\subsubsection{Element 0: Po co nam funkcja charakterystyczna?}

Mamy $N$ liczb $u_1, \ldots, u_N$ (rzuty embeddingów).
Chcemy odpowiedzieć na pytanie: \textit{``Czy te liczby pochodzą z~rozkładu $\mathcal{N}(0,1)$?''}

Moglibyśmy narysować histogram i~``na oko'' porównać z~krzywą dzwonową ---
ale to nie jest precyzyjne i~nie nadaje się jako funkcja straty do treningu sieci.
Potrzebujemy narzędzia, które:
\begin{enumerate}[leftmargin=2em]
  \item \textbf{jednoznacznie identyfikuje} rozkład (nie tylko średnią czy wariancję,
        ale \textit{cały kształt}),
  \item daje \textbf{liczbę}, którą można zróżniczkować (gradient dla optymalizatora),
  \item działa dobrze nawet z~\textbf{małą próbką} ($N = 64$).
\end{enumerate}
Takim narzędziem jest właśnie \textbf{funkcja charakterystyczna}.

%% ----- Element 0.5: Liczby zespolone --- minimum potrzebne -----
\subsubsection{Element 0.5: Minimum o~liczbach zespolonych}

Funkcja charakterystyczna używa liczb zespolonych, więc przypomnijmy
absolutne minimum:

\begin{tcolorbox}[colback=gray!5, colframe=gray!60, breakable,
  title={Liczby zespolone --- co musimy wiedzieć}]

\textbf{Jednostka urojona:}
\[
i = \sqrt{-1}, \qquad i^2 = -1
\]
Liczba zespolona to para: $z = a + bi$, gdzie $a$ to \textbf{część rzeczywista},
$b$ to \textbf{część urojona}.

\textbf{Przykłady:}
\begin{itemize}[leftmargin=2em]
  \item $3 + 2i$ --- część rzeczywista $= 3$, część urojona $= 2$
  \item $5$ --- to też liczba zespolona (z~częścią urojoną $= 0$)
  \item $4i$ --- część rzeczywista $= 0$, część urojona $= 4$
\end{itemize}

\textbf{Geometrycznie:} Liczba zespolona to punkt na płaszczyźnie.
Oś pozioma to część rzeczywista, oś pionowa to część urojona:

\begin{center}
\begin{tikzpicture}[scale=0.9]
  \draw[->] (-1.5,0) -- (4.5,0) node[right] {Re (rzeczywista)};
  \draw[->] (0,-1.5) -- (0,3.5) node[above] {Im (urojona)};
  \fill[lejepaBlue] (3,2) circle (3pt) node[above right] {$3 + 2i$};
  \draw[dashed, gray] (3,0) node[below] {$3$} -- (3,2) -- (0,2) node[left] {$2$};
  \fill[lejepaRed] (0,0) circle (2pt);
\end{tikzpicture}
\end{center}

\textbf{Moduł} (odległość od zera):
$|a + bi| = \sqrt{a^2 + b^2}$

\end{tcolorbox}

%% ----- Element 0.6: Wzór Eulera -----
\subsubsection{Element 0.6: Wzór Eulera --- klucz do CF}

Najważniejszy wzór, na którym opiera się cała teoria CF:

\begin{tcolorbox}[colback=blue!5, colframe=blue!60!black, breakable,
  title={Wzór Eulera (Leonhard Euler, 1748)}]
\begin{equation}
e^{i\theta} = \cos\theta + i\sin\theta
\label{eq:euler}
\end{equation}
gdzie $\theta$ to dowolna liczba rzeczywista (kąt w~radianach).
\end{tcolorbox}

Skąd się bierze ten wzór? I~dlaczego kreśli okrąg?
Wyprowadźmy to krok po kroku --- dla czystej przyjemności matematyki.

%% --- Wyprowadzenie wzoru Eulera ---
\begin{tcolorbox}[colback=violet!5, colframe=violet!70!black, breakable,
  title={Wyprowadzenie wzoru Eulera z~szeregów Taylora}]

\textbf{Punkt wyjścia:} trzy funkcje, które znamy z~analizy matematycznej,
mają rozwinięcia w~\textbf{szereg Taylora} (nieskończoną sumę potęg):

\medskip
\textbf{Krok 1.} Szereg Taylora dla $e^x$ (wokół $x = 0$):
\[
e^x = 1 + x + \frac{x^2}{2!} + \frac{x^3}{3!} + \frac{x^4}{4!} + \frac{x^5}{5!} + \cdots
= \sum_{n=0}^{\infty} \frac{x^n}{n!}
\]
(Przypomnienie: $n! = 1 \cdot 2 \cdot 3 \cdots n$, np.\ $4! = 24$.)

\medskip
\textbf{Krok 2.} Szeregi Taylora dla $\cos$ i~$\sin$:
\[
\cos\theta = 1 - \frac{\theta^2}{2!} + \frac{\theta^4}{4!} - \frac{\theta^6}{6!} + \cdots
\qquad\text{(same \textbf{parzyste} potęgi, znaki naprzemienne)}
\]
\[
\sin\theta = \theta - \frac{\theta^3}{3!} + \frac{\theta^5}{5!} - \frac{\theta^7}{7!} + \cdots
\qquad\text{(same \textbf{nieparzyste} potęgi, znaki naprzemienne)}
\]

\medskip
\textbf{Krok 3.} Wstawiamy $x = i\theta$ do szeregu $e^x$:
\[
e^{i\theta} = 1 + (i\theta) + \frac{(i\theta)^2}{2!} + \frac{(i\theta)^3}{3!}
+ \frac{(i\theta)^4}{4!} + \frac{(i\theta)^5}{5!} + \cdots
\]

\textbf{Krok 4.} Obliczamy kolejne potęgi $i$:
\begin{center}
\renewcommand{\arraystretch}{1.2}
\begin{tabular}{c c l}
\toprule
Potęga & Wartość & Komentarz \\
\midrule
$i^0$ & $1$   & (każda liczba do potęgi $0$ to $1$) \\
$i^1$ & $i$   &  \\
$i^2$ & $-1$  & (definicja $i$) \\
$i^3$ & $-i$  & ($i^2 \cdot i = -1 \cdot i$) \\
$i^4$ & $1$   & ($i^2 \cdot i^2 = (-1)(-1) = 1$ --- \textbf{cykl się powtarza!}) \\
$i^5$ & $i$   & jak $i^1$ \\
$i^6$ & $-1$  & jak $i^2$ \\
\bottomrule
\end{tabular}
\end{center}
Cykl: $1, \; i, \; -1, \; -i, \; 1, \; i, \; -1, \; -i, \; \ldots$

\medskip
\textbf{Krok 5.} Wstawiamy potęgi $i$ do każdego wyrazu:
\begin{align*}
e^{i\theta} &= \underbrace{1}_{i^0}
+ \underbrace{i\theta}_{i^1}
+ \underbrace{\frac{i^2\theta^2}{2!}}_{=\, -\frac{\theta^2}{2!}}
+ \underbrace{\frac{i^3\theta^3}{3!}}_{=\, -\frac{i\theta^3}{3!}}
+ \underbrace{\frac{i^4\theta^4}{4!}}_{=\, +\frac{\theta^4}{4!}}
+ \underbrace{\frac{i^5\theta^5}{5!}}_{=\, +\frac{i\theta^5}{5!}}
+ \cdots
\end{align*}

\textbf{Krok 6.} Grupujemy --- wyrazy \textbf{bez $i$} osobno, wyrazy \textbf{z~$i$} osobno:
\[
e^{i\theta} =
\underbrace{\left(1 - \frac{\theta^2}{2!} + \frac{\theta^4}{4!} - \cdots\right)}_{\text{to jest }\cos\theta\text{!}}
+ \;i\;\underbrace{\left(\theta - \frac{\theta^3}{3!} + \frac{\theta^5}{5!} - \cdots\right)}_{\text{to jest }\sin\theta\text{!}}
\]

\medskip
\begin{center}
\fbox{\Large $e^{i\theta} = \cos\theta + i\sin\theta \qquad\checkmark$}
\end{center}

Wzór wyłania się sam --- wystarczyło wstawić $i\theta$ do szeregu $e^x$
i~pogrupować wyrazy!
\end{tcolorbox}

%% --- Dowód, że to okrąg ---
\begin{tcolorbox}[colback=green!5, colframe=green!60!black, breakable,
  title={Dowód, że $e^{i\theta}$ kreśli okrąg jednostkowy}]

Wiemy, że $e^{i\theta} = \cos\theta + i\sin\theta$.
Na płaszczyźnie zespolonej ten punkt ma współrzędne:
\[
(\underbrace{\cos\theta}_{\text{Re}},\; \underbrace{\sin\theta}_{\text{Im}})
\]

\textbf{Pytanie:} Jaką figurę kreślą te punkty, gdy $\theta$ przebiega od $0$ do $2\pi$?

\medskip
\textbf{Odległość od zera} (moduł):
\[
|e^{i\theta}| = \sqrt{(\cos\theta)^2 + (\sin\theta)^2}
= \sqrt{\underbrace{\cos^2\theta + \sin^2\theta}_{= \; 1 \text{ (tożsamość trygonometryczna!)}}}
= \sqrt{1} = 1
\]

Odległość od zera wynosi \textbf{zawsze 1}, niezależnie od~$\theta$.

\medskip
A~co to jest zbiór punktów w~odległości dokładnie $1$ od punktu $(0,0)$?

To jest \textbf{okrąg o~promieniu~1} ze środkiem w~zerze --- \textbf{okrąg jednostkowy}!

\[
\boxed{|e^{i\theta}| = 1 \quad\text{dla każdego }\theta
\qquad\Longrightarrow\qquad
e^{i\theta}\text{ leży na okręgu jednostkowym}}
\]

\medskip
\textbf{Co więcej:}
\begin{itemize}[leftmargin=2em]
  \item Gdy $\theta = 0$: \quad $e^{i\cdot 0} = \cos 0 + i\sin 0 = 1 + 0 = 1$
        \quad(start na prawo)
  \item Gdy $\theta$ rośnie: punkt przesuwa się \textbf{w~górę} i~w~lewo
        (kierunek $\sin$ rośnie)
  \item Gdy $\theta = 2\pi$: \quad $e^{i\cdot 2\pi} = \cos 2\pi + i\sin 2\pi = 1$
        \quad(wraca na start --- pełen obrót!)
\end{itemize}

\begin{center}
\begin{tikzpicture}[scale=1.4]
  \draw[gray!30] (0,0) circle (1);
  \draw[->] (-1.5,0) -- (1.7,0) node[right] {Re};
  \draw[->] (0,-1.5) -- (0,1.7) node[above] {Im};
  % Ścieżka z strzałkami
  \draw[lejepaBlue, thick, ->, >=stealth,
    postaction={decorate, decoration={markings,
      mark=at position 0.15 with {\arrow{>}},
      mark=at position 0.40 with {\arrow{>}},
      mark=at position 0.65 with {\arrow{>}},
      mark=at position 0.90 with {\arrow{>}}
    }}]
    (1,0) arc (0:350:1);
  % Punkty
  \fill[lejepaRed] (1,0) circle (2.5pt)
    node[below right] {$\theta\!=\!0$: start};
  \fill[lejepaBlue] ({cos(90)},{sin(90)}) circle (2pt)
    node[above right] {$\theta\!=\!\frac{\pi}{2}$};
  \fill[lejepaBlue] ({cos(180)},{sin(180)}) circle (2pt)
    node[above left] {$\theta\!=\!\pi$};
  \fill[lejepaBlue] ({cos(270)},{sin(270)}) circle (2pt)
    node[below left] {$\theta\!=\!\frac{3\pi}{2}$};
  % Promień = 1
  \draw[dashed, gray, thick] (0,0) -- ({cos(45)},{sin(45)})
    node[midway, below right] {$r = 1$};
  % Etykieta
  \node[text width=4cm, align=center] at (2.8, 0)
    {$\theta$ rośnie\\$\Downarrow$\\punkt obraca się\\przeciwnie do\\zegara};
\end{tikzpicture}
\end{center}

\end{tcolorbox}

\textbf{Bonus --- najpiękniejszy wzór matematyki.}
Wstawiając $\theta = \pi$ do wzoru Eulera:
\[
e^{i\pi} = \cos\pi + i\sin\pi = -1 + 0 = -1
\]
Przenosząc na drugą stronę:
\begin{equation}
\boxed{e^{i\pi} + 1 = 0}
\label{eq:euler_identity}
\end{equation}
Ten wzór łączy pięć najważniejszych stałych matematyki
($e$, $i$, $\pi$, $1$, $0$) w~jednym równaniu.
Nazywamy go \textbf{tożsamością Eulera}
--- często uważaną za najpiękniejszy wzór w~całej matematyce.

\bigskip
\textbf{Co to znaczy?} Wyrażenie $e^{i\theta}$ to punkt na \textbf{okręgu jednostkowym}
(okrąg o~promieniu~1) na płaszczyźnie zespolonej:

\begin{center}
\begin{tikzpicture}[scale=1.6]
  % okrąg
  \draw[gray!40] (0,0) circle (1);
  \draw[->] (-1.4,0) -- (1.6,0) node[right] {Re};
  \draw[->] (0,-1.4) -- (0,1.6) node[above] {Im};
  % punkt
  \def\ang{40}
  \coordinate (P) at ({cos(\ang)},{sin(\ang)});
  \draw[thick, lejepaBlue, ->] (0,0) -- (P);
  \fill[lejepaBlue] (P) circle (1.5pt) node[above right]
    {$e^{i\theta} = \cos\theta + i\sin\theta$};
  % kąt
  \draw[lejepaRed, thick] (0.3,0) arc (0:\ang:0.3);
  \node[lejepaRed] at (0.45,0.15) {$\theta$};
  % rzuty
  \draw[dashed, gray] (P) -- ({cos(\ang)},0)
    node[below, black] {\small$\cos\theta$};
  \draw[dashed, gray] (P) -- (0,{sin(\ang)})
    node[left, black] {\small$\sin\theta$};
  % specjalne punkty
  \fill[gray] (1,0) circle (1pt) node[below right] {\small$e^{i\cdot 0}=1$};
  \fill[gray] (0,1) circle (1pt) node[above left] {\small$e^{i\pi/2}=i$};
  \fill[gray] (-1,0) circle (1pt) node[below left] {\small$e^{i\pi}=-1$};
  \fill[gray] (0,-1) circle (1pt) node[below left] {\small$e^{i\cdot 3\pi/2}=-i$};
\end{tikzpicture}
\end{center}

\textbf{Kluczowe obserwacje:}
\begin{itemize}[leftmargin=2em]
  \item Gdy $\theta$ rośnie, punkt $e^{i\theta}$ \textbf{obraca się} po okręgu
        (przeciwnie do zegara).
  \item Moduł jest zawsze $|e^{i\theta}| = \sqrt{\cos^2\theta + \sin^2\theta} = 1$ ---
        punkt nigdy nie oddala się od zera.
  \item $e^{i\theta}$ to \textbf{obrót o~kąt $\theta$} na płaszczyźnie zespolonej.
\end{itemize}

\textbf{Przykłady konkretne:}
\begin{center}
\renewcommand{\arraystretch}{1.3}
\begin{tabular}{c c c c}
\toprule
$\theta$ & $\cos\theta$ & $\sin\theta$ & $e^{i\theta}$ \\
\midrule
$0$ & $1$ & $0$ & $1$ \\
$\pi/2 \approx 1{,}57$ & $0$ & $1$ & $i$ \\
$\pi \approx 3{,}14$ & $-1$ & $0$ & $-1$ \\
$2\pi \approx 6{,}28$ & $1$ & $0$ & $1$ (pełen obrót) \\
\bottomrule
\end{tabular}
\end{center}

%% ----- Element 0.7: Od Eulera do CF -----
\subsubsection{Element 0.7: Od wzoru Eulera do funkcji charakterystycznej}

Teraz zastosujmy wzór Eulera. Weźmy zmienną losową $X$
(np.\ nasze rzuty $u$) i~wpiszmy $\theta = tx$:
\[
e^{itx} = \cos(tx) + i\sin(tx)
\]

\begin{tcolorbox}[colback=gray!5, colframe=gray!60, breakable,
  title={Co oznacza $tx$ w~wyrażeniu $e^{itx}$?}]

W~wyrażeniu $e^{itx}$ mamy \textbf{dwie zmienne}:
\begin{itemize}[leftmargin=2em]
  \item $x$ --- \textbf{dana} (wartość z~naszych danych, np.\ rzut embeddingu).
        To jest konkretna liczba, np.\ $x = 1{,}3$ albo $x = -0{,}7$.
  \item $t$ --- \textbf{parametr}, który my wybieramy.
        Można go rozumieć jako ``częstotliwość skanowania''.
\end{itemize}

\textbf{Co robi iloczyn $tx$?} Określa \textbf{kąt obrotu} na okręgu jednostkowym.

Przypomnijmy: $e^{i\theta}$ to punkt na okręgu pod kątem $\theta$.
Tutaj $\theta = tx$, więc:
\begin{itemize}[leftmargin=2em]
  \item Gdy $t$ jest \textbf{małe} (np.\ $t = 0{,}1$),
        kąt $tx$ zmienia się \textbf{powoli} gdy $x$ rośnie ---
        nawet duże różnice w~$x$ dają małe różnice kąta.
  \item Gdy $t$ jest \textbf{duże} (np.\ $t = 10$),
        kąt $tx$ zmienia się \textbf{szybko} ---
        małe różnice w~$x$ powodują duże obroty na okręgu.
\end{itemize}

\textbf{Przykład liczbowy} --- dwa punkty danych $x_1 = 1{,}0$ i~$x_2 = 1{,}5$:

\begin{center}
\renewcommand{\arraystretch}{1.3}
\begin{tabular}{c c c c l}
\toprule
$t$ & $tx_1$ & $tx_2$ & $tx_2 - tx_1$ & Interpretacja \\
\midrule
$0{,}5$ & $0{,}5$\,rad & $0{,}75$\,rad & $0{,}25$\,rad ($14°$)
  & wolne skanowanie --- punkty blisko siebie \\
$2$ & $2$\,rad & $3$\,rad & $1$\,rad ($57°$)
  & średnie skanowanie \\
$10$ & $10$\,rad & $15$\,rad & $5$\,rad ($286°$)
  & szybkie skanowanie --- punkty prawie po drugiej stronie \\
\bottomrule
\end{tabular}
\end{center}

\textbf{Analogia radiowa:} $t$ to ``częstotliwość stacji radiowej'', na którą się stroimy.
Różne częstotliwości $t$ wychwytują różne cechy rozkładu,
tak jak różne stacje radiowe nadają różną muzykę.
CF~$\varphi(t)$ mówi nam ``co słychać na częstotliwości $t$''.
\end{tcolorbox}

\textbf{Wyobraź sobie:} Masz $N$ wartości $x_1, x_2, \ldots, x_N$ (np.\ rzuty embeddingów).
Dla ustalonego $t$, każda wartość $x_j$ daje punkt na okręgu:
\[
x_j \;\longrightarrow\; e^{itx_j} = \cos(tx_j) + i\sin(tx_j)
\quad\text{--- punkt na okręgu jednostkowym}
\]
Mamy więc $N$ punktów na okręgu. Ich \textbf{średnia} (środek ciężkości)
to przybliżenie funkcji charakterystycznej:

\begin{tcolorbox}[colback=orange!5, colframe=orange!70!black, breakable,
  title={Intuicja: CF to średnia punktów na okręgu}]

\begin{center}
\begin{tikzpicture}[scale=1.3]
  % okrąg
  \draw[gray!30] (0,0) circle (1);
  \draw[->] (-1.4,0) -- (1.6,0) node[right] {Re};
  \draw[->] (0,-1.4) -- (0,1.6) node[above] {Im};
  % punkty -- symulacja "blisko Gaussa"
  \foreach \ang/\lab in {15/1, 55/2, 130/3, 200/4, 280/5, 340/6} {
    \fill[lejepaBlue!70] ({cos(\ang)},{sin(\ang)}) circle (2pt);
  }
  % średnia ~blisko (0.1, 0.05)
  \fill[lejepaRed] (0.12,0.05) circle (3pt)
    node[above right] {$\overline{e^{itx_j}}$ = CF};
  \draw[lejepaRed, thick, ->] (0,0) -- (0.12,0.05);
  \node[below, text width=6cm, align=center] at (0,-1.7)
    {$N=6$ punktów (niebieskie) i~ich średnia (czerwona)};
\end{tikzpicture}
\end{center}

Jeśli dane $x_j$ są ``ładnie rozłożone'' (bliskie Gaussowi),
punkty rozkładają się \textbf{równomiernie} po okręgu i~ich średnia
jest bliska zeru. Jeśli dane mają strukturę (nie-Gaussowską),
punkty \textbf{skupiają się} w~jednym miejscu i~średnia jest daleko od zera.

To właśnie mierzy funkcja charakterystyczna!
\end{tcolorbox}

%% ----- Element 1: Formalna definicja CF -----
\subsubsection{Element 1: Formalna definicja funkcji charakterystycznej}

Teraz możemy podać formalną definicję --- po powyższych wyjaśnieniach
powinna być już zrozumiała:

\begin{tcolorbox}[
  colback=lejepaGreen!8,
  colframe=lejepaGreen!80,
  fonttitle=\bfseries,
  title={Definicja: Funkcja charakterystyczna (CF)},
  breakable,
]

Dla zmiennej losowej $X$ o~rozkładzie $p(x)$:
\begin{equation}
\varphi_X(t) = \mathbb{E}\big[e^{itX}\big]
= \int_{-\infty}^{\infty} e^{itx}\,p(x)\,dx
\label{eq:cf_def}
\end{equation}

Rozbijmy ten wzór na elementy:
\begin{center}
\renewcommand{\arraystretch}{1.4}
\begin{tabular}{c p{10cm}}
\toprule
\textbf{Symbol} & \textbf{Znaczenie} \\
\midrule
$\varphi_X(t)$ & wynik: jedna liczba zespolona dla każdej częstotliwości $t$ \\
$\mathbb{E}[\cdot]$ & wartość oczekiwana (``średnia'') \\
$e^{itx}$ & punkt na okręgu jednostkowym (wzór Eulera: $\cos(tx) + i\sin(tx)$) \\
$t$ & parametr --- ``częstotliwość'', po której skanujemy rozkład \\
$p(x)$ & gęstość prawdopodobieństwa rozkładu $X$ \\
$\int_{-\infty}^{\infty} \ldots\, dx$ & sumujemy po wszystkich możliwych wartościach $x$ \\
\bottomrule
\end{tabular}
\end{center}

\medskip
\textbf{Słownie:} ``Dla każdej częstotliwości $t$, bierzemy średnią ważoną
punktów na okręgu $e^{itx}$, gdzie wagami jest prawdopodobieństwo $p(x)$.''

\end{tcolorbox}

\textbf{Kluczowe własności CF} (dlaczego jest tak użyteczna):
\begin{enumerate}[leftmargin=2em]
  \item \textbf{Jednoznaczność:} Dwa rozkłady są identyczne
        $\iff$ ich CF-y są identyczne.
        Dlatego porównując CF-y, porównujemy \textit{całe} rozkłady ---
        nie tylko średnią czy wariancję, ale cały kształt.
  \item \textbf{Zawsze istnieje:} W~przeciwieństwie do momentów (średnia, wariancja, \ldots),
        CF istnieje dla \textit{każdego} rozkładu --- nawet takiego,
        który nie ma skończonej wariancji.
  \item \textbf{Gładkość:} CF jest ciągłą, różniczkowalną funkcją parametru $t$ ---
        idealna do optymalizacji gradientowej.
\end{enumerate}

\textbf{Analogia:} Pomyśl o~strojeniu gitary. Uderzasz w~strunę i~słyszysz dźwięk ---
mieszaninę częstotliwości. Funkcja charakterystyczna robi to samo z~rozkładem
prawdopodobieństwa: ``rozkłada'' go na częstotliwości.
Różne rozkłady dają różne ``dźwięki''.

\begin{tcolorbox}[colback=yellow!5, colframe=yellow!70!black,
  title={Przykład: CF dla rzutu monetą}]
Rzut monetą: $X = -1$ (orzeł) z~prawdopodobieństwem $\frac{1}{2}$,
$X = +1$ (reszka) z~prawdopodobieństwem $\frac{1}{2}$.
\begin{align*}
\varphi_X(t) &= \mathbb{E}[e^{itX}]
= \tfrac{1}{2}\,e^{it\cdot(-1)} + \tfrac{1}{2}\,e^{it\cdot(+1)} \\
&= \tfrac{1}{2}\,e^{-it} + \tfrac{1}{2}\,e^{it}
= \tfrac{1}{2}\big(\underbrace{e^{-it} + e^{it}}_{= 2\cos t}\big) = \cos t
\end{align*}
CF rzutu monetą to po prostu $\cos t$ --- czysta oscylacja!
Wygląda zupełnie inaczej niż CF Gaussa (który jest $e^{-t^2/2}$, gładki dzwonek).
\end{tcolorbox}

%% ----- Element 2: CF standardowego Gaussa -----
\subsubsection{Element 2: CF standardowego Gaussa --- cel, do którego dążymy}

Jeśli $X \sim \mathcal{N}(0,1)$, to jego funkcja charakterystyczna ma piękną, prostą postać:
\begin{equation}
\varphi_{\mathcal{N}(0,1)}(t) = e^{-t^2/2}
\label{eq:cf_gauss}
\end{equation}

\textbf{Skąd ten wzór?} Wstawiamy gęstość Gaussa $p(x) = \frac{1}{\sqrt{2\pi}}e^{-x^2/2}$
do definicji CF:
\[
\varphi(t) = \int_{-\infty}^{\infty} e^{itx} \cdot \frac{1}{\sqrt{2\pi}}e^{-x^2/2}\,dx
= \frac{1}{\sqrt{2\pi}} \int_{-\infty}^{\infty} e^{itx - x^2/2}\,dx
\]
Wykładnik $itx - x^2/2$ można uzupełnić do kwadratu (``completing the square'').
Pokażemy to krok po kroku:

\medskip
\textbf{Krok 1.} Wyciągamy $-\frac{1}{2}$ przed wyrażenie z~$x$:
\[
itx - \frac{x^2}{2}
= -\frac{1}{2}\!\left(x^2 - 2it\,x\right)
\]
(sprawdzenie: $-\frac{1}{2}\cdot x^2 = -\frac{x^2}{2}$ \checkmark\quad
$-\frac{1}{2}\cdot(-2it\,x) = +itx$ \checkmark)

\medskip
\textbf{Krok 2.} Uzupełniamy do kwadratu wewnątrz nawiasu.
Chcemy zapisać $x^2 - 2it\,x$ jako $(x - \text{coś})^2 - \text{reszta}$.

Przypomnijmy wzór skróconego mnożenia:
\[
(x - a)^2 = x^2 - 2ax + a^2
\]
Porównujemy z~naszym wyrażeniem $x^2 - 2it\,x$:
\[
\underbrace{x^2 - 2\,it\,x}_{\text{nasze}} \quad\longleftrightarrow\quad
\underbrace{x^2 - 2\,a\,x + a^2}_{(x-a)^2}
\qquad\Rightarrow\quad a = it
\]
Zatem:
\[
x^2 - 2it\,x = \underbrace{(x - it)^2}_{= x^2 - 2it\,x + (it)^2} - (it)^2
\]
Musimy odjąć $(it)^2$, bo $(x-it)^2$ zawiera dodatkowy składnik $(it)^2$,
którego nie było w~oryginale.

\medskip
\textbf{Krok 3.} Obliczamy $(it)^2$:
\[
(it)^2 = i^2 \cdot t^2 = (-1)\cdot t^2 = -t^2
\]
(bo $i = \sqrt{-1}$, więc $i^2 = -1$).

\medskip
\textbf{Krok 4.} Wstawiamy z~powrotem:
\[
x^2 - 2it\,x = (x - it)^2 - (-t^2) = (x-it)^2 + t^2
\]
A~z~czynnikiem $-\frac{1}{2}$ z~Kroku~1:
\[
-\frac{1}{2}\!\left(x^2 - 2it\,x\right)
= -\frac{1}{2}\!\left[(x - it)^2 + t^2\right]
= -\frac{1}{2}(x - it)^2 - \frac{t^2}{2}
\]

\begin{tcolorbox}[colback=blue!5, colframe=blue!50!black, title={Wynik: uzupełnienie do kwadratu}]
\[
\boxed{itx - \frac{x^2}{2} = -\frac{1}{2}(x - it)^2 - \frac{t^2}{2}}
\]
\end{tcolorbox}
Czynnik $e^{-t^2/2}$ nie zależy od~$x$, więc wychodzi przed całkę,
a~pozostała całka $\int e^{-(x-it)^2/2}\,dx = \sqrt{2\pi}$ (Gaussowska!),
skraca się z~$1/\sqrt{2\pi}$:
\[
\varphi(t) = e^{-t^2/2} \cdot \underbrace{\frac{1}{\sqrt{2\pi}}\int e^{-(x-it)^2/2}\,dx}_{=\,1} = e^{-t^2/2}
\]

\textbf{Intuicja:} $e^{-t^2/2}$ to ``dzwonek'' --- szybko maleje dla dużych $|t|$.
To znaczy, że Gauss $\mathcal{N}(0,1)$ jest ``gładki'' i~nie ma ostrych krawędzi
(wysokie częstotliwości są tłumione).

\bigskip

\textbf{To jest nasz cel:} chcemy, żeby rzuty embeddingów miały CF
wyglądającą dokładnie jak $e^{-t^2/2}$.

%% ----- Element 3: Empiryczna CF -----
\subsubsection{Element 3: Empiryczna CF --- co mamy w~praktyce}

W~treningu nie znamy rozkładu $p(x)$ analitycznie --- mamy tylko \textbf{próbkę}:
$N$ liczb $u_1, u_2, \ldots, u_N$ (rzuty embeddingów z~batcha).

\textbf{Empiryczna funkcja charakterystyczna (ECF)} to przybliżenie prawdziwej CF
na podstawie próbki:
\begin{equation}
\hat{\varphi}_X(t) = \frac{1}{N}\sum_{j=1}^{N} e^{itu_j}
\label{eq:ecf}
\end{equation}

\textbf{Co tu robimy?}
\begin{enumerate}[leftmargin=2em]
  \item Dla każdego punktu $u_j$ obliczamy $e^{itu_j} = \cos(tu_j) + i\sin(tu_j)$
        --- to punkt na okręgu jednostkowym w~płaszczyźnie zespolonej.
  \item Uśredniamy te $N$ punktów (suma podzielona przez $N$).
  \item Wynik to jedno przybliżenie CF dla częstotliwości $t$.
  \item Powtarzamy dla różnych wartości $t$, żeby dostać całą krzywą $\hat{\varphi}(t)$.
\end{enumerate}

\textbf{Analogia:} Wyobraź sobie, że rzucasz $N$ kamyków do jeziora.
Każdy kamyk tworzy falę. ECF to \textit{średnia} tych fal --- jeśli kamyki
są rozłożone jak Gauss, średnia fal będzie wyglądała jak $e^{-t^2/2}$.
Jeśli nie --- będzie wyglądała inaczej.

\bigskip

Im więcej punktów ($N$ większe), tym ECF jest bliższa prawdziwej CF
(prawo wielkich liczb: średnia z~próbki zbliża się do wartości oczekiwanej).

%% ----- Element 4: Porównanie --- odległość między CF-ami -----
\subsubsection{Element 4: Porównanie --- jak mierzymy różnicę?}

Mamy dwie krzywe:
\begin{itemize}[leftmargin=2em]
  \item $\hat{\varphi}_X(t)$ --- to, co mamy (z~danych),
  \item $\varphi_{\mathcal{N}(0,1)}(t) = e^{-t^2/2}$ --- to, co chcemy.
\end{itemize}

Mierzymy, jak bardzo się różnią --- kwadrat różnicy, scałkowany po wszystkich
częstotliwościach $t$:
\[
\int_{-\infty}^{\infty}
\left|\hat{\varphi}_X(t) - e^{-t^2/2}\right|^2 \, dt
\]

\textbf{Dlaczego kwadrat $|\cdot|^2$?}
Ponieważ $\hat{\varphi}$ i~$\varphi$ to liczby zespolone, ``zwykłe'' odejmowanie
może dawać wartości ujemne i~dodatnie, które by się znosiły.
Kwadrat modułu $|a - b|^2$ jest zawsze $\geq 0$ --- mierzy ``odległość'' bez znaków.

\textbf{Dlaczego całka po~$t$?}
Bo chcemy porównać krzywe \textit{na całej dziedzinie}, nie tylko w~jednym punkcie.
Gdybyśmy sprawdzili tylko $t = 0$, obie CF dają $1$ (zawsze!) --- nic byśmy nie wykryli.

%% ----- Element 5: Waga $w(t)$ -----
\subsubsection{Element 5: Waga $w(t)$ --- dlaczego nie wszystkie częstotliwości są równie ważne}

Jest jeden problem: całka $\int_{-\infty}^{\infty} |\cdot|^2\,dt$ po \textit{wszystkich}
częstotliwościach może nie istnieć (być nieskończona).
Poza tym, bardzo wysokie częstotliwości ($|t| \gg 1$) niosą mało informacji,
bo obie CF i~tak są tam bliskie zeru.

Rozwiązanie: mnożymy przez \textbf{wagę} $w(t)$, która tłumi wysokie częstotliwości:
\begin{equation}
w(t) = e^{-t^2/\sigma^2}
\label{eq:weight}
\end{equation}

To Gaussowski ``filtr'' --- dla małych $|t|$ waga jest bliska $1$ (liczymy normalnie),
dla dużych $|t|$ waga spada do~$0$ (ignorujemy).
Parametr $\sigma$ kontroluje, jak szybko waga maleje (typowo $\sigma = 1$).

%% ----- Element 6: Pełny wzór EP -----
\subsubsection{Element 6: Pełny wzór testu Eppsa-Pulleya}

Składamy wszystkie elementy razem:

\begin{equation}
\boxed{
\text{EP} = N \int_{-\infty}^{\infty}
\underbrace{\left|\hat{\varphi}_X(t) - \varphi_{\mathcal{N}(0,1)}(t)\right|^2}_{\text{kwadrat różnicy CF-ów}}
\;\underbrace{w(t)}_{\text{waga}}\,dt
}
\label{eq:ep}
\end{equation}

gdzie:
\begin{center}
\renewcommand{\arraystretch}{1.8}
\begin{tabular}{p{3cm}p{4cm}p{7cm}}
\toprule
\textbf{Symbol} & \textbf{Wzór} & \textbf{Co robi?} \\
\midrule
$\hat{\varphi}_X(t)$ &
$\dfrac{1}{N}\displaystyle\sum_{j=1}^{N} e^{itu_j}$ &
ECF --- ``odcisk palca'' naszych danych (z~próbki) \\
$\varphi_{\mathcal{N}(0,1)}(t)$ &
$e^{-t^2/2}$ &
CF standardowego Gaussa --- ``odcisk palca'' celu \\
$w(t)$ &
$e^{-t^2/\sigma^2}$ &
Waga --- tłumi wysokie częstotliwości \\
$N$ &
rozmiar batcha &
Skalowanie --- większy batch $\Rightarrow$ bardziej czuły test \\
\bottomrule
\end{tabular}
\end{center}

\bigskip

\textbf{Interpretacja wyniku:}
\begin{itemize}[leftmargin=2em]
  \item $\text{EP} = 0$ $\Rightarrow$ embeddingi mają \textit{dokładnie} rozkład $\mathcal{N}(0,1)$
        w~tym kierunku (CF-y się pokrywają).
  \item $\text{EP} > 0$ $\Rightarrow$ embeddingi \textit{odstają} od Gaussa ---
        im większa wartość, tym gorsze dopasowanie.
  \item SIGReg minimalizuje EP $\Rightarrow$ ``pcha'' embeddingi w~stronę $\mathcal{N}(0,1)$.
\end{itemize}

\begin{keyinsight}[Dlaczego CF a~nie np.\ histogram?]
Histogram wymaga \textbf{sortowania} danych ($O(N \log N)$) i~nie ma gładkiego gradientu
(jest ``schodkowy''). CF jest \textbf{gładką} funkcją parametrów sieci ---
każdy $e^{itu_j}$ jest różniczkowalny po~$u_j$, a~$u_j = \mathbf{a}^\top f_\theta(\mathbf{x}_j)$
jest różniczkowalny po wagach $\theta$.
Dlatego gradient EP po~$\theta$ istnieje i~jest ograniczony --- idealne do SGD.
\end{keyinsight}

\begin{figure}[H]
\centering
\includegraphics[width=\textwidth]{figures/characteristic_functions.pdf}
\caption{\textbf{Lewo}: Gęstości rzutów 1D --- izotropowy ma $\sigma^2=1$ (cel),
anizotropowy ma różne wariancje w~różnych kierunkach.
\textbf{Prawo}: Funkcje charakterystyczne i~błąd Eppsa-Pulleya ---
różnica między empiryczną CF a~docelową $e^{-t^2/2}$.}
\label{fig:cf}
\end{figure}

\subsection{SIGReg: pełna definicja}

\begin{definition}[Sketched Isotropic Gaussian Regularization]
\begin{equation}
\boxed{
\mathrm{SIGReg}_T(\mathbb{A}, \{f_\theta(\mathbf{x}_n)\}_{n=1}^N)
\triangleq \frac{1}{|\mathbb{A}|} \sum_{\mathbf{a} \in \mathbb{A}}
T\!\left(\{\mathbf{a}^\top f_\theta(\mathbf{x}_n)\}_{n=1}^N\right)
}
\label{eq:sigreg}
\end{equation}
gdzie $\mathbb{A} = \{\mathbf{a}_1, \ldots, \mathbf{a}_M\}$ to losowe kierunki jednostkowe,
$T$ to test Eppsa-Pulleya, $M \approx 1024$.
\end{definition}

\subsection{Dlaczego Epps-Pulley a nie inne testy?}

\begin{center}
\begin{tabular}{lccc}
\toprule
\textbf{Test} & \textbf{Gradient} & \textbf{Stabilność} & \textbf{DDP} \\
\midrule
Momenty (Jarque-Bera) & eksplodujący & niska & tak \\
CDF (Cramér-von Mises) & wymaga sortowania & brak & nie \\
CF (Epps-Pulley) & ograniczony & wysoka & tak \\
\bottomrule
\end{tabular}
\end{center}

Epps-Pulley ma:
\begin{itemize}
  \item Ograniczony gradient: $|\partial \text{EP}/\partial z_i| \leq 4\sigma^2/N$ (Tw.~4 w artykule),
  \item Liniową złożoność: $O(N)$ pamięci i czasu,
  \item Naturalną kompatybilność z DDP: ECF to średnia $\Rightarrow$ \texttt{all\_reduce}.
\end{itemize}


\clearpage
%% ============================================================
\section{Pełny loss LeJEPA}
\label{sec:fulloss}
%% ============================================================

\begin{equation}
\boxed{
\mathcal{L}_{\text{LeJEPA}} =
\underbrace{\frac{\lambda}{V} \sum_{v=1}^{V} \mathrm{SIGReg}\!\left(\{\mathbf{z}_{n,v}\}_{n=1}^B\right)}_{\text{regularyzacja: wymuś } \mathcal{N}(\mathbf{0}, \mathbf{I})}
+ \underbrace{\frac{1-\lambda}{B} \sum_{n=1}^{B} \mathcal{L}_{\text{pred}}^{(V_g)}\!\left(\{\mathbf{z}_{n,v}\}_{v=1}^V\right)}_{\text{predykcja: wymuś semantyczną spójność}}
}
\label{eq:lejepa}
\end{equation}

gdzie loss predykcyjny to:
\begin{equation}
\mathcal{L}_{\text{pred}} = \frac{1}{V}\sum_{v'=1}^{V}
\left\|\boldsymbol{\mu}_n - \mathbf{z}_{n,v'}\right\|_2^2,
\quad
\boldsymbol{\mu}_n \triangleq \frac{1}{V_g}\sum_{v=1}^{V_g} \mathbf{z}_{n,v}
\label{eq:pred_loss}
\end{equation}

\begin{keyinsight}[Jeden hiperparametr!]
$\lambda$ kontroluje trade-off między:
\begin{itemize}
  \item $\lambda \to 0$: czysta predykcja (ryzyko kolapsu),
  \item $\lambda \to 1$: czysty SIGReg (brak semantycznej struktury),
  \item $\lambda = 0.05$: \textbf{rekomendowane} — stabilne na wielu datasetach i architekturach.
\end{itemize}
\end{keyinsight}

%% --- Szczegółowe wyjaśnienie wzoru (25) ---
\subsection{Jak czytać wzór (\ref{eq:lejepa}) --- słownik symboli}

Zanim przeczytamy wzór, ustalmy co oznacza \textbf{każdy symbol}:

\begin{center}
\renewcommand{\arraystretch}{1.4}
\begin{tabular}{c p{10cm}}
\toprule
\textbf{Symbol} & \textbf{Znaczenie} \\
\midrule
$\mathcal{L}_{\text{LeJEPA}}$ &
  Pełna \textbf{funkcja straty} (loss) --- jedna liczba, którą sieć minimalizuje
  w~każdym kroku treningu. Im mniejsza, tym lepiej. \\
$\lambda$ &
  \textbf{Hiperparametr balansujący} --- liczba z~przedziału $[0, 1]$
  (typowo $0{.}05$). Kontroluje proporcję:
  ile ``uwagi'' poświęcamy regularyzacji, a~ile predykcji. \\
$V$ &
  \textbf{Liczba wszystkich widoków} (views) jednego obrazu.
  W~LeJEPA: $V = V_g + V_l$, np.\ $V = 2 + 6 = 8$
  (2~globalne + 6~lokalnych). \\
$V_g$ &
  \textbf{Liczba widoków globalnych} (global views) ---
  duże kadrowania ($224 \times 224$) z~jednego obrazu. Typowo $V_g = 2$. \\
$V_l$ &
  \textbf{Liczba widoków lokalnych} (local views) ---
  małe kadrowania ($96 \times 96$). Typowo $V_l = 6$. \\
$B$ &
  \textbf{Rozmiar batcha} --- ile obrazów przetwarzamy na raz
  (np.\ $B = 256$). \\
$n$ &
  \textbf{Indeks obrazu} w~batchu: $n \in \{1, 2, \ldots, B\}$. \\
$v$ &
  \textbf{Indeks widoku}: $v \in \{1, 2, \ldots, V\}$. \\
$\mathbf{z}_{n,v} \in \mathbb{R}^d$ &
  \textbf{Embedding} $n$-tego obrazu z~$v$-tego widoku ---
  wektor $d$-wymiarowy (np.\ $d = 128$) wyprodukowany
  przez sieć (ViT + projektor). \\
$\mathrm{SIGReg}(\cdot)$ &
  Regularyzator z~sekcji~9.5 --- mierzy, jak daleko embeddingi
  są od rozkładu $\mathcal{N}(\mathbf{0}, \mathbf{I})$. \\
$\mathcal{L}_{\text{pred}}$ &
  \textbf{Loss predykcyjny} --- mierzy, jak bardzo embeddingi różnych
  widoków \textit{tego samego} obrazu różnią się od siebie. \\
$\boldsymbol{\mu}_n$ &
  \textbf{Centroid} (średnia) embeddingów widoków globalnych
  $n$-tego obrazu. Punkt odniesienia, do którego
  porównujemy wszystkie widoki. \\
$\|\cdot\|_2^2$ &
  \textbf{Kwadrat normy euklidesowej} --- suma kwadratów
  różnic po współrzędnych: $\|\mathbf{a} - \mathbf{b}\|_2^2 = \sum_i (a_i - b_i)^2$. \\
$\triangleq$ &
  ``\textbf{definiujemy jako}'' --- symbol oznacza, że lewa strona jest
  \textit{nową definicją}, nie wynikiem obliczeń. \\
\bottomrule
\end{tabular}
\end{center}

\subsection{Czytanie wzoru (\ref{eq:lejepa}) od lewej do prawej}

Teraz czytamy wzór po polsku, symbol po symbolu:

\begin{tcolorbox}[colback=blue!4, colframe=blue!60!black, breakable,
  title={\textbf{Wzór (\ref{eq:lejepa}) --- jak go przeczytać}}]

\[
\mathcal{L}_{\text{LeJEPA}} =
\underbrace{\frac{\lambda}{V} \sum_{v=1}^{V} \mathrm{SIGReg}\!\left(\{\mathbf{z}_{n,v}\}_{n=1}^B\right)}_{\text{część A}}
+
\underbrace{\frac{1-\lambda}{B} \sum_{n=1}^{B} \mathcal{L}_{\text{pred}}^{(V_g)}\!\left(\{\mathbf{z}_{n,v}\}_{v=1}^V\right)}_{\text{część B}}
\]

\textbf{Część A --- regularyzacja SIGReg:}

\medskip
Czytamy: \textit{``Weź wagę $\lambda$, podziel przez liczbę widoków $V$,
i~dla każdego widoku $v$ od 1 do $V$ policz SIGReg
na zbiorze embeddingów wszystkich $B$ obrazów w~tym widoku.''}

\begin{enumerate}[leftmargin=2em]
  \item Fiksujemy widok $v$ (np.\ ``widok globalny nr~1'').
  \item Zbieramy embeddingi \textbf{tego widoku} ze~wszystkich obrazów w~batchu:
        $\{\mathbf{z}_{1,v},\; \mathbf{z}_{2,v},\; \ldots,\; \mathbf{z}_{B,v}\}$
        --- to $B$ wektorów w~$\mathbb{R}^d$.
  \item Sprawdzamy SIGRegiem: czy te $B$ wektorów tworzą
        rozkład $\mathcal{N}(\mathbf{0}, \mathbf{I})$?
  \item Powtarzamy dla każdego widoku $v = 1, \ldots, V$ i~uśredniamy ($\div V$).
  \item Mnożymy przez wagę $\lambda$ (typowo $0{.}05$).
\end{enumerate}

\textbf{Intuicja}: regularyzacja sprawdza \textit{kolumnami} ---
``czy chmura embeddingów z~każdego widoku wygląda jak izotropowy Gauss?''

\bigskip
\textbf{Część B --- predykcja:}

\medskip
Czytamy: \textit{``Weź wagę $(1-\lambda)$, podziel przez liczbę obrazów $B$,
i~dla każdego obrazu $n$ policz loss predykcyjny na wszystkich jego widokach.''}

\begin{enumerate}[leftmargin=2em]
  \item Fiksujemy obraz $n$ (np.\ ``zdjęcie kota nr~17'').
  \item Zbieramy embeddingi \textbf{tego obrazu} ze~wszystkich widoków:
        $\{\mathbf{z}_{n,1},\; \mathbf{z}_{n,2},\; \ldots,\; \mathbf{z}_{n,V}\}$.
  \item Liczymy centroid z~widoków globalnych:
        $\boldsymbol{\mu}_n = \frac{1}{V_g}(\mathbf{z}_{n,1} + \mathbf{z}_{n,2})$.
  \item Mierzymy, jak daleko każdy widok jest od centroidu:
        $\|\boldsymbol{\mu}_n - \mathbf{z}_{n,v'}\|_2^2$.
  \item Uśredniamy po widokach ($\div V$) i~po obrazach ($\div B$).
  \item Mnożymy przez wagę $(1 - \lambda)$ (typowo $0{.}95$).
\end{enumerate}

\textbf{Intuicja}: predykcja sprawdza \textit{wierszami} ---
``czy różne widoki tego samego obrazu dają podobne embeddingi?''
\end{tcolorbox}

\subsubsection*{Skąd biorą się widoki? Obrazy vs wideo}

Pojęcie ``widoku'' (view) oznacza coś innego w~zależności od danych:

\begin{tcolorbox}[colback=orange!5, colframe=orange!70!black, breakable,
  title={\textbf{Widoki --- obrazy (np.\ ImageNet)}}]
Mamy jeden obraz, np.\ zdjęcie kota.
\textbf{Wszystkie} widoki powstają z~\textbf{tego samego} zdjęcia
przez \textbf{augmentacje} --- losowe kadrowania, obroty, zmiany kolorów:

\begin{center}
\begin{tikzpicture}[>=stealth, scale=0.85]
  % Oryginalny obraz
  \draw[thick] (0,0) rectangle (3,3);
  \node at (1.5,1.5) {\large obraz $\mathbf{x}_n$};
  \node[below] at (1.5,-0.2) {\small (np.\ zdjęcie kota)};

  % Strzałki
  \draw[->, thick] (3.3, 2.5) -- (5.0, 3.2);
  \draw[->, thick] (3.3, 2.0) -- (5.0, 2.0);
  \draw[->, thick] (3.3, 1.5) -- (5.0, 1.0);
  \draw[->, thick] (3.3, 0.8) -- (5.0, -0.2);

  % Widoki globalne
  \draw[blue!70!black, thick] (5.0, 2.7) rectangle (7.5, 3.7);
  \node[blue!70!black] at (6.25, 3.2) {\small glob.\ $224^2$};

  \draw[blue!70!black, thick] (5.0, 1.5) rectangle (7.5, 2.5);
  \node[blue!70!black] at (6.25, 2.0) {\small glob.\ $224^2$};

  % Widoki lokalne
  \draw[red!70!black, thick] (5.0, 0.5) rectangle (6.8, 1.3);
  \node[red!70!black] at (5.9, 0.9) {\small lok.\ $96^2$};

  \draw[red!70!black, thick] (5.0, -0.7) rectangle (6.8, 0.1);
  \node[red!70!black] at (5.9, -0.3) {\small lok.\ $96^2$};

  \node at (7.5, 0.2) {\small $\ldots$};

  % Etykiety
  \node[blue!70!black, anchor=west] at (7.7, 3.2) {\small $V_g = 2$};
  \node[red!70!black, anchor=west] at (7.1, -0.3) {\small $V_l = 6$};

  % Opis
  \node[anchor=west, align=left] at (8.2, 1.5) {\small Ten sam kot,\\
    \small różne wycinki\\
    \small \textbf{tej samej} klatki};
\end{tikzpicture}
\end{center}

Widoki globalne i~lokalne to \textbf{różne wycinki tej samej fotografii}.
Sieć uczy się: ``duży wycinek kota i~mały wycinek ucha
$\Rightarrow$ to~ten sam kot $\Rightarrow$ bliskie embeddingi.''
\end{tcolorbox}

\begin{tcolorbox}[colback=violet!5, colframe=violet!70!black, breakable,
  title={\textbf{Widoki --- wideo (np.\ chirurgia katarakty)}}]
Mamy \textbf{film}, nie pojedyncze zdjęcie.
Widoki pochodzą z~\textbf{różnych klatek}:

\begin{center}
\begin{tikzpicture}[>=stealth, scale=0.85]
  % Oś czasu
  \draw[->, thick] (0,0) -- (12,0) node[right] {\small czas};

  % Klatka anchor
  \fill[blue!20] (5.0, 0.3) rectangle (6.0, 2.8);
  \draw[blue!70!black, thick] (5.0, 0.3) rectangle (6.0, 2.8);
  \node[blue!70!black, rotate=90] at (5.5, 1.55) {\small anchor};
  \node[below, blue!70!black] at (5.5, 0.0) {\small klatka $t$};

  % Widoki globalne z anchora
  \draw[->, blue!70!black, thick] (6.1, 2.3) -- (7.5, 3.0);
  \draw[->, blue!70!black, thick] (6.1, 1.8) -- (7.5, 2.2);
  \node[blue!70!black, anchor=west] at (7.6, 3.0) {\small glob.\ 1 ($224^2$)};
  \node[blue!70!black, anchor=west] at (7.6, 2.2) {\small glob.\ 2 ($224^2$)};

  % Klatki sąsiednie
  \foreach \x/\lab in {2.5/$t{-}10$, 3.5/$t{-}5$, 7.5/$t{+}5$, 8.5/$t{+}8$, 9.5/$t{+}12$, 10.5/$t{+}15$} {
    \fill[red!15] (\x, 0.3) rectangle ({\x+0.7}, 2.0);
    \draw[red!70!black] (\x, 0.3) rectangle ({\x+0.7}, 2.0);
    \node[below, red!70!black, font=\tiny] at ({\x+0.35}, 0.2) {\lab};
  }

  % Strzałki od sąsiadów do lokalnych widoków
  \node[red!70!black, anchor=west] at (7.6, 1.2) {\small lok.\ 1--6 ($96^2$)};
  \draw[->, red!70!black] (8.85, 1.5) -- (7.5, 1.2);

  % Okno ±15
  \draw[<->, gray, thick] (2.5, -0.7) -- (10.5+0.7, -0.7);
  \node[gray, below] at (6.35, -0.7) {\small okno $\pm 15$ klatek};
\end{tikzpicture}
\end{center}

\begin{itemize}[leftmargin=2em]
  \item \textbf{Widoki globalne} ($V_g = 2$): duże kadrowania z~\textbf{klatki anchor}
        (jeden konkretny moment $t$) --- te same piksele, różne wycinki.
  \item \textbf{Widoki lokalne} ($V_l = 6$): małe kadrowania z~\textbf{sąsiednich klatek}
        w~oknie $\pm 15$ klatek wokół anchora
        (np.\ klatka z~$t - 5$, $t + 8$, $t + 12$, \ldots).
\end{itemize}

\textbf{Dlaczego to działa?}
W~filmie chirurgicznym klatki oddalone o~kilka--kilkanaście klatek
pokazują \textbf{tę samą scenę}: to samo narzędzie, ta sama tkanka,
lekko zmieniony kąt.
Sieć uczy się: ``klatka z~sekundy 5.0 i~klatka z~sekundy 5.3
przedstawiają ten sam etap operacji $\Rightarrow$ bliskie embeddingi.''

\medskip
\textbf{Przewaga nad obrazami}: w~wideo nie potrzebujemy sztucznych augmentacji
(obroty, zmiana kolorów), bo \textbf{sam upływ czasu} naturalnie generuje
różne widoki tej samej sceny --- zmienia się oświetlenie,
kąt mikroskopu, pozycja narzędzia.
To są \textbf{prawdziwe} transformacje, nie syntetyczne.
\end{tcolorbox}

\subsection{Czytanie wzoru (\ref{eq:pred_loss}) --- loss predykcyjny}

\begin{tcolorbox}[colback=orange!5, colframe=orange!70!black, breakable,
  title={\textbf{Wzór (\ref{eq:pred_loss}) --- co dokładnie liczy $\mathcal{L}_{\text{pred}}$}}]

\[
\mathcal{L}_{\text{pred}} = \frac{1}{V}\sum_{v'=1}^{V}
\left\|\boldsymbol{\mu}_n - \mathbf{z}_{n,v'}\right\|_2^2,
\qquad
\boldsymbol{\mu}_n \triangleq \frac{1}{V_g}\sum_{v=1}^{V_g} \mathbf{z}_{n,v}
\]

\textbf{Krok 1 --- oblicz centroid $\boldsymbol{\mu}_n$:}

Bierzemy tylko widoki \textbf{globalne} (duże kadrowania) i~liczymy ich średnią:
\[
\boldsymbol{\mu}_n = \frac{\mathbf{z}_{n,1} + \mathbf{z}_{n,2}}{2}
\quad\text{(bo $V_g = 2$)}
\]

Centroid to ``punkt środkowy'' --- najlepsza reprezentacja tego, co sieć ``myśli''
o~obrazie $n$ na podstawie globalnych widoków.

\textbf{Dlaczego tylko globalne?}
Globalne widoki widzą duży fragment obrazu $\Rightarrow$ mają najwięcej informacji
$\Rightarrow$ są najbardziej wiarygodnym punktem odniesienia.

\bigskip
\textbf{Krok 2 --- zmierz odległość każdego widoku od centroidu:}

Dla \textit{każdego} widoku $v'$ (globalnego i~lokalnego) liczymy:
\[
\left\|\boldsymbol{\mu}_n - \mathbf{z}_{n,v'}\right\|_2^2
= \sum_{i=1}^{d} \left(\mu_{n,i} - z_{n,v',i}\right)^2
\]

To jest zwykły \textbf{kwadrat odległości euklidesowej} w~$\mathbb{R}^d$
--- suma kwadratów różnic po każdej współrzędnej.

\bigskip
\textbf{Krok 3 --- uśrednij:}

$\frac{1}{V}\sum_{v'=1}^{V}$ --- średnia z~$V$ odległości.

\medskip
\textbf{Co to wymusza?}
Jeśli ten sam kot sfotografowany z~bliska i~z~daleka daje
\textit{bliskie} embeddingi (mała odległość od centroidu),
to sieć nauczyła się rozpoznawać ``kotowość'' niezależnie od kadrowania.
\end{tcolorbox}

\subsection{Konkretny przykład liczbowy}

Weźmy $B = 2$ obrazy, $V_g = 2$ widoki globalne, $V_l = 1$ widok lokalny
($V = 3$), $d = 2$ wymiary (zamiast 128, dla prostoty), $\lambda = 0.05$.

\medskip
\textbf{Embeddingi} (sieć wypluwa):
\[
\begin{array}{c|ccc}
 & v=1 \text{ (glob.)} & v=2 \text{ (glob.)} & v=3 \text{ (lok.)} \\
\hline
\text{obraz } n=1 & \mathbf{z}_{1,1} = (1.0,\; 0.5) & \mathbf{z}_{1,2} = (0.8,\; 0.7)
  & \mathbf{z}_{1,3} = (1.3,\; 0.3) \\
\text{obraz } n=2 & \mathbf{z}_{2,1} = (-0.6,\; 1.2) & \mathbf{z}_{2,2} = (-0.4,\; 0.8)
  & \mathbf{z}_{2,3} = (-0.8,\; 1.0)
\end{array}
\]

\textbf{Część B --- predykcja (czytamy wierszami):}

Centroidy z~widoków globalnych:
\begin{align*}
\boldsymbol{\mu}_1 &= \tfrac{1}{2}\bigl((1.0, 0.5) + (0.8, 0.7)\bigr) = (0.9,\; 0.6) \\
\boldsymbol{\mu}_2 &= \tfrac{1}{2}\bigl((-0.6, 1.2) + (-0.4, 0.8)\bigr) = (-0.5,\; 1.0)
\end{align*}

Odległości od centroidu (obraz 1):
\begin{align*}
\|\boldsymbol{\mu}_1 - \mathbf{z}_{1,1}\|^2 &= (0.9-1.0)^2 + (0.6-0.5)^2 = 0.01 + 0.01 = 0.02 \\
\|\boldsymbol{\mu}_1 - \mathbf{z}_{1,2}\|^2 &= (0.9-0.8)^2 + (0.6-0.7)^2 = 0.01 + 0.01 = 0.02 \\
\|\boldsymbol{\mu}_1 - \mathbf{z}_{1,3}\|^2 &= (0.9-1.3)^2 + (0.6-0.3)^2 = 0.16 + 0.09 = 0.25
\end{align*}
\[
\mathcal{L}_{\text{pred}}^{(1)} = \frac{0.02 + 0.02 + 0.25}{3} = 0.097
\]

Analogicznie dla obrazu 2: powiedzmy $\mathcal{L}_{\text{pred}}^{(2)} = 0.063$.

\[
\text{Część B} = \frac{1 - 0.05}{2}(0.097 + 0.063) = \frac{0.95}{2} \cdot 0.16 = 0.076
\]

\textbf{Część A --- regularyzacja (czytamy kolumnami):}

Dla widoku $v = 1$: zbieramy $\{\mathbf{z}_{1,1}, \mathbf{z}_{2,1}\} = \{(1.0, 0.5),\; (-0.6, 1.2)\}$
i~sprawdzamy SIGRegiem, czy to $\mathcal{N}(\mathbf{0}, \mathbf{I})$.
Powiedzmy $\mathrm{SIGReg}_1 = 0.31$.
Analogicznie dla $v = 2, 3$.

\[
\text{Część A} = \frac{0.05}{3}(0.31 + 0.28 + 0.45) = \frac{0.05}{3} \cdot 1.04 = 0.017
\]

\textbf{Pełny loss:}
\[
\mathcal{L}_{\text{LeJEPA}} = \underbrace{0.017}_{\text{SIGReg}} + \underbrace{0.076}_{\text{predykcja}} = 0.093
\]

\begin{keyinsight}[Dwa cele, jedna liczba]
Widzimy, że predykcja dominuje ($0.076$ vs $0.017$) --- tak ma być przy $\lambda = 0.05$.
Sieć skupia się na \textbf{semantycznej spójności} widoków (``ten sam kot $\Rightarrow$ bliskie wektory''),
a~SIGReg działa w~tle jako \textbf{strażnik} (``ale nie kolapsuj do jednego punktu --- rozpychaj się gaussowsko!'').
\end{keyinsight}

\subsection{Podsumowanie: regularyzacja vs predykcja}

\begin{center}
\renewcommand{\arraystretch}{1.3}
\begin{tabular}{lcc}
\toprule
& \textbf{Regularyzacja (SIGReg)} & \textbf{Predykcja} \\
\midrule
\textbf{Iteruje po} & widokach ($v$) & obrazach ($n$) \\
\textbf{Zbiera} & embeddingi jednego widoku z~$B$ obrazów & embeddingi jednego obrazu z~$V$ widoków \\
\textbf{Sprawdza} & ``czy rozkład $\approx \mathcal{N}(\mathbf{0}, \mathbf{I})$?'' & ``czy widoki się zgadzają?'' \\
\textbf{Chroni przed} & kolapsem (wszystko $\to$ jeden punkt) & chaosem (losowe embeddingi) \\
\textbf{Waga} & $\lambda = 0.05$ (5\%) & $1 - \lambda = 0.95$ (95\%) \\
\bottomrule
\end{tabular}
\end{center}


\clearpage
%% ============================================================
\section{Geometryczna intuicja}
%% ============================================================

\begin{warningbox}[Dlaczego kolaps jest katastrofalny?]
Jeśli $f_\theta(\mathbf{x}) = \mathbf{c}$ dla każdego $\mathbf{x}$:
\begin{itemize}
  \item Linear probe: $\mathbf{Z} = \mathbf{1}\mathbf{c}^\top \Rightarrow \mathrm{rank}(\mathbf{Z})=1 \Rightarrow$
        niemożliwa klasyfikacja,
  \item k-NN: $\|\mathbf{z}_i - \mathbf{z}_j\| = 0 \, \forall i,j \Rightarrow$
        wszyscy sąsiedzi są identyczni,
  \item Loss predykcyjny: $\mathcal{L}_{\text{pred}} = 0$ (trywialne minimum!).
\end{itemize}
\end{warningbox}

Izotropowy Gauss rozwiązuje ten problem:
\begin{enumerate}
  \item Embeddingi \textbf{wypełniają} przestrzeń $\mathbb{R}^K$ równomiernie,
  \item Żaden kierunek nie jest preferowany $\Rightarrow$ linear probe działa jednakowo dobrze
        dla \textit{dowolnego} zadania (niezależnie od orientacji granicy decyzyjnej),
  \item Maksymalna ``pojemność informacyjna'' — entropia Gaussa jest maksymalna
        wśród rozkładów o ustalonej wariancji:
        \begin{equation}
        H(\mathbf{z}) = \frac{K}{2}\ln(2\pi e) \quad \text{(maksimum entropii)}
        \end{equation}
\end{enumerate}


\clearpage
%% ============================================================
\section{Podsumowanie}
%% ============================================================

\begin{center}
\renewcommand{\arraystretch}{1.3}
\begin{tabular}{p{0.25\textwidth}p{0.65\textwidth}}
\toprule
\textbf{Pytanie} & \textbf{Odpowiedź} \\
\midrule
Jaki rozkład embeddingów? & $\mathcal{N}(\mathbf{0}, \mathbf{I}_K)$ — izotropowy Gauss \\
Dlaczego izotropowy? & Minimalizuje bias i wariancję linear probe (Lem.\ 1--2) \\
Dlaczego Gauss? & Minimalizuje ISB k-NN/kernel (Tw.\ 1) \\
Jak wymusić? & SIGReg: test Eppsa-Pulleya na losowych rzutach 1D \\
Ile hiperparametrów? & Jeden: $\lambda = 0.05$ \\
\bottomrule
\end{tabular}
\end{center}



\end{document}
